% implementation.tex

% structures.tex

\section{Structures de donnée}\label{structures}
Nous utilisons de nombreux graphes différents que ce soit pour stocker des informations
ou même pour vérifier les contraintes de certaines classes de règles. Ainsi nous avons
mis en place une structure générique, permettant de choisir le type de sommets et d'arcs,
ainsi que de coder différemment les ensembles d'arêtes (ou d'arcs) et de sommets de
manière différentes selon la situation, tout en fournissant les algorithmes de graphes
classiques.

\subsection{Règle}
Nous avons donc choisi d'utiliser la représentation graphique des règles 
telle que définie dans la section \ref{def_representation_regle}.
Ainsi une structure de graphe biparti non orienté a été mise en place, l'ensemble des
arêtes est codé par une liste de voisinage pour chaque sommet.
En effet l'opération de parcours des voisins doit être la plus efficace possible, étant
donné qu'elle nécessaire pour grand nombre d'algorithmes ainsi pour la création des
atomes.
Le graphe étant biparti, l'ensemble de sommets est divisé en deux parties permettant des
accès rapides à l'une comme à l'autre, mais perdant de l'efficacité sur certaines
opérations peu fréquentes telles que la suppression d'un sommet.
Les arêtes sont quant à elle typées par un entier, permettant de connaître la position
d'un terme dans un atome, les termes devraient être de préférence ajoutés dans l'ordre
(et après les prédicats) de manière à faciliter leur parcours (et à optimiser le temps
nécessaire à la création de la règle).

Notons tout de même qu'une règle est une spécialisation d'une conjonction d'atomes, et
que c'est cette-dernière qui est en charge de la gestion du graphe hormis la séparation
entre l'hypothèse et la conclusion.

\subsection{Graphe de dépendances des règles}
Ensuite, le graphe de dépendances des règles est quant à lui un graphe orienté dont les
arcs sont également codés par listes de voisinage et ne possèdent pas de poids.
Il est évident que les arcs d'un graphe de
dépendances ont plus intérêt à être implémentés de cette manière puisque l'objectif de
construire un tel graphe est de connaître rapidement de quel sommet dépend quel autre.

\subsection{Graphe de dépendances des positions}
Enfin le graphe de dépendances des positions utilisés pour vérifier l'appartenance à
certaines classes de règles (faiblement acyclique (\ref{classe_wa}) et faiblement sticky
(\ref{classe_ws})) est aussi un graphe orienté, mais ici la fonction de poids sur les
arcs permet de savoir si un arc est {\em spécial} ou non.
Les sommets représentent
les positions des prédicats et sont stockées à la suite. De plus, une table de hachage gère
l'accès à la première position d'un prédicat en fonction de son nom, 
ainsi les opérations d'accès sont
aussi légères que possible, et ce type de graphe n'étant conservé en mémoire que le
temps de la vérification des contraintes, l'espace supplémentaire utilisé par la table est
négligeable.



\section{GRDAnalyser}\label{impl_grd_analyser}
Les règles (les données) sont donc stockées directement dans le graphe de dépendances des
règles dont une instance est encapsulée dans le \grdanalyser.
De plus le \grdanalyser est en fait constitué de deux parties distinctes supplémentaires
: la première est en charge de la détermination des classes concrètes, tandis que la
seconde vérifie si la base est décidable et combine les classes
abstraites de manière à savoir quels algorithmes utiliser.

\section{Détermination d'une classe concrète}\label{impl_classe_concrete}
En section \ref{classes_concretes} nous avons défini de nombreuses classes de règles
qu'il faut donc pouvoir déterminer.
Pour cela, nous avons déclaré une interface de fonction \code{DecidableClassCheck}
fournissant une méthode renvoyant une étiquette à partir d'un ensemble de règles.
L'analyseur de classes concrètes contient une liste des contraintes à tester, et lors de
son exécution, il vérifie tout d'abord l'ensemble complet des règles sur chacune de
celles-ci, puis ensuite sur chaque composante fortement connexe du graphe de dépendances.

\section{Combinaison des classes abstraites}\label{impl_combine}
Comme expliqué plus en détails dans la section \ref{combine} il est ensuite
nécessaire de vérifier si l'ensemble est bel et bien décidable.
L'analyseur de classes abstraites regarde donc tout d'abord si l'ensemble des règles est
étiqueté par une classe concrète, si tel est le cas, une des approches pour répondre à
une requête est donc d'exécuter l'algorithme correspondant.
Si ce n'est pas le cas, il associe la valeur 1 à l'étiquette $FES$, 
2 à $GBTS$ et 3 à $FUS$ de manière à avoir un ordre sur celles-ci, 
puis il effectue un parcours  en largeur du graphe des 
composantes fortement connexes à partir de l'ensemble des sources de celui-ci, attribuant à
chaque sommet découvert (qu'il soit déjà traité ou non) la plus petite étiquette
fournie par ses classes concrètes et supérieure à celle de son prédécesseur ou 0 si 
ce n'est pas possible.
Une fois cette opération effectuée, si tous les sommets sont étiquetés par des valeurs 
strictement positives, l'ensemble de règles est décidable.


\section{Formats de fichiers}\label{formats_fichiers}
Le graphe de dépendances des règles est capable de charger une base à partir d'un
fichier, celui-ci devant être écrit dans un format spécifique : chaque ligne doit être
une règle de la forme suivante :\\
$atome_1;atome_2;...;atome_n-->atome_c$\\
avec $n$ le nombre d'atomes dans l'hypothèse et $atome_c$ l'unique atome de la conclusion et
où chaque atome $i$ est écrit :
$p_i(t_{i1},t_{i2},...,t_{ik})$\\
Les termes sont interprêtés comme des constantes s'ils sont encadrés par des simple
guillemets.\\%TODO
Par exempe la règle $\forall x ( \rightarrow (\exists z ) )$ doit être écrite : \\
{\em }

En plus du format interne ci-dessus, il est également possible de fournir un fichier
Datalog (.dtg) qui ne contient que des règles à hypothèse atomique. Chaque ligne est soit
une règle, soit un commentaire auquel cas elle doit débuter par //.
Ici les règles sont sous le format suivant :\\
$[!]atome_c$ :- $atome_h$.\\
Le point d'explamation est utilisé pour signaler la négation d'une conclusion, celle-ci
sera convertie en une règle contenant ses deux atomes actuels dans son hypothèse et ayant
une conclusion au prédicat spécial {\em ABSURD}.
De plus, les termes sont maintenant considérés comme des variables s'ils commencent par
un point d'interrogation et comme des constantes sinon.\\%TODO
Ainsi la ligne du fichier correspondant à la règle $\forall x ( \rightarrow (\exists z ) )$ 
doit être :\\
{\em .}

