% implementation.tex

\section{Conjonction d'atomes}\label{def_graphe}
%TODO
TODO => à adapter à la section

Par la suite on supposera que le graphe $G_{K}$ associé à la conjonction $K$ 
est impl\'ement\'e par deux ensembles de sommets (les atomes et les termes), 
ainsi que par une liste de voisinage pour chaque sommet.
De plus, chaque sommet peut avoir connaissance de son type de contenu parmis : 
$atome$, $terme$, $constante$, $\forall-var$, $\exists-var$
en temps constant.

On note $n$ (resp. $t$) le nombre d'atomes (resp. de termes) dans $K$. 

Op\'erations (et temps d'ex\'ecution associ\'e) :
\begin{itemize}
	\item Parcourir les sommets $atomes$ : $n$.
	\item Parcourir tous les sommets : $n + t$.
	\item Acc\'eder au $i^{eme}$ voisin d'un sommet : $constant$.
\end{itemize}


\section{Formats de fichiers}\label{formats_fichiers}
Format interne : p(x,y);q('a')-->r(z,'a',x)
Format Datalog : [!]head :- body

