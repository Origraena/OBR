% structures.tex

\section{Structures de donnée}\label{structures}
Nous utilisons de nombreux graphes différents que ce soit pour stocker des informations
ou même pour vérifier les contraintes de certaines classes de règles. Ainsi nous avons
mis en place une structure générique, permettant de choisir le type de sommets et d'arcs,
ainsi que de coder différemment les ensembles d'arêtes (ou d'arcs) et de sommets de
manière différentes selon la situation, tout en fournissant les algorithmes de graphes
classiques.

\subsection{Règle}
Nous avons donc choisi d'utiliser la représentation graphique des règles 
telle que définie dans la section \ref{def_representation_regle}.
Ainsi une structure de graphe biparti non orienté a été mise en place, l'ensemble des
arêtes est codé par une liste de voisinage pour chaque sommet.
En effet l'opération de parcours des voisins doit être la plus efficace possible, étant
donné qu'elle nécessaire pour grand nombre d'algorithmes ainsi pour la création des
atomes.
Le graphe étant biparti, l'ensemble de sommets est divisé en deux parties permettant des
accès rapides à l'une comme à l'autre, mais perdant de l'efficacité sur certaines
opérations peu fréquentes telles que la suppression d'un sommet.
Les arêtes sont quant à elle typées par un entier, permettant de connaître la position
d'un terme dans un atome, les termes devraient être de préférence ajoutés dans l'ordre
(et après les prédicats) de manière à faciliter leur parcours (et à optimiser le temps
nécessaire à la création de la règle).

Notons tout de même qu'une règle est une spécialisation d'une conjonction d'atomes, et
que c'est cette-dernière qui est en charge de la gestion du graphe hormis la séparation
entre l'hypothèse et la conclusion.

\subsection{Graphe de dépendances des règles}
Ensuite, le graphe de dépendances des règles est quant à lui un graphe orienté dont les
arcs sont également codés par listes de voisinage et ne possèdent pas de poids.
Il est évident que les arcs d'un graphe de
dépendances ont plus intérêt à être implémentés de cette manière puisque l'objectif de
construire un tel graphe est de connaître rapidement de quel sommet dépend quel autre.

\subsection{Graphe de dépendances des positions}
Enfin le graphe de dépendances des positions utilisés pour vérifier l'appartenance à
certaines classes de règles (faiblement acyclique (\ref{classe_wa}) et faiblement sticky
(\ref{classe_ws})) est aussi un graphe orienté, mais ici la fonction de poids sur les
arcs permet de savoir si un arc est {\em spécial} ou non.
Les sommets représentent
les positions des prédicats et sont stockées à la suite. De plus, une table de hachage gère
l'accès à la première position d'un prédicat en fonction de son nom, 
ainsi les opérations d'accès sont
aussi légères que possible, et ce type de graphe n'étant conservé en mémoire que le
temps de la vérification des contraintes, l'espace supplémentaire utilisé par la table est
négligeable.

