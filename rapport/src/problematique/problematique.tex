% problematique.tex

Malgré le fait que de manière général, il n'existe aucun algorithme permettant de
répondre à ce problème, certaines règles peuvent entrer dans des catégories (qui seront
nommées {\em classes de règles}) qui en ajoutant des contraintes sur la forme des règles
s'assurent que le problème soit décidable.
Selon quelles contraintes sont satisfaites, il est nécessaire d'appliquer différentes
méthodes de réponse sur différents sous-ensembles des règles.

L'objectif de ce TER est donc d'implémenter un outil permettant d'analyser une base de règles 
afin de construire son graphe
de dépendances associés, de déterminer quelles contraintes sont satisfaites, sur quel
sous-ensemble, et si la base est décidable d'en déduire quels algorithmes utiliser 
sur chacun d'eux. 
De plus, cet outil devra pouvoir charger des bases à partir de fichiers, ainsi que les y
écrire, et être suffisamment modulable pour permettre l'ajout de nouvelles vérifications
de contrainte.

