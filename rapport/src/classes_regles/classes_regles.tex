% classes_regles.tex

\section{Classes abstraites}\label{classes_abstraites}
Trois classes abstraites ont été définies, chacune permettant l'usage de certains
algorithmes sur l'ensemble de règles considéré.
Elles sont dites {\em abstraites} puisque déterminer si un ensemble de règles appartient
à l'une de ces classes est un problème non décidable.
De plus elles sont incomparables entre elles, et non exclusives. 

\subsection{Finite Expansion Set}\label{classes_abstraites_fus}
La première classe est définie comme assurant la finition des algorithmes de chaînage
avant. Ainsi tout ensemble de règles appartenant à cette classe peut être utilisé pour
les dérivations de ces algorithmes.
Dans le cas de certaines classes, il est par contre nécessaire d'ajouter des conditions
d'arrêt, celles-ci sont détaillées plus loin.

\subsection{Finite Unification Set}\label{classes_abstraites_fes}
La deuxième classe abstraite quant à elle assure la finition des algorithmes de chaînage
arrière. De la même manière que précedemment il est parfois nécessaire de modifier les
conditions d'arrêt.

\subsection{Bounded Treewidth Set}\label{classes_abstraites_bts}
La dernière définit les ensembles de règles où la production de nouvelles règles suit la
forme d'un arbre.
Cette classe ne permet pas l'utilisation direct d'algorithmes, mais par contre la classe
abstraite Greedy Bounded Treewidth qui est une spécialisation de celle-ci, s'assure que
le chaînage avant s'exécute en temps fini, et ce dès qu'un algorithme glouton de ... est
utilisé.


\section{Classes concrètes}\label{classes_concretes}

\subsection{Acyclicité du graphe de dépendance des règles}\label{classe_agrd}
Le seul fait que le graphe de dépendance soit sans circuit suffit à certifier que le chaînage
avant et arrière s'exécutent en temps fini, impliquant que si cette contrainte est satisfaite,
l'ensemble des règles appartient à \fes et à \fus.

\subsection{Domaine restreint}\label{classe_dr}
Un ensemble de règles satisfait cette contrainte si tous les atomes de leur conclusion
contiennent soit toutes les variables de l'hypothèse, soit aucune.
Dans le cas des règles à conclusion atomique, cela revient à s'assurer que la frontière
de chaque règle est soit égale à 0 soit au nombre de variables universelles.
Cette contrainte est suffisante pour que l'ensemble de règles appartienne à \fus.

%%%%	\subsection{}\label{classe_}
%%%%	\subsection{}\label{classe_}
%%%%	\subsection{}\label{classe_}
%%%%	\subsection{}\label{classe_}
%%%%	\subsection{}\label{classe_}
%%%%	\subsection{}\label{classe_}
%%%%	\subsection{}\label{classe_}
%%%%	\subsection{}\label{classe_}
%%%%	\subsection{}\label{classe_}

