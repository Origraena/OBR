% classes_regles.tex

\section{Classes abstraites}\label{classes_abstraites}
Trois classes abstraites ont été définies, chacune permettant l'usage de certains
algorithmes sur l'ensemble de règles considéré.
Elles sont dites {\em abstraites} puisque déterminer si un ensemble de règles appartient
à l'une de ces classes est un problème non décidable.
De plus elles sont incomparables entre elles, et non exclusives. 

\subsection{Finite Expansion Set}\label{classes_abstraites_fus}
La première classe est définie comme assurant la finition des algorithmes de chaînage
avant. Ainsi tout ensemble de règles appartenant à cette classe peut être utilisé pour
les dérivations de ces algorithmes.
Dans le cas de certaines classes, il est par contre nécessaire d'ajouter des conditions
d'arrêt, celles-ci sont détaillées plus loin.

\subsection{Finite Unification Set}\label{classes_abstraites_fes}
La deuxième classe abstraite quant à elle assure la finition des algorithmes de chaînage
arrière. De la même manière que précedemment il est parfois nécessaire de modifier les
conditions d'arrêt.

\subsection{Bounded Treewidth Set}\label{classes_abstraites_bts}
La dernière définit les ensembles de règles où la production de nouvelles règles suit la
forme d'un arbre.
Cette classe ne permet pas l'utilisation direct d'algorithmes, mais par contre la classe
abstraite Greedy Bounded Treewidth qui est une spécialisation de celle-ci, s'assure que
le chaînage avant s'exécute en temps fini, et ce dès qu'un algorithme glouton de ... est
utilisé.


\section{Classes concrètes}\label{classes_concretes}
% concretes.tex

%TODO REFS

\subsection{Acyclicité du graphe de dépendance des règles}\label{classe_agrd}
Le seul fait que le graphe de dépendance soit sans circuit suffit à certifier que le chaînage
avant et arrière s'exécutent en temps fini, impliquant que si cette contrainte est satisfaite,
l'ensemble des règles appartient à \fes et à \fus.

\subsection{Faiblement acyclique}\label{classe_wa}
Cette classe est quant à elle un peu particulière puisqu'elle demande la génération d'une autre
structure de graphes et qu'elle s'applique directement sur un ensemble de règles et pas
uniquement sur chacune des règles indépendamment des autres.

On crée donc un graphe de dépendances des positions dont les sommets sont les positions
des prédicats et dont la construction des arcs est la suivante : pour chaque variable
$x$ d'une règle $R$ apparaissant dans l'hypothèse en position $p_i$, 
si $x$ appartient à la frontière de $R$ alors il existe un arc de $p_i$ vers chacune des
positions $r_j$ de la conclusion de $R$ dans laquelle apparaît $x$, de plus pour chacune
des variables existentielles apparaîssant en position $q_k$ il existe un arc {\em
spécial} de $p_i$ vers $q_k$.

Si le graphe de dépendances des positions associés à un ensemble de règle ne contient
aucun circuit passant par un arc {\em spécial} alors il est dit {\em faiblement
acyclique} et appartient à \fes.

Voir \cite{nonguarded} pour les détails et les preuves. 

\subsection{Sticky}\label{classe_sticky}
Un ensemble de règles satisfait cette contrainte si certaines de ses variables (marquées)
n'apparaissent pas plusieurs fois dans l'hypothèse d'une de ses règles.

Plus précisemment, on marque tout d'abord les variables en deux étapes :\\
\begin{itemize}
\item premièrement, pour chaque règle $R$, et pour chacune des variables $x$ de l'hypothèse 
$H$ de $R$, si $x$ n'apparait pas dans la conclusion de $R$ alors on marque chaque 
occurence de $x$ dans $H$.
\item Deuxièmement, pour chaque règle $R$ si une variable marquée apparaît en position $p_i$
alors pour chaque règle $R'$ (incluant $R = R'$), et pour chacune des variables $x$
apparaîssant en position $p_i$ dans la conclusion de $R'$, on marque chaque occurence de
$x$ dans l'hypothèse de $R'$.\\
\end{itemize}

Ensuite si aucune règle ne contient plusieurs occurence d'une variable marquée dans son
hypothèse, l'ensemble est dit {\em sticky}, et assure la finition des algorithmes basés
sur le chaînage arrière, cette classe appartient donc à la classe abstraite \fus (voir
\cite{nonguarded}).

\subsection{Faiblement sticky}\label{classe_ws}
Cette classe est une généralisation des deux précédentes qui malheureusement n'appartient
à aucune des classes abstraites citées dans la section précédente (cf \cite{nonguarded}).

Tout comme la vérification de l'appartenance à la classe {\em faiblement acyclique
(\ref{classe_wa})}, il est nécessaire de construire le graphe de dépendances des
positions. De plus, on dit qu'une position (un sommet de ce graphe) est {\em sûre} si
elle n'apparait dans aucun circuit contenant un arc {\em spécial}, et on marque chacune
des variables en suivant le même algorithme que pour la classe {\em sticky
(\ref{classe_sticky})}.

On peut maintenant définir un ensemble de règles {\em faiblement sticky} comme ne
contenant que des variables sûres ou non marquées.

\subsection{Guardée}\label{classe_guarded}
Une règle gardée est définie comme étant une règle 
dont un atome de son hypothèse (nommé {\em garde}) contient toutes les variables de celle-ci.
Si toutes les règles de l'ensemble contiennent un garde, alors l'ensemble appartient à
\gbts.

\subsection{Frontière gardée}\label{classe_frg}
On dit qu'une règle a une {\em frontière gardée} si un atome de son hypothèse possède toutes
les variables de la frontière.
On peut remarquer que cette classe est une généralisation de la précédente.
Et dans le cas où toutes les règles de l'ensemble possède cette propriété, celui-ci
appartient également à \gbts.

\subsection{Frontière-1}\label{classe_fr1}
Cette classe contient les règles dont la frontière est de taille 1,
elle est donc une spécialisation des règles à frontière gardée, 
Ainsi un ensemble de règles satisfaisant cette propriété appartient à \gbts.

\subsection{Hypothèse atomique}\label{classe_ah}
Les règles ne contenant que des hypothèses atomiques s'assurent que la règle est {\em
gardée (\ref{classe_guarded})}, et de plus assurent que les algorithmes basés sur le 
chaînage arrière se terminent en temps fini.
Donc si toutes les règles d'un ensemble sont à hypothèse atomique alors celui-ci est \gbts et
\fus.

\subsection{Domaine restreint}\label{classe_dr}
Une règle satisfait cette contrainte si tous les atomes de sa conclusion
contiennent soit toutes les variables de l'hypothèse, soit aucune.
Dans le cas des règles à conclusion atomique, cela revient à s'assurer que la frontière
de chaque règle est soit égale à 0 soit au nombre de variables universelles.
Cette contrainte est suffisante pour que l'ensemble de règles appartienne à \fus.


\subsection{Règle déconnectée}\label{classe_disc}
Une règle est dite déconnectée si sa frontière est vide, cette classe est donc une
spécialisation des règles à {\em domaine restreint (\ref{classe_dr})}, à {\em frontière
gardée (\ref{classe_frg})} et {\em faiblement acyclique (\ref{classe_wa})}.
Ce type de règle n'est pas très utilisé étant donné que seules des constantes sont
partagées entre l'hypothèse et la conclusion ce qui limite leur usage, mais elles ont
l'avantage d'être à la fois \fes, \gbts et \fus.

\subsection{Règles universelles}\label{classe_rr}
Les règles ne contenant aucune variable existentielle ($vars(C) \subseteq vars(H)$) forment bien entendu
un ensemble décidable. Celles-ci sont à la fois \fes et \gbts. On peut également
remarquer que toute ensemble de règles universelles est également {\em faiblement
acyclique (\ref{classe_wa})}.



\section{Combinaisons}\label{combine}
Les composantes fortement connexes du graphe de dépendances sont utilisées pour le 
{\em découper} de façon à attribuer des étiquettes différentes à celles-ci en fonction
des classes concrètes auxquelles son ensemble de règles appartient.

De plus une fois chaque sous ensemble étiqueté, il faut encore vérifier que celles-ci
sont compatibles entre elles.

Pour cela, nous définissons le graphe orienté des composantes fortement connexes associé
tel que son ensemble de sommets est l'ensemble des composantes du graphe, et
qu'il existe un arc entre deux composantes $C_i$ et $C_j$ si et seulement s'il existe un
arc d'un sommet de $C_i$ vers un sommet de $C_j$.
Par définition des composantes fortement connexes, ce graphe est évidemment sans
circuit.
%et les arcs de celui-ci influent directement sur la décidabilité de l'ensemble
%de règles.

On dit que $C_i$ {\em précède} $C_j$ s'il n'existe aucun arc de $C_j$ vers $C_i$, et on
note cette relation $C_i \triangleright C_j$.

De plus,
on associe à chaque $C_i$ une étiquette qui déterminera la classe abstraite considérée pour
cette composante.

pour que l'ensemble de règles soit décidable est la suivante :\\
$\{C_i : etiquette(C_i) = FES\} \triangleright \{C_i : etiquette(C_i) = GBTS\} \triangleright
\{C_i : etiquette(C_i) = FUS\}$
C'est à dire qu'aucune règle \fes ne doit dépendre d'une règle \fus ou \gbts, et
qu'aucune règle \gbts ne doit dépendre d'une règle \fus.

En effet les algorithmes de chaînage arrière par exemple réécrivent la requête jusqu'à ce
qu'elle corresponde à la base, tandis que ceux avant ajoutent des faits jusqu'à obtenir
la requête. Il est donc évident que si une composante n'accepte que le chaînage arrière,
il ne doit exister aucune règle de celle-ci de laquelle dépende une règle de la
composante acceptant uniquement le chaînage avant (si tel était le cas, cette règle ne
serait jamais déclenchée).


