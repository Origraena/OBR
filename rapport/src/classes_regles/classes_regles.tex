% classes_regles.tex

\section{Classes abstraites}\label{classes_abstraites}
Trois classes abstraites ont été définies, chacune permettant l'usage de certains
algorithmes sur l'ensemble de règles considéré.
Elles sont dites {\em abstraites} puisque déterminer si un ensemble de règles appartient
à l'une de ces classes est un problème non décidable. En effet ces classes n'imposent
aucune contrainte vérifiable.
De plus elles sont incomparables entre elles, et non exclusives. 

\subsection{Finite Expansion Set}\label{classes_abstraites_fus}
La première classe est définie comme assurant la finition des algorithmes de chaînage
avant. Ainsi tout ensemble de règles appartenant à cette classe peut être utilisé pour
les dérivations de ces algorithmes.

\subsection{Bounded Treewidth Set}\label{classes_abstraites_bts}
La deuxième définit quant à elle 
les ensembles de règles où la production de nouveaux faits suit la
forme d'un arbre de largeur bornée.
Cette classe ne permet pas l'utilisation direct d'algorithmes, mais par contre la classe
abstraite Greedy Bounded Treewidth qui est une spécialisation de celle-ci, s'assure que
le chaînage avant s'exécute en temps fini, et ce via un algorithme glouton permettant de
construire la décomposition de l'arbre et en utilisant un algorithme d'arrêt spécifique.

\subsection{Finite Unification Set}\label{classes_abstraites_fes}
Enfin la dernière classe abstraite assure la finition des algorithmes de chaînage
arrière utilisant les méthodes en {\em largeur} qui ne garderait que les faits {\em le plus
généraux} lors de leur génération (voir \cite{}).


\section{Classes concrètes}\label{classes_concretes}
Comme expliqué dans la section précédente, il est impossible de déterminer à coup sûr 
si un ensemble
de règles appartient à l'une de ces classes, par contre de nombreuses classes dites
{\em concrètes }
ont été définies. Chacune d'entre elle, en ajoutant des contraintes sur la forme des
règles voir sur l'ensemble des règles, 
permet de s'assurer que celui-ci appartient à certaines des classes
abstraites.

% concretes.tex

%TODO REFS

\subsection{Acyclicité du graphe de dépendance des règles}\label{classe_agrd}
Le seul fait que le graphe de dépendance soit sans circuit suffit à certifier que le chaînage
avant et arrière s'exécutent en temps fini, impliquant que si cette contrainte est satisfaite,
l'ensemble des règles appartient à \fes et à \fus.

\subsection{Faiblement acyclique}\label{classe_wa}
Cette classe est quant à elle un peu particulière puisqu'elle demande la génération d'une autre
structure de graphes et qu'elle s'applique directement sur un ensemble de règles et pas
uniquement sur chacune des règles indépendamment des autres.

On crée donc un graphe de dépendances des positions dont les sommets sont les positions
des prédicats et dont la construction des arcs est la suivante : pour chaque variable
$x$ d'une règle $R$ apparaissant dans l'hypothèse en position $p_i$, 
si $x$ appartient à la frontière de $R$ alors il existe un arc de $p_i$ vers chacune des
positions $r_j$ de la conclusion de $R$ dans laquelle apparaît $x$, de plus pour chacune
des variables existentielles apparaîssant en position $q_k$ il existe un arc {\em
spécial} de $p_i$ vers $q_k$.

Un des sommets de ce graphe (et donc une position de prédicat) est dite {\em de rang
fini} s'il n'existe aucun circuit contenant passant par un arc {\em spécial} et par ce
sommet.

Si toutes les positions de prédicat sont de {\em rang fini} alors l'ensemble de règles
est dit {\em faiblement acyclique} et appartient à \fes (\cite{nonguarded}).

\subsection{Sticky}\label{classe_sticky}
Un ensemble de règles satisfait cette contrainte si certaines de ses variables (marquées)
n'apparaissent pas plusieurs fois dans l'hypothèse d'une de ses règles.

Plus précisemment, on marque tout d'abord les variables en deux étapes :\\
\begin{itemize}
\item premièrement, pour chaque règle $R$, et pour chacune des variables $x$ de l'hypothèse 
$H$ de $R$, si $x$ n'apparait pas dans la conclusion de $R$ alors on marque chaque 
occurence de $x$ dans $H$.
\item Deuxièmement, pour chaque règle $R$ si une variable marquée apparaît en position $p_i$
alors pour chaque règle $R'$ (incluant $R = R'$), et pour chacune des variables $x$
apparaîssant en position $p_i$ dans la conclusion de $R'$, on marque chaque occurence de
$x$ dans l'hypothèse de $R'$.\\
\end{itemize}

Ensuite si aucune règle ne contient plusieurs occurence d'une variable marquée dans son
hypothèse, l'ensemble est dit {\em sticky}, et assure la finition des algorithmes basés
sur le chaînage arrière, cette classe appartient donc à la classe abstraite \fus (voir
\cite{nonguarded}).

\subsection{Guardée}\label{classe_guarded}
Une règle gardée est définie comme étant une règle 
dont un atome de son hypothèse (nommé {\em garde}) contient toutes les variables de celle-ci.
Si toutes les règles de l'ensemble contiennent un garde, alors l'ensemble appartient à
\gbts (\cite{walking}).

\subsection{Frontière gardée}\label{classe_frg}
On dit qu'une règle a une {\em frontière gardée} si un atome de son hypothèse possède toutes
les variables de la frontière.
On peut remarquer que cette classe est une généralisation de la précédente.
Et dans le cas où toutes les règles de l'ensemble possède cette propriété, celui-ci
appartient également à \gbts (\cite{walking}).

\subsection{Frontière-1}\label{classe_fr1}
Cette classe contient les règles dont la frontière est de taille 1,
elle est donc une spécialisation des règles à frontière gardée, 
Ainsi un ensemble de règles satisfaisant cette propriété appartient à \gbts.

\subsection{Hypothèse atomique}\label{classe_ah}
Les règles ne contenant que des hypothèses atomiques s'assurent que la règle est {\em
gardée (\ref{classe_guarded})}, et de plus assurent que les algorithmes basés sur le 
chaînage arrière se terminent en temps fini.
Donc si toutes les règles d'un ensemble sont à hypothèse atomique alors celui-ci est \gbts et
\fus (\cite{extending}).

\subsection{Domaine restreint}\label{classe_dr}
Une règle satisfait cette contrainte si tous les atomes de sa conclusion
contiennent soit toutes les variables de l'hypothèse, soit aucune.
Dans le cas des règles à conclusion atomique, cela revient à s'assurer que la frontière
de chaque règle est soit égale à 0 soit au nombre de variables universelles.
Cette contrainte est suffisante pour que l'ensemble de règles appartienne à \fus
(\cite{extending}).


\subsection{Règle déconnectée}\label{classe_disc}
Une règle est dite déconnectée si sa frontière est vide, cette classe est donc une
spécialisation des règles à {\em domaine restreint (\ref{classe_dr})}, à {\em frontière
gardée (\ref{classe_frg})} et {\em faiblement acyclique (\ref{classe_wa})}.
Ce type de règle n'est pas très utilisé étant donné que seules des constantes sont
partagées entre l'hypothèse et la conclusion ce qui limite leur usage, mais elles ont
l'avantage d'être à la fois \fes, \gbts et \fus puisque qu'une règle déconnectée n'a
besoin de s'appliquer qu'une seule fois (\cite{ontological11}).

\subsection{Règles universelles}\label{classe_rr}
Les règles ne contenant aucune variable existentielle ($vars(C) \subseteq vars(H)$) forment bien entendu
un ensemble décidable. Celles-ci sont à la fois \fes et \gbts. On peut également
remarquer que toute ensemble de règles universelles est également {\em faiblement
acyclique (\ref{classe_wa})}.

\subsection{Faiblement sticky}\label{classe_ws}
Cette classe est une généralisation des deux précédentes qui malheureusement n'appartient
à aucune des classes abstraites citées dans la section précédente (cf \cite{nonguarded}).

Tout comme la vérification de l'appartenance à la classe {\em faiblement acyclique
(\ref{classe_wa})}, il est nécessaire de construire le graphe de dépendances des
positions. De plus, on dit qu'une position (un sommet de ce graphe) est {\em de rang fini} si
elle n'apparait dans aucun circuit contenant un arc {\em spécial}, et on marque chacune
des variables en suivant le même algorithme que pour la classe {\em sticky
(\ref{classe_sticky})}.

On peut maintenant définir un ensemble de règles {\em faiblement sticky} comme ne
contenant que des variables sûres ou non marquées.




\section{Schéma d'inclusion des classes de règles}
Sur la figure \ref{fig_recap_classes}, les flèches représentent la spécialisation tandis
que les couleurs l'appartenance aux classes abstraites. De plus, seules les classes cités
ci-dessus apparaissent sur celle-ci.
% schema.tex

\begin{figure}[H]
\begin{center}
\begin{tikzpicture}[
	auto,
	node distance=1.5cm,
	inner sep=1mm
]
	\draw [blue] (-3.8,-5.5) -- (3.2,-5.5) -- (3.2,-2.5) -- (-3.8,-2.5) -- (-3.8,-5.5);
	\draw [red] (0.3,1) -- (3.7,1) -- (3.7,-5.7) -- (0.3,-5.7) -- (0.3,1);
	\draw [green] (4,1) -- (9,1) -- (9,-4.5) -- (-2.2,-4.5) -- (-2.2,-3.5) --
	(0.5,-3.5) -- (0.5,-2) -- (4,-2) -- (4,1);

	\node [blue] (fes) at (-3,-5) {FES};
	\node [red] (fus) at (1,0.5) {FUS};
	\node [green] (gbts) at (8,0.5) {GBTS};

	% terms
	\node [class]	(ws)		at (-2,0)	{faiblement sticky};
	\node [class]	(dr)		at (2,-1)	{domaine restreint};
	\node [class]	(agrd)		at (2,-5)	{grd acyclique};
	\node [class]	(fg)		at (6,0)	{frontière guardée};
	\node [class]	(sticky)	at (2,0)	{sticky}
		edge[classedge]	(ws);
	\node [class]	(wa)		at (-2,-3)	{faiblement acyclique}
		edge[classedge]	(ws);
	\node [class]	(disc)		at (2,-3)	{déconnectée}
		edge[classedge] (wa)
		edge[classedge] (dr)
		edge[classedge] (fg);
	\node [class]	(guarded)	at (6,-2)	{guardée}
		edge[classedge]	(fg);
	\node [class]	(fr1)		at (8,-2)	{frontière 1}
		edge[classedge] (fg);
	\node [class]	(ah)		at (6,-4)	{hypothèse atomique}
		edge[classedge] (guarded);
	\node [class]	(rr)		at (-1,-4)	{universelles}
		edge[classedge] (wa);
\end{tikzpicture}
\end{center}
\caption{Schéma récapitulatif des classes de règles}
\label{fig_recap_classes}
\end{figure}



\section{Combinaisons}\label{combine}
Si l'ensemble des règles appartient à une des classes concrètes, alors il suffit
d'appliquer les algorithmes correspondant pour répondre aux requêtes. Toutefois, il est
possible que ce ne soit pas le cas, et il faut donc pouvoir {\em découper} l'ensemble des
règles de manière à appliquer des méthodes de réponse différentes en fonction du
sous-ensemble.

Dans cette optique nous utilisons les composantes fortement connexes du graphe de
dépendances
de façon à attribuer des étiquettes différentes à celles-ci en fonction
des classes concrètes auxquelles leurs ensembles de règles appartiennent.

Puis une fois chaque sous ensemble étiqueté, il faut encore vérifier que celles-ci
sont compatibles entre elles.
Pour cela, nous définissons le graphe orienté des composantes fortement connexes associé
$G_{C} = (V_C,E_C)$
tel que son ensemble de sommets est l'ensemble des composantes du graphe, et
qu'il existe un arc entre deux composantes $C_i$ et $C_j$ si et seulement s'il existe un
arc d'un sommet de $C_i$ vers un sommet de $C_j$.
Par définition des composantes fortement connexes, ce graphe est évidemment sans
circuit.
%et les arcs de celui-ci influent directement sur la décidabilité de l'ensemble
%de règles.

On dit que $C_i$ {\em précède} $C_j$ s'il n'existe aucun arc de $C_j$ vers $C_i$, et on
note cette relation $C_i \triangleright C_j$.

De plus, on définit une fonction $etiquette : V_C \rightarrow \{FES,GBTS,FUS\}$ qui
associe à chaque $C_i$ une étiquette qui déterminera la classe abstraite considérée pour
cette composante.

La propriété suivante s'assure que l'ensemble des règles est décidable :
$\{C_i : etiquette(C_i) = FES\} \triangleright \{C_i : etiquette(C_i) = GBTS\} \triangleright
\{C_i : etiquette(C_i) = FUS\}$
C'est à dire qu'aucune règle \fes ne doit dépendre d'une règle \fus ou \gbts, et
qu'aucune règle \gbts ne doit dépendre d'une règle \fus.

En effet les algorithmes de chaînage arrière par exemple réécrivent la requête jusqu'à ce
qu'elle corresponde à la base, tandis que ceux avant ajoutent des faits jusqu'à obtenir
la requête. Il est donc évident que si une composante n'accepte que le chaînage arrière,
il ne doit exister aucune règle de celle-ci de laquelle dépende une règle de la
composante acceptant uniquement le chaînage avant (si tel était le cas, cette règle ne
serait jamais déclenchée).


