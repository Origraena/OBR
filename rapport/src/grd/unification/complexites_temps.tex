Ainsi, les complexités en temps d'exécution des algorithmes nécessaires à l'unification 
sont les suivantes :
\begin{itemize}
	\item UnificationLocale (pire des cas) $= \bigcirc(k \times p)$
	\item Extension (pire des cas) $= \bigcirc(k \times p)$ (\cite{algorithmique})
	\item Unification (pire des cas) $= \bigcirc(k^{2} \times p)$
\end{itemize}

%%%%	Extension\\
%%%%	On effectue simplement un parcours en largeur \`a partir d'un sommet donn\'e qui peut \'eventuellement s'arr\^eter plus
%%%%	t\^ot qu'un parcours classique. La complexit\'e en temps dans le pire des cas est donc au plus la m\^eme, c'est \`a dire lin\'eaire
%%%%	au nombre de sommets plus le nombre d'arcs (voir \cite{algorithmique} page 517-525).
%%%%	Le graphe repr\'esentant la conjonction d'atomes $H_{1}$ poss\`edent $n \times ariteMax(H_{1})$ ar\^etes
%%%%	et $n \times (ariteMax(H_{1}) + 1)$ sommets.
%%%%	On sait donc que $C_{Extension}^{temps} = \bigcirc(n \times ariteMax(H_1))$.


%%%%	Deux cas :\\
%%%%		i) D\'ecouverte d'un sommet atome : \\
%%%%			parcours classique => on r\'ecup\`ere tous les voisins blancs \\
%%%%			parcours extension => on r\'ecup\'ere tous les voisins blancs de couleur logique noire (en effet arr\^et imm\'ediat si un voisin de couleur logique noire est d\'ecouvert.\\
%%%%		ii) D\'ecouverte d'un sommet variable :\\
%%%%			pas de diff\'erence\\
%%%%	On a donc que l'algo ne suit que les sommets atomes pouvant \^etre unifies localement. C'est \`a dire (entre autres) que leur pr\'edicat est \'egal \`a pr\'edicat($C_{2}$).



%Le parcours suit au pire $2(n \times p)+1$ ar\^etes.
%Supposons que l'algorithme ait d\'ej\`a parcouru $2(n \times p)$ ar\^etes %et qu'il d\'ecouvre un sommet toujours blanc.
%...


%On a donc $C_{Extension}^{temps} = \bigcirc$


