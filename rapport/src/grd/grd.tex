% grd.tex

Le graphe de dépendances de règles est une représentation d'une base de règles très
intéressante. En effet il permet de regarder rapidement quelles règles pourront
éventuellement être déclenchées après l'application d'une règle donnée.

De plus il permet de déterminer l'appartenance à certaines classes de règles, et le
calcul de ses composantes permet de "découper" la base de manière à effectuer
les requêtes de manières différentes selon celles-ci. 


\section{Définition}\label{grd_def}
Le graphe de dépendances des règles associé à une base de règles $B_R$ est défini comme le 
graphe orienté $GRD = (V_{GRD},E_{GRD})$ avec :
\begin{itemize}
	\item $V_{GRD} = \{R_i \in B_R\}$,
	\item $E_{GRD} = \{(R_i,R_j) : \exists$ un bon unificateur $\mu : \mu(C_i) =
	\mu(H_j)\}$. 
\end{itemize}
Intuitivement, on crée un sommet par règle et on relie $R_i$ à $R_j$ si $R_i$ "peut
amener à déclencher" $R_j$ ($R_j$ dépend de $R_i$).
La notion d'unificateur est abordée dans la section suivante.



\section{Unification de règles}\label{grd_unif}

Afin de pouvoir construire ce graphe, il faut donc pouvoir déterminer si une règle peut
en déclencher une autre, c'est à dire s'il existe un unificateur entre la conclusion de
la première et l'hypothèse de la seconde.
Tout d'abord, l'unification est définie, s'en suit un algorithme permettant de vérifier
si un tel unificateur existe, puis la correction de celui-ci ainsi que ses complexités. 

\subsection{Définitions}\label{grd_unif_def}
% unification.tex

\subsubsection{Substitution}\label{def_unification}
Une substitution de taille $n$ d'un ensemble de symboles $X$ dans un ensemble de symboles $Y$ est une fonction de $X$ 
vers $Y$ repr\'esent\'ee par l'ensemble de couples suivants (avec $n \leq |X|$) :
\begin{itemize}
	\item $s = \{(x_{i},y_{i})\ :\ \forall i \in [1,n]\ x_{i} \in X,\ y_{i} \in Y, \forall j \neq i\ x_{i} \neq x_{j}\}$
	\item $s(x_{i}) = y_{i}\ \forall i \in [1,n]$
	\item $s(x_{i}) = x_{i}\ \forall i \in [n+1,|X|]\ x_{i} \in X$
\end{itemize}

\subsubsection{Unificateur logique}\label{def_unificateur}
Un unificateur logique entre deux atomes $a_{1}$ et $a_{2}$ est une substitution $\mu$ telle que :
\begin{itemize}
	\item $\mu : var(a_{1}) \cup var(a_{2}) \rightarrow dom(a_{1}) \cup dom(a_{2})$
	\item $\mu(a_{1}) = \mu(a_{2})$
\end{itemize}
Cette d\'efinition s'\'etend aux conjonctions d'atomes.

\subsubsection{Unificateur de conclusion atomique}\label{def_unificateuratomique}
Un unificateur de conclusion atomique est un unificateur logique $\mu = \{(x_{i},t_{i}) : \forall i \in [1,n]\}$
entre l'hypoth\`ese d'une r\`egle $R_{1} = (H_{1},C_{1})$ et la conclusion atomique d'une r\`egle $R_{2} = (H_{2},C_{2})$ et est d\'efini de la mani\`ere suivante :
\begin{itemize}
	\item $\mu : fr(R_{2}) \cup var(H_{1}) \rightarrow dom(C_{2}) \cup cst(H_{1})$
	\item $\forall (x_{i},t_{i}) \in \mu\ si\ x_{i} \in fr(R_{2})\ alors\ t_{i} \in cutp(R_{2}) \cup cst(H_{1})$
\end{itemize}
    
\subsubsection{Bonne unification atomique}\label{def_bonneunification}
Un bon unificateur de conclusion atomique est un unificateur de conclusion atomique $\mu = \{x_{i},t_{i}) : \forall i \in [1,n]\}$
entre un sous ensemble $Q$ de l'hypoth\`ese d'une r\`egle $R_{1} = (H_{1},C_{1})$ et la conclusion d'une r\`egle atomique
$R_{2} = (H_{2},C_{2})$ tel que :\\
$\forall (x_{i},t_{i}) \in \mu : si\ x_{i} \in H_{1} \setminus Q\ alors\ t_{i}\ n'est\ pas\ une\ \exists-var$

Un tel ensemble $Q$ est appel\'e un {\em bon ensemble d'unification atomique} de l'hypoth\`ese de $R_{1}$ par la conclusion de $R_{2}$.
On note que $Q$ est donc d\'efini comme suit :
\begin{itemize}
	\item $Q \subseteq H_{1}$
	\item $\forall\ position\ i\ de\ \exists-var\ dans\ C_{2}, \forall\ atome\ a\ \in Q, si\ a_i \in var(H_{1})\ alors\ \forall\ atome\ b \in H_{1} : si\ \exists\ b_j \in b : a_i = b_j, alors\ b \in Q$
\end{itemize}

\subsubsection{Bon ensemble d'unification atomique minimal}\label{def_bonensemble}
Un {\em bon ensemble d'unification atomique minimal} $Q$ de $H_{1}$ par $C_{2}$ enracin\'e en $a$ est d\'efini tel que :
\begin{itemize}
	\item $Q$ est un bon ensemble d'unification atomique de $H_{1}$ par $C_{2}$
	\item $a \in Q$
	\item $|Q| = min(|Q_i| : Q_i\ est\ un\ bon\ ensemble\ d'unification\ atomique\ de\ H_1\ par\ C_2\ et\ a \in Q_i)$
\end{itemize}




\subsection{Algorithmes}\label{grd_algo}
	Vérifier qu'une règle atomique $R_i$ peut déclencher $R_j$ consiste donc à trouver un bon
	unificateur atomique entre $R_i$ et $R_j$. Dans cette section, un algorithme
	permettant de répondre à ce problème est détaillé.

	Dans la suite, les règles sont supposées représentées par des graphes (tels que
	définis en \ref{def_representation}) et à conclusion atomique.

	% TODO changer les couleurLogique en estLocalementUnifiable, et les couleur en
	% estTraite
	Le premier algorithme fait appel aux deux suivants de manière à déterminer si il
	existe au moins un unificateur entre les deux conjonctions d'atomes.
	En première phase, il vérifie l'exitence d'unificateurs avec chaque atome de manière
	indépendante. S'ensuit une extension à partir des atomes préselectionnés, et dès
	qu'un bon ensemble d'unification est entièrement unifié, l'algorithme s'arrête en
	répondant avec succès.
	\begin{center}
\begin{algorithm}[H]
\caption{Unification}\label{algo_unification}
\algsetup{indent=2em,linenodelimiter= }
\begin{algorithmic}[1]
\REQUIRE $H_{1}$ : conjonction d'atomes, $R = (H_{2},C_{2})$ : r\`egle \`a conclusion atomique 
\ENSURE succ\`es si $C_{2}$ peut s'unifier avec $H_{1}$, i.e. si $\exists H \subseteq H_{1}, \mu\ une\ substitution : \mu(H_{1}) = \mu(C_{2})$, \'echec sinon 
\STATE $\triangleright$ Pr\'ecoloration
\FORALL {$sommet\ atome\ a \in H_{1}$}
	\IF {$UnificationLocale(a,R) \neq$ \'echec}
		\STATE $localementUnifiable[a] \leftarrow vrai$
	\ELSE
		\STATE $localementUnifiable[a] \leftarrow faux$
	\ENDIF
\ENDFOR 

\STATE $\triangleright$ Initialisation du tableau contenant les positions des variables existentielles de $C_{2}$
\STATE $E \leftarrow \{i : c_{i}\ est\ une\ \exists-var\ de\ C_{2}\}$

\STATE $\triangleright$ Extension des ensembles
\FORALL {$sommet\ atome\ a \in H_{1} : localementUnifiable[a] = noir$}
	\IF {$Q \leftarrow Extension(H_{1},a,localementUnifiable,E) \neq $ \'echec} 
		\IF {$UnificationLocale(Q,R) \neq$ \'echec}
			\RETURN succ\`es
		\ENDIF
	\ENDIF
	\STATE $localementUnifiable[a] \leftarrow faux$
\ENDFOR
\RETURN \'echec
\end{algorithmic}
\end{algorithm}
\end{center}



	Le deuxième algorithme est utilisé pour le calcul des bons ensembles d'unification à
	partir d'un atome racine. Tant qu'aucune erreur n'est détectée il "avale" les atomes
	voisins aux termes en position "existentielles".
	\begin{center}
\begin{algorithm}[H]
\caption{Extension}\label{algo_extension}
\algsetup{indent=2em,linenodelimiter= }
\begin{algorithmic}[1]
\REQUIRE	$H_{1}$ : conjonction d'atomes, 
			$a \in H_{1}$ : sommet atome racine, 
			$localementUnifiable$ : tableau de taille \'egal au nombre d'atomes dans
			$H_{1}$ tel que $localementUnifiable[a] = vrai$ ssi $UnificationLocale(a,R) =$ succ\`es, 
			$E$ : ensemble des positions des variables existentielles
\ENSURE $Q$ : bon ensemble d'unification minimal des atomes de ${H_1}$ construit \`a partir de $a$ s'il existe, \'echec sinon.
\STATE $\triangleright$ Initialisation du parcours
%\FORALL {$sommet\ v \in H_{1}$}
%	\STATE $couleur[v] \leftarrow blanc$   $\triangleright$ coloration pour le parcours
%\ENDFOR
\FORALL {$sommet\ atome\ a \in H_{1}$}
	\IF {$localementUnifiable[a] = vrai$}
		\FORALL {$sommet\ terme\ t\ \in voisins(a)$}
			\STATE $couleur[t] = blanc$
		\ENDFOR
		\STATE $couleur[a] = blanc$
	\ENDIF
\ENDFOR
\STATE $couleur[a] \leftarrow noir$
\STATE $Q \leftarrow \{a\}$   $\triangleright$ conjonction d'atomes \`a traiter
\STATE $attente \leftarrow \{a\}$   $\triangleright$ file d'attente du parcours
\WHILE {$attente \neq \emptyset$}
	\STATE $u \leftarrow haut(attente)$
	\IF {$u\ est\ un\ atome$}
		\FORALL {$i \in E$}
			\STATE $v \leftarrow voisin(u,i)$
			\IF {$v\ est\ une\ constante$}
				\RETURN \'echec	
			\ELSIF {$couleur[v] = blanc$}
				\STATE $\triangleright$ v est une $\forall-var$ non marqu\'ee par le parcours
				\STATE $couleur[v] \leftarrow noir$
				\STATE $attente \leftarrow attente \cup \{v\}$
			\ENDIF
		\ENDFOR
	\ELSE
		\STATE $\triangleright$ u est un terme
		\FORALL {$v \in voisins(u)$}
			\IF {$localementUnifiable[v] = faux$}
				\RETURN \'echec
			\ELSE
				\IF {$couleur[v] = blanc$}
					\STATE $couleur[v] \leftarrow noir$
					\STATE $attente \leftarrow attente \cup \{v\}$
					\STATE $Q \leftarrow Q \cup \{v\}$
				\ENDIF
			\ENDIF
		\ENDFOR
	\ENDIF
\ENDWHILE $\ \ \ \triangleright$ Fin du parcours
\RETURN $Q$
\end{algorithmic}
\end{algorithm}
\end{center}



	Remarque : \\
	La phase d'initialisation du parcours pourrait simplement parcourir tous les sommets
	de $H_{1}$ et mettre leur couleur \`a blanc.
	En pratique, cette solution serait sans doute plus efficace, mais d\'ependrait donc
	du nombre de sommets total dans $H_{1}$.
	Ce qui en th\'eorie am\`enerait la complexit\'e de cette boucle en 
	$\bigcirc(nombre\ d'atomes\ \times arite\ max\ de\ H_{1})$.
	Or ici, la complexit\'e ne d\'epend pas de cette arit\'e max, mais uniquement de
	l'arit\'e du pr\'edicat de la conclusion $C_{2}$.

	Le dernier algorithme est celui qui teste réellement si il existe un unificateur
	entre une conclusion atomique, et un bon ensemble d'unification atomique minimal.
	Il est appelé une première fois pour tester les atomes de la conjonction séparement,
	et permettre une préselection des atomes (qui vont servir de racine).
	Durant la dernière phase (lorsqu'il est appelé sur les ensembles étendus), s'il trouve
	un unificateur, celui-ci assure que la règle peut amener à déclencher la conjonction.
	\begin{center}
\begin{algorithm}[H]
\caption{UnificationLocale}\label{unification_locale}
\algsetup{indent=2em,linenodelimiter= }
\begin{algorithmic}[1]
\REQUIRE $H_{1}$ : conjonction d'atomes, $R = (H_{2},C_{2})$ : r\`egle \`a conclusion atomique 
\ENSURE succ\`es si $C_{2}$ peut s'unifier avec $H_{1}$ 
\STATE $\triangleright$ V\'erification des pr\'edicats
\FORALL {$atome\ a \in H_{1}$}
	\IF {pr\'edicat($a$) $\neq$ pr\'edicat($C_{2}$)}
		\RETURN \'echec
	\ENDIF
\ENDFOR
\STATE $u \leftarrow \emptyset\ \ \ \triangleright$ substitution 
\FORALL {$terme\ t_{i} \in C_{2}$}
	\STATE $\triangleright$ def : $a_{i}$ = terme de a en position i
	\STATE $E \leftarrow \{a_{i} : \forall\ atome\ a \in H_{1}\}$
	\IF {$t_{i}\ est\ une\ constante$}
		\IF {$\exists\ v \in E : v\ est\ une\ constante\ et\ v \neq t_{i},\ ou\ v\ est\ une\ \exists-variable$}
			\RETURN \'echec
		\ELSE
			\STATE $u \leftarrow \{(v,t_{i}) : v \in E\ et\ v \neq t_{i}\}$
			%\STATE $u \leftarrow u \circ \{(v,t_{i}) : v \in E\ et\ v \neq t_{i}\}$
		\ENDIF
	\ELSIF {$t_{i}\ est\ une\ \exists-variable$}
		\IF {$\exists\ v \in E : v\ est\ une\ \exists-variable\ et\ v \neq t_{i},\ ou\ v\ est\ une\ constante$}
			\RETURN \'echec
		\ELSE
			\STATE $u \leftarrow \{(v,t_{i}) : v \in E\ et\ v \neq t_{i}\}$
			%\STATE $u \leftarrow u \circ \{(v,t_{i}) : v \in E\ et\ v \neq t_{i}\}$
		\ENDIF
	\ELSE
		\IF {$\exists\ v_{1},v_{2} \in E : v_{1} \neq v_{2}\ et\ v_{1},v_{2}\ ne\ sont\ pas\ des\ \forall-variables$}
			\RETURN \'echec
		\ELSIF {$\exists\ c \in E : c\ est\ une\ constante$}
			\STATE $u \leftarrow \{(v,c) : v \in E \cup \{t_{i}\}\ et\ v \neq c\}$
			%\STATE $u \leftarrow u \circ \{(v,c) : v \in E\ et\ v \neq c\} \cup \{(t_{i},c)\}$
		\ELSE
			\STATE $u \leftarrow \{(v,t_{i}) : v \in E\ et\ v \neq t_{i}\}$
			%\STATE $u \leftarrow u \circ \{(v,t_{i}) : v \in E\ et\ v \neq t_{i}\}$
		\ENDIF
	\ENDIF	
	\STATE $H_{1} \leftarrow u(H_{1})$
	\STATE $C_{2} \leftarrow u(C_{2})$
\ENDFOR
\RETURN succ\`es 

\end{algorithmic}
\end{algorithm}
\end{center}





\subsection{Correction}\label{grd_correction}
%\subsubsection{Propri\'et\'es}\label{unification_correction_proprietes}
%	{\em Propri\'et\'e 1} \\
%	Soit $R = (H,C)$ une r\`egle \`a conclusion atomique, \\
%	Soit $K = (k_1, k_2, ..., k_n)$ une conjonction d'atomes,\\
%	si $\exists a \in K$ : a ne peut pas s'unifier avec C alors $K$ ne peut pas s'unifier avec $C$

\subsection{Complexites}\label{grd_complexites}
	
Soit $R$ une règle, et $G_R$ son graphe associé.
On note :
\begin{itemize}
	\item $k$ : nombre d'atomes dans $R$
	\item $p$ : arit\'e maximum des prédicats de $R$
	\item $t \leq n \times p$ : nombre de termes de $R$
\end{itemize}
et :
\begin{itemize}
	\item $n = k + t$ : nombre de sommets dans $G_R$
	\item $m \leq n \times p$ : nombre d'arêtes dans $G_R$
\end{itemize}

Avec notre représentation nous avons donc les complexités suivantes :
\begin{itemize}
	\item Parcourir les sommets prédicats : $\bigcirc(k)$.
	\item Parcourir tous les sommets : $\bigcirc(n)$.
	\item Acc\'eder au $i^{eme}$ voisin d'un sommet : $\bigcirc(1)$.
\end{itemize}



\begin{itemize}
	\item $n$ = nombre d'atomes dans $H_{1}$
	\item $p$ = arit\'e de $C_{2}$
	\item $t$ = nombre de termes "colorables" dans $H_{1}$
	\item $m$ = nombre d'ar\^etes "suivables" dans $H_{1}$
\end{itemize}

On remarque que dans le pire des cas on a :
\begin{itemize}
	\item $t = n \times p$
	\item $m = n \times p$
\end{itemize}

\subsubsection{Temps}
	Ainsi, les complexités en temps d'exécution des algorithmes nécessaires à l'unification 
sont les suivantes :
\begin{itemize}
	\item UnificationLocale (pire des cas) $= \bigcirc(k \times p)$
	\item Extension (pire des cas) $= \bigcirc(k \times p)$
	\item Unification (pire des cas) $= \bigcirc(k^{2} \times p)$
\end{itemize}

%%%%	Extension\\
%%%%	On effectue simplement un parcours en largeur \`a partir d'un sommet donn\'e qui peut \'eventuellement s'arr\^eter plus
%%%%	t\^ot qu'un parcours classique. La complexit\'e en temps dans le pire des cas est donc au plus la m\^eme, c'est \`a dire lin\'eaire
%%%%	au nombre de sommets plus le nombre d'arcs (voir \cite{algorithmique} page 517-525).
%%%%	Le graphe repr\'esentant la conjonction d'atomes $H_{1}$ poss\`edent $n \times ariteMax(H_{1})$ ar\^etes
%%%%	et $n \times (ariteMax(H_{1}) + 1)$ sommets.
%%%%	On sait donc que $C_{Extension}^{temps} = \bigcirc(n \times ariteMax(H_1))$.


%%%%	Deux cas :\\
%%%%		i) D\'ecouverte d'un sommet atome : \\
%%%%			parcours classique => on r\'ecup\`ere tous les voisins blancs \\
%%%%			parcours extension => on r\'ecup\'ere tous les voisins blancs de couleur logique noire (en effet arr\^et imm\'ediat si un voisin de couleur logique noire est d\'ecouvert.\\
%%%%		ii) D\'ecouverte d'un sommet variable :\\
%%%%			pas de diff\'erence\\
%%%%	On a donc que l'algo ne suit que les sommets atomes pouvant \^etre unifies localement. C'est \`a dire (entre autres) que leur pr\'edicat est \'egal \`a pr\'edicat($C_{2}$).



%Le parcours suit au pire $2(n \times p)+1$ ar\^etes.
%Supposons que l'algorithme ait d\'ej\`a parcouru $2(n \times p)$ ar\^etes %et qu'il d\'ecouvre un sommet toujours blanc.
%...


%On a donc $C_{Extension}^{temps} = \bigcirc$




\subsubsection{Espace}
	% complexites_espace.tex

\begin{itemize}
	\item $|localementUnifiable|$ : $k$
	\item $|couleur|$ : $n$
	\item 
\end{itemize}





\section{Composantes fortement connexes}\label{grd_scc}


