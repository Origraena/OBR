% base.tex

\section{Base de connaissance}\label{def_base}
Une base de connaissance $K = (F,R)$ 
est constituée d'un ensemble de {\em faits} représenté
par une conjonction d'atomes $F$, ainsi que d'une {\em ontologie} représentée 
par un ensemble de règles $R$.

Les faits sont souvent considérés comme complètement instanciés, c'est à dire ne
contenant que des constantes, mais ici, les règles contenant des variables existentielles
peuvent générer de nouveaux individus. Donc nous définissons $F$ comme une conjonction
d'atomes existentiellement fermée.

%\section{Requête conjonctive bouléenne}\label{def_requete}
%Une requête conjonctive bouléenne est représentée par une conjonction d'atomes
%existentiellement fermée.
%Il existe plusieurs types d'algorithmes permettant de répondre à ce type de requête, 

%Prenons l'exemple suivant :\\
%$R = \{\forall x \exists y (homme(x) \rightarrow homme(y) \wedge parent(y,x))\}$\\
%$F = \{homme('A')\}$\\


