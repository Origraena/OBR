% conjonctions.tex

\section{R\`egle}\label{def_regle}
Une r\`egle $R = (H,C)$ est d\'efinie par :
\begin{itemize}
	\item $H$ : conjonction d'atomes repr\'esentant l'hypoth\`ese de $R$,
	\item $C$ : conjonction d'atomes repr\'esentant la conclusion de $R$.
\end{itemize}

On note :
\begin{itemize}
	\item $dom(R) = dom(H) \cup dom(C)$
	\item $var(R) = var(H) \cup var(C)$
	\item $cst(R) = cst(H) \cup cst(C)$
	\item $fr(R)$ l'ensemble des variables fronti\`eres de $R$, avec $a_{i}$ est une variable fronti\`ere ssi $a_{i}$ est une variable telle que $a_{i} \in var(H) \cap var(C)$.
	\item $cutp(R) = fr(R) \cup cst(R)$ l'ensemble des points de coupure de $R$
\end{itemize}

\section{Repr\'esentation des Conjonctions d'Atomes}\label{def_representation}
Une conjonction d'atomes 
$K = k_{1} \wedge k_{2} \wedge ... \wedge k_{n}$ avec $k_{i} = q(t_{1}, t_{2}, ..., t_{p})$
peut \^etre repr\'esent\'ee par le graphe non orient\'e $G_{K} = (V_{K},E_{K},\omega)$
avec $V_{K}$ son ensemble de sommets, $E_{K}$,
son ensemble d'ar\^etes et $\omega$ une fonction de poids sur les ar\^etes construits de la mani\`ere suivante : \\
$V_{K} = A_{K} \cup T_{K}$ avec $A_{K} = \{i : k_{i} \in K\}$ et $T_{K} = \{t_{j} \in dom(K)\}$ \\
$E_{K} = \{(i,t_{j}) : \forall k_{i} \in K, \forall t_{j} \in dom(k_{i})\}$ \\
$\omega(i,t_{j}) = j : \forall (i,t_{j}) \in E_{K}$ \\

On remarque que quelques propri\'et\'es sont directement induites par cette repr\'esentation :
\begin{itemize}
	\item $G_{K}$ admet une bipartition de ses sommets, en effet toutes les ar\^etes ont une extr\'emit\'e dans $A_{K}$ etre
	l'autre dans $T_{K}$, or par construction $A_{K} \cap T_{K} = \emptyset$.
\end{itemize}

\section{Impl\'ementation du Graphe}\label{def_graphe}
Par la suite on supposera que le graphe $G_{K}$ est impl\'ement\'e par deux ensembles de sommets (les atomes et les termes), ainsi que par une liste de voisinage pour chaque sommet.
De plus, chaque sommet peut avoir connaissance de son type de contenu parmis : $atome$, $terme$, $constante$, $\forall-var$, $\exists-var$
en temps constant.

On note $n$ (resp. $t$) le nombre d'atomes (resp. de termes) dans $K$. 

Op\'erations (et temps d'ex\'ecution associ\'e) :
\begin{itemize}
	\item Parcourir les sommets $atomes$ : $n$.
	\item Parcourir tous les sommets : $n + t$.
	\item Acc\'eder au $i^{eme}$ voisin d'un sommet : $constant$.
\end{itemize}


