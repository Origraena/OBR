% regles.tex

\section{R\`egle}\label{def_regle}
Une règle $R = (H,C)$ est constituée de deux conjonctions d'atomes $H$ et $C$ représentant
respectivement l'hypothèse (le corps) et la conclusion (la tête) de $R$.
Toutes les variables apparaissant dans $H$ sont quantifiées universellement tandis que
celles apparaissant uniquement dans $C$ le sont existentiellement.
%Ainsi ne pouvant plus y avoir ambiguité, par la suite les quantificateurs seront omis
Ainsi une règle est toujours sous la forme 
$R : \forall x_i (H \rightarrow \exists z_j (C))$.

On note :
\begin{itemize}
	\item $dom(R) = dom(H) \cup dom(C)$, le domaine de $R$
	\item $var(R) = var(H) \cup var(C)$, les variables de $R$
	\item $cst(R) = cst(H) \cup cst(C)$, les constantes de $R$
	\item $fr(R) = var(H) \cap var(C)$, l'ensemble des variables fronti\`eres de $R$
	\item $cutp(R) = fr(R) \cup cst(R)$, l'ensemble des points de coupure de $R$\\
\end{itemize}

\section{Représentation graphique d'une règle}
\label{def_representation_regle}
Tout comme une simple conjonction d'atomes, une règle $R = (H,C)$ peut être représentée
par un graphe similaire, en ajoutant une coloration à deux couleurs : une pour les atomes de
l'hypothèse, l'autre pour ceux de la conclusion.

Ainsi le graphe associé $G_R = (V_R,E_R,\omega,\chi)$ est défini comme :
\begin{itemize}
	\item 
		$V_{R} = P_R \cup T_{R}$ avec $P_R = \{i : r_{i} \in R\}$ et $T_{R} = 
		\{t_{j} \in dom(R)\}$
	\item 
		$E_{R} = \{(i,t_{j}) : \forall r_{i} \in R, \forall t_{j} \in dom(R_{i})\}$
	\item 
		$\omega : E_R \rightarrow [1,p]$ telle que
		$\omega(i,t_{j}) = j : \forall (i,t_{j}) \in E_{R}$
	\item 
		$\chi : P_R \rightarrow \{1,2\}$ telle que 
		$\chi(r_i) = 1$ si $r_i \in H$ et $\chi(r_i) = 2$ si $r_i \in C$.\\
\end{itemize}
\par{}

Par exemple la règle 
$R : \forall x \forall y\ (salle(x) \wedge date(y) \wedge reservee(x,y) 
\rightarrow \exists z\ (cours(z) \wedge aLieu(z,x,y)))$ 
peut être visualisée de la façon suivante :
% fig_representation_regle.tex

\begin{figure}[H]
\begin{center}
\begin{tikzpicture}[
	auto,
	node distance=1.5cm,
	inner sep=1mm
]
	% terms
	\node [term]		(x)			[]						{};
	\node [term]		(y)			[below of=x]			{};
	\node [term]		(z)			[right of=x] 			{};

	% atoms
	\node[atombody] 	(salle)		[above left of=x] 		{salle$\backslash$1}
		edge[atomedge]	node {1} 	(x)
	;
	\node[atombody] 	(date)		[below of=y] 			{date$\backslash$1}
		edge[atomedge]	node {1}	(y)
	;
	\node[atombody]		(reservee)	[below left of=salle]	{reservée$\backslash$2}
		edge[atomedge]	node {1}	(x)
		edge[atomedge]	node {2}	(y)
	;
	\node[atomhead] 	(cours) 	[above of=z] 			{cours$\backslash$1}
		edge[atomedge]	node {1}	(z)
	;
	\node[atomhead] 	(alieu) 	[below of=z] 			{aLieu$\backslash$3}
		edge[atomedge]	node {1}	(z)
		edge[atomedge]	node {2}	(x)
		edge[atomedge]	node {3}	(y)
	;
\end{tikzpicture}
\end{center}
\caption{Exemple de représentation d'une règle}
\label{fig_representation_regle}
\end{figure}



Sur la figure suivante (\ref{fig_representation_regle_labeled})
représentant la même règle, on peut voir les différentes parties
de celle-ci, de plus les variables ont été étiquetées pour mieux visualiser les
différents termes.


% fig_representation_regle_labeled.tex

\begin{figure}[h]
\label{fig_representation_regle_labeled}
\begin{center}
\begin{tikzpicture}[
	auto,
	node distance=1.5cm,
	inner sep=1mm
]
	% legend
	\atomlegend{4}{0.5}


	% terms
	\node [termfr]		(x)			[]						{$\forall$};
	\node [termfr]		(y)			[below of=x]			{$\forall$};
	\node [termhead]	(z)			[right of=x] 			{$\exists$};

	% atoms
	\node[atombody] 	(salle)		[above left of=x] 		{salle$\backslash$1}
		edge[atomedge]	node {1} 	(x)
	;
	\node[atombody] 	(date)		[below of=y] 			{date$\backslash$1}
		edge[atomedge]	node {1}	(y)
	;
	\node[atombody]		(reservee)	[below left of=salle]	{reservée$\backslash$2}
		edge[atomedge]	node {1}	(x)
		edge[atomedge]	node {2}	(y)
	;
	\node[atomhead] 	(cours) 	[above of=z] 			{cours$\backslash$1}
		edge[atomedge]	node {1}	(z)
	;
	\node[atomhead] 	(alieu) 	[below of=z] 			{aLieu$\backslash$3}
		edge[atomedge]	node {1}	(z)
		edge[atomedge]	node {2}	(x)
		edge[atomedge]	node {3}	(y)
	;
\end{tikzpicture}
\end{center}
\caption{Les différents éléments d'une règle}
\end{figure}

% EDGE :
%\draw[-] (z) -- (b);
%\draw[->] (z) -- (b);
%\draw[<->] (z) -- (b);


\section{Règle à conclusion atomique}\label{def_regle_atomique}
Une règle à conclusion atomique ajoute une contrainte sur la forme de sa conclusion qui
ne doit contenir qu'un seul atome.
Ces règles ont l'avantage d'être plus simples à unifier (voir section \ref{grd_unif}), et
la plupart des algorithme présentés dans ce rapport sont plus efficaces sur ce type de
règle, tandis que d'autres ne fonctionnent uniquement sur celles-ci.

Mais ceci n'est pas un problème puisqu'il est possible de réécrire une règle à
conclusion non atomique en un ensemble de règles à conclusions atomiques équivalent.

En effet, quelle que soit une règle $R = (H,C)$, nous pouvons définir un nouveau prédicat
$p_R$ d'arité $|var(R)|$ ainsi qu'un nouvel ensemble de règles à conclusions atomiques 
$R^{A}$ dont le premier
élément aura la même hypothèse que $R$ et une conclusion de prédicat $p_R$ contenant 
toutes ses variables, et dont les suivants auront pour hypothèse cette nouvelle
conclusion, et comme conclusion atomique les atomes de $C$.

Cet ensemble est donc défini de la manière suivante :\\
$R^{A} =\{R^{A}_i = (H^{A}_i,C^{A}_i) : \forall\ i \in [0,|C|]\}$, tels que :\\
$$
	R^{A}_0 =
	\begin{cases}
		H^{A}_0 = H,\\
		C^{A}_0 = p_R(\{x_j \in var(R)\})
	\end{cases}
	\forall\ i \in [1,|C|]\ R^{A}_i =
	\begin{cases}
		H^{A}_i = C^{A}_0,\\
		C^{A}_i = c_i \in C
	\end{cases}
$$

Ainsi nous pouvons utiliser l'algorithme \ref{algo_conversion} afin d'effectuer cette
conversion.
% algo_conversion.tex

\begin{center}
\begin{algorithm}[h]
\caption{Conversion d'une règle à conclusion non atomique}\label{algo_conversion}
\algsetup{indent=2em,linenodelimiter= }
\begin{algorithmic}[1]
\REQUIRE	$R = (H,C)$ : une règle quelconque 
\ENSURE		$R^A$ : un ensemble de règles à conlusions atomiques équivalent à $R$
	\STATE $H^A_0 \leftarrow H$
	\STATE $C^A_0 \leftarrow p_R(\{x_i \in var(R)\})$
	\STATE $R^A \leftarrow \{(H^A_0,C^A_0)\}$
	\FORALL{atome $c_i \in C$}
		\STATE $R^A_i \leftarrow (C^A_0,c_i)$
		\STATE $R^A \leftarrow R^A \cup \{R^A_i\}$
	\ENDFOR
	\RETURN $R^A$
\end{algorithmic}
\end{algorithm}
\end{center}

Toute règle pouvant donc se réécrire de manière 
%TODO strictement 
équivalente en un ensemble de
règles à conclusion atomique, dans la suite nous ne considèrerons que des règles sous cette 
forme.




