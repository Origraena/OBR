% requetes.tex

\section{Chaînage avant}\label{def_forward}
% TODO
%%%%	\subsection{Dérivation d'une règle}\label{def_derivation}
%%%%	Une règle $R_i = (H_i,C_i)$ est dérivable sur un fait $F$ si et seulement s'il existe un
%%%%	homomorphisme $\pi$ de l'hypothèse $H_i$ de $R_i$ dans $F$.
%%%%	Le résultat d'une telle dérivation est un nouveau fait $F' = F \cup \pi^{safe}(C_i)$ 
%%%%	% avec $\pi^{safe} \in \PI$ l'ensemble des homomorphismes de $H$ dans $F$.
%%%%	où $\pi^{safe}$ est une substitution qui remplace chacune des variables $x$ de la
%%%%	frontière de $R_i$ par $\pi(x)$,
%%%%	et chacune des variables restantes (donc existentielles dans $R_i$) par une nouvelle
%%%%	variable générée.

%\subsection{Principe}\label{principe_forward}
Afin de déterminer si une requête bouléenne $Q$ peut être déduite d'une base de connaissance 
$B = (R,F)$, 
le chaînage avant {\em dérive} de manière itérative la base de faits $F$ (qui peut être 
vue comme un fait unique) par l'ensemble de  règles R de manière à générer (en cherchant
de nouveaux homomorphismes) de nouveaux faits qui devront à leur tour être étudiés.
Avant chaque dérivation, l'algorithme vérifie si $F^i$ contient $Q$ auquel cas la réponse 
est positive, et si $F^{i} = F^{i-1}$ afin de savoir si la chaîne de dérivation est finie et 
dans ce cas donner une réponse négative.

%%%%	\begin{center}
%%%%	\begin{algorithm}
%%%%	\caption{Chaînage avant}\label{algo_forward}
%%%%	\algsetup{indent=2em,linenodelimiter= }
%%%%	\begin{algorithmic}[1]
%%%%	\REQUIRE	$Q$ : conjonction d'atomes correspondant à une requête conjonctive bouléenne,
%%%%				$R$ : un ensemble de règles,
%%%%				$F$ : un fait
%%%%	\ENSURE Succès si $Q$ peut être déduit de $R$ et $F$, échec sinon
%%%%		\IF {$F 

%%%%	\end{algorithmic}
%%%%	\end{algorithm}
%%%%	\end{center}

% On part des faits et on essaie d'arriver à Q

\section{Chaînage arrière}\label{def_backward}
Le chaînage arrière quant à lui consiste à réécrire la requête $Q$ en plusieurs nouvelles
requêtes juqu'à ce que la réponse à une de celles-ci soit dans $F$.
Pour cela les conclusions des règles de la base sont {\em unifiées} avec 
des sous-ensembles de $Q$ (voir section \ref{grd_unif}
pour plus de détails sur l'unification).
% On part de Q et on réécrit jusqu'à ce que Q soit dans K

Que l'on considère le chaînage avant ou arrière, la notion de dépendance de règles est
très importante. En effet lors de la génération de nouveaux faits seules certaines règles
ont besoin d'être appliquées et il en est de même lors de la réécriture de la requête.
