% requetes.tex

\section{Chaînage avant}\label{def_forward}

\subsection{Dérivation d'une règle}\label{def_derivation}
Une règle $R = (H,C)$ est dérivable sur un fait $F$ si et seulement s'il existe un
homomorphisme $\pi$ de l'hypothèse $H$ de $R$ dans $F$.
Le résultat d'une telle dérivation est un nouveau fait $F' = F \cup \pi_i(C)$

Le chaînage avant consiste à {\em appliquer} une règle sur une base de faits (qui peut
être vue comme un fait unique) de manière à générer de nouveaux faits qui devront à leur
tour être étudiés.

% On part des faits et on essaie d'arriver à Q

\section{Chaînage arrière}\label{def_backward}
% On part de Q et on réécrit jusqu'à ce que Q soit dans K

