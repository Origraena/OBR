% atomes.tex

\section{Pr\'edicat}\label{def_predicat}
%\begin{itemize}
%	\item 
Un pr\'edicat not\'e $p \backslash n$ est un symbole relationnel d'arit\'e $n$.
%	\item 
Dans la suite, on supposera que tout nom de pr\'edicat est unique et
%	\item 
on notera $p_{i}$ la $i^{eme}$ position de $p$.
%	\item On note $P_{A}$ l'ensemble des pr\'edicats de $K$.
%\end{itemize}

\section{Atome}\label{def_atome}
Un atome $a = p(a_{1},a_{2},...,a_{n})$ associe un terme \`a chaque position d'un pr\'edicat $p \backslash n$.
On note :
\begin{itemize}
	\item $a_{i}$ le terme en position $i$ dans $a$. Un terme peut \^etre une {\em constante} 
	ou une {\em variable}.
	Une variable peut \^etre {\em libre}  ou {\em quantifiée} universellement 
	(not\'ee $\forall-var$) ou existentiellement (not\'ee $\exists-var$).
	\item $dom(a) = \{a_{i} : \forall\ i \in [1,n]\}$, l'ensemble des termes de $a$
	\item $var(a)$ l'ensemble des variables de $a$
	\item $cst(a)$ l'ensemble des constantes de $a$
\end{itemize}

\section{Conjonction d'atomes}\label{def_conjonction}
Une conjonction de $n$ atomes $A$ est définie telle que :\\
$A = \bigwedge_{i = 1}^{n}k_i$
avec $\forall\ i \in [1,n]\ a_i = p_i(a_{i1},a_{i2},...,a_{in_i})$ un atome de prédicat 
$p_i \backslash n_i$. 

\section{Représentation graphique d'une conjonction d'atomes}
\label{def_representation_conjonction}
Une conjonction d'atomes 
%$K = k_{1} \wedge k_{2} \wedge ... \wedge k_{n}$ avec $k_{i} = p_i(t_{1}, t_{2}, ..., t_{p})$
peut \^etre repr\'esent\'ee par le graphe non orient\'e $G_{A} = (V_{A},E_{A},\omega)$
avec $V_{A}$ son ensemble de sommets, $E_{A}$,
son ensemble d'ar\^etes et $\omega$ une fonction de poids sur les ar\^etes construits de la mani\`ere suivante :
\begin{itemize}
	\item $V_{A} = P_{A} \cup T_{A}$ avec $P_{A} = \{i : a_{i} \in A\}$ et 
	$T_{A} = \{t_{j} \in dom(A)\}$
	\item $E_{A} = \{(i,t_{j}) : \forall a_{i} \in A, \forall t_{j} \in dom(a_{i})\}$
	\item $\omega : E_A \rightarrow \N$ telle que
	$\omega(i,t_{j}) = j : \forall (i,t_{j}) \in E_{A}\\$
\end{itemize}

Cette représentation a de nombreux avantages, elle permet notamment de parcourir
rapidement les atomes liés à un terme (et réciproquement), ainsi que de pouvoir être
visualisée agréablement (voir figure \ref{fig_representation_conjunct}).
% fig_representation_conjunct.tex

\begin{figure}[h]
\begin{center}
\begin{tikzpicture}[
	auto,
	node distance=1.5cm,
	inner sep=1mm
]
	\node [term]		(x)			at (0,0)				{};
	\node [term]		(y)			at (3,0)				{};
	\node[atom]		 	(salle)		[left of=x] 	{salle$\backslash$1}
		edge[atomedge]	node {1} 	(x)
	;
	\node[atom]			(reservee)	[right of=x]		{reservée$\backslash$2}
		edge[atomedge]	node {1}	(x)
		edge[atomedge]	node {2}	(y)
	;
	\node[atom] 		(date)		[right of=y] 	{date$\backslash$1}
		edge[atomedge]	node {1}	(y)
	;
	\end{tikzpicture}
\end{center}
\caption{Représentation de 
$\forall x, \forall y (salle(x) \wedge date(y) \wedge reservee(x,y))$}
\label{fig_representation_conjunct}
\end{figure}



On remarque que %quelques propri\'et\'es sont directement induites par cette repr\'esentation :
%\begin{itemize}
	%\item 
	$G_{A}$ admet une bipartition de ses sommets, en effet toutes les ar\^etes ont 
	une extr\'emit\'e dans $P_{A}$ et l'autre dans $T_{A}$, or par construction $P_{A} 
	\cap T_{A} = \emptyset$.
%\end{itemize}


