% conclusion.tex

\subsection{Gestion du projet}
Comme signalé dans les outils utilisés (section \ref{outils}) git a été utilisé comme
gestionnaire de version.
La communication a quant à elle était principalement effectuée par courriels, et des
réunions avec les encadrants ont été organisées presque hebdomadairement.

\subsection{Problèmes rencontrés}
Le principal problème a été l'abandon du projet par l'un des membres du groupe. De plus,
il a été relativement difficile de prendre du recul rapidement. Ainsi certaines
implémentations auraient sans doute été faites différemment avec une meilleure
organisation du temps.

\subsection{Contributions}
Malgré tout, les contraintes du projet ont été respectées.
Tout d'abord, afin de subvenir aux besoins de l'implémentation nous avons donc programmé une
bibliothèque adaptée pour nos graphes, et nous avons conçu un algorithme permettant
de vérifier la dépendance entre deux règles afin de pouvoir construire le graphe de
dépendances associé à un ensemble.
De plus une grande partie des classes concrètes exhibées à ce
jour est reconnue par notre analyseur, et ainsi celui-ci est la plupart du temps 
capable de déterminer si
le problème d'effectuer une requête sur un ensemble de règles est décidable ou non.
Comme prévu, une entrée à partir des
fichiers respectant un format spécifique interne ou le format Datalog a été ajoutée.
Et enfin nous avons également trouvé agréable de mettre en place une sortie PostScript
minimale pour la visualisation des différents graphes, ainsi qu'une entrée 

\subsection{Perspectives}

Mais à l'heure actuelle, d'autres classes sont définies
régulièrement et il serait intéressant de les implémenter afin de fournir un outil
complet, voir d'en rechercher de nouvelles.

Quant à elle, la recherche sur la combinaison des classes abstraites en est à ses débuts. 
Ainsi il est
éventuellement possible de mettre en évidence d'autres situations où le problème reste
décidable sans pour autant satisfaire la propriété de précédance utilisée.

De plus, pour le moment notre outil ne possède pas d'interface graphique, or il pourrait
être
agréable de visualiser le graphe de dépendances des règles et ses composantes fortement 
connexes directement. En effet, on pourrait ainsi ajouter, ou retirer des règles pour
étudier les changements de décidabilité lors de ces modifications.

Enfin il existe de nombreux formats différents utilisés pour représenter des ontologies,
et il serait intéressant de pouvoir transformer un maximum de bases afin de les
analyser, et dans cette optique il faudrait donc mettre en place d'autres convertisseurs
de fichiers.

