% conclusion.tex

En conclusion, nous pouvons dire qu'une grande partie des classes concrètes exhibées à ce
jour sont reconnues par notre analyseur, et ainsi celui-ci est la plupart du temps 
capable de déterminer si
le problème d'effectuer une requête sur un ensemble de règles est décidable ou non.

Mais à l'heure actuelle, d'autres classes sont définies
régulièrement et il serait intéressant de les implémenter afin de fournir un outil
complet, voir d'en rechercher de nouvelles.

Quant à elle, la recherche sur la combinaison des classes abstraites en est à ses débuts. 
Ainsi il est
éventuellement possible de mettre en évidence d'autres situations où le problème reste
décidable sans pour autant satisfaire la propriété de précédance utilisée.

De plus, pour le moment notre outil ne possède pas d'interface graphique, or il pourrait
être
agréable de visualiser le graphe de dépendances des règles et ses composantes fortement 
connexes directement. En effet, on pourrait ainsi ajouter, ou retirer des règles pour
étudier les changements de décidabilité lors de ces modifications.

Enfin il existe de nombreux formats différents utilisés pour représenter des ontologies,
et il serait intéressant de pouvoir transformer un maximum de bases afin de les
analyser, et dans cette optique il faudrait donc mettre en place d'autres convertisseurs
de fichiers.

