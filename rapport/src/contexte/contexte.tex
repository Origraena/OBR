% contexte.tex

\section{Introduction}
A l'heure actuelle, les bases de connaissance sont de plus en plus répandues, et ce dans
un grand nombre de domaines.
En effet pouvoir représenter les informations et obtenir des réponses à des requêtes sur
celles-ci est primordial.
Si nous considérons une base de données classique, les seules réponses envisageables sont
celle contenues directement dans ces données.
Ainsi une requête consiste à vérifier la présence de certaines propriétés sur celles-ci.

Mais pouvoir déduire des informations supplémentaires à partir des initiales est très
important, et donc pour cela les bases de connaissance définissent une {\em ontologie}
autour des données.
Cette dernière met en place un ensemble de règles qui peuvent être plus ou moins
complexes.

Pendant longtemps, seules des règles {\em simples} étaient prises en compte. Mais le
développement du Web-sémantique et le besoin d'effectuer des requêtes toujours plus
complexes ont amené à ajouter aux règles la possibilité de créer de nouveaux individus.
Ainsi, bien qu'il est soit possible d'obtenir une réponse d'une base de connaissance {\em simple} (par exemple
au format Datalog) à coup sûr, si nous ajoutons ces règles particulières (par exemple au
format Datalog+/-) alors il peut
arriver qu'aucune réponse ne soit donnée en temps fini.
Nous disons donc que le problème de réponse à une requête dans une base de connaissance
est de manière général indécidable.

Prenons l'exemple d'une règle disant que "si un individu est un homme, alors il existe un
autre individu qui est un homme et le père du premier", et une donnée "Tom est un homme".
Il est évident que si nous souhaitons déduire toutes les informations possibles à partir
de cette base, il se posera vite un problème.
En effet, Tom est un homme, donc il existe un autre homme $x1$ qui est le père de Tom.
Et donc il existe encore un autre homme $x2$ qui a également un père $x3$, etc...

Il est donc nécessaire de pouvoir déterminer des classes de règles (les plus générales
possibles), permettant de s'assurer que les réponses aux requêtes soient données en temps
fini. 

% problematique.tex

\begin{frame}
	\frametitle{Problématique}
	\begin{itemize}
		\item Développement d'un outil analysant une base de règles
		\item Construction du graphe de dépendances associé
		\item Détermination des classes de règles 
		\item Décidabilité de la base
		\item Déduction des algorithmes à utiliser
		\item Lecture et écriture d'une base à partir et vers un fichier
		\item Langage Java
	\end{itemize}
\end{frame}



