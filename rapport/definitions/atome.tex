\section{D\'efinitions}

\subsection{Substitution}
	Une substitution s est une fonction de X dans Y, repr\'esent\'ee par un ensemble de couples :
	\begin{itemize}
		\item $s = \{(x_{i},y_{i})\ :\ \forall i \in [1,n]\ x_{i} \in X,\ y_{i} \in Y, \forall j \neq i\ x_{i} \neq x_{j}\}$
		\item $s(x_{i}) = y_{i}$
	\end{itemize}

\subsection{Unificateur logique}
	Un unificateur logique entre deux atomes $a_{1}$ et $a_{2}$ est une substitution $\mu$ telle que :
	\begin{itemize}
		\item $\mu : var(a_{1}) \cup var(a_{2}) \rightarrow dom(a_{1}) \cup dom(a_{2})$
		\item $\mu(a_{1}) = \mu(a_{2})$
	\end{itemize}

\subsection{Base de connaissance}
	Une base de connaissance est not\'ee $K = F_{K} \cup R_{K}$ avec :
	\begin{itemize}
		\item $F_{K}$ : l'ensemble des {\em faits} de $K$
		\item $R_{K}$ : l'ensemble des {\em r\`egles} de $K$
	\end{itemize}

\subsection{Pr\'edicat}
	\begin{itemize}
		\item Un pr\'edicat not\'e $p \backslash n$ est un symbole relationnel d'arit\'e $n$.
		\item On suppose que tout nom de pr\'edicat est unique.
		\item On note $p_{i}$ la $i^{eme}$ position de $p$.
		\item On note $P_{K}$ l'ensemble des pr\'edicats de $K$.
	\end{itemize}

\subsection{Atome}
	Un atome $a = p(a_{1},a_{2},...,a_{n})$ associe un terme \`a chaque position d'un pr\'edicat $p \backslash n$.
	On note :
	\begin{itemize}
		\item $a_{i}$ le terme en position $i$ dans $a$. Un terme peut \^etre une {\em constante} ou une {\em variable}.
		Une variable peut \^etre universelle (not\'ee $\forall-var$) ou existentielle (not\'ee $\exists-var$).
		\item $dom(a) = \{a_{i} : \forall i \in [1,n]\}$, l'ensemble des termes de $a$
		\item $var(a)$ l'ensemble des variables de $a$
		\item $cst(a)$ l'ensemble des constantes de $a$
	\end{itemize}
	Ces notations s'\'etendent aux {\em conjonctions d'atomes}.

\subsection{R\`egle}
	Une r\`egle $R = (H,C)$ est d\'efinie par :
	\begin{itemize}
		\item $H$ : conjonction d'atomes repr\'esentant l'hypoth\`ese de $R$,
		\item $C$ : conjonction d'atomes repr\'esentant la conclusion de $R$.
	\end{itemize}
	On note :
	\begin{itemize}
		\item $dom(R) = dom(H) \cup dom(C)$
		\item $var(R) = var(H) \cup var(C)$
		\item $cst(R) = cst(H) \cup cst(C)$
		\item $fr(R)$ l'ensemble des variables fronti\`eres de $R$, avec $a_{i}$ est une variable fronti\`ere ssi $a_{i}$ est une variable telle que $a_{i} \in var(H) TODO_ET var(C)$.
	\end{itemize}


