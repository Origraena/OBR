\section{D\'efinitions}

\subsection{Pr\'edicat}
	\begin{itemize}
		\item Un pr\'edicat not\'e $p \backslash n$ est un symbole relationnel d'arit\'e $n$.
		\item On suppose que tout nom de pr\'edicat est unique.
		\item On note $p_{i}$ la $i^{eme}$ position de $p$.
		\item On note $P_{K}$ l'ensemble des pr\'edicats de $K$.
	\end{itemize}

\subsection{Atome}
	Un atome $a = p(a_{1},a_{2},...,a_{n})$ associe un terme \`a chaque position d'un pr\'edicat $p \backslash n$.
	On note :
	\begin{itemize}
		\item $a_{i}$ le terme en position $i$ dans $a$. Un terme peut \^etre une {\em constante} ou une {\em variable}.
		Une variable peut \^etre universelle (not\'ee $\forall-var$) ou existentielle (not\'ee $\exists-var$).
		\item $dom(a) = \{a_{i} : \forall i \in [1,n]\}$, l'ensemble des termes de $a$
		\item $var(a)$ l'ensemble des variables de $a$
		\item $cst(a)$ l'ensemble des constantes de $a$
	\end{itemize}
	Ces notations s'\'etendent aux {\em conjonctions d'atomes}.

\subsection{R\`egle}
	Une r\`egle $R = (H,C)$ est d\'efinie par :
	\begin{itemize}
		\item $H$ : conjonction d'atomes repr\'esentant l'hypoth\`ese de $R$,
		\item $C$ : conjonction d'atomes repr\'esentant la conclusion de $R$.
	\end{itemize}
	On note :
	\begin{itemize}
		\item $dom(R) = dom(H) \cup dom(C)$
		\item $var(R) = var(H) \cup var(C)$
		\item $cst(R) = cst(H) \cup cst(C)$
		\item $fr(R)$ l'ensemble des variables fronti\`eres de $R$, avec $a_{i}$ est une variable fronti\`ere ssi $a_{i}$ est une variable telle que $a_{i} \in var(H) \cap var(C)$.
		\item $cutp(R) = fr(R) \cup cst(R)$ l'ensemble des points de coupure de $R$
	\end{itemize}

\subsection{Repr\'esentation des Conjonctions d'Atomes}
	Une conjonction d'atomes $K = k_{1} \and k_{2} \and ... \and k_{n}$ avec $k_{i} = q(t_{1}, t_{2}, ..., t{p})$ peut
	\^etre repr\'esent\'ee par le graphe non orient\'e $G_{K} = (V_{K},E_{K},\omega)$ avec $V_{K}$ son ensemble de sommets, $E_{K}$,
	son ensemble d'ar\^etes et $\omega$ une fonction de poids sur les ar\^etes construits de la mani\`ere suivante : \\
	$V_{K} = A_{K} \cup T_{K}$ avec $A_{K} = \{i : k_{i} \in K\}$ et $T_{K} = \{t_{j} \in dom(K)\}$ \\
	$E_{K} = \{(i,t_{j}) : \forall k_{i} \in K, \forall t_{j} \in dom(k_{i})\}$ \\
	$\omega(i,t_{j}) = j : \forall (i,t_{j}) \in E_{K}$ \\

	On remarque que quelques propri\'et\'es sont directement induites par cette repr\'esentation :
	\begin{itemize}
		\item $G_{K}$ admet une bipartition de ses sommets, en effet toutes les ar\^etes ont une extr\'emit\'e dans $A_{K}$ etre
		l'autre dans $T_{K}$, or par construction $A_{K} \cap T_{K} = \emptyset$.
	\end{itemize}

\subsection{Impl\'ementation du Graphe}
	Par la suite on supposera que le graphe $G_{K}$ est impl\'ement\'e par deux ensembles de sommets (les atomes et les termes), ainsi que par une liste de voisinage pour chaque sommet.
	De plus, chaque sommet peut avoir connaissance de son type de contenu parmis : $atome$, $terme$, $constante$, $\forall-var$, $\exists-var$
	en temps constant.

	On note $n$ (resp. $t$) le nombre d'atomes (resp. de termes) dans $K$. 

	Op\'erations (et temps d'ex\'ecution associ\'e) :
	\begin{itemize}
		\item Parcourir les sommets $atomes$ : $n$.
		\item Parcourir tous les sommets : $n + t$.
		\item Acc\'eder au $i^{eme}$ voisin d'un sommet : $constant$.
	\end{itemize}

\subsection{Substitution}
	Une substitution de taille $n$ d'un ensemble de symboles $X$ dans un ensemble de symboles $Y$ est une fonction de $X$ 
	vers $Y$ repr\'esent\'ee par l'ensemble de couples suivants (avec $n \leq |X|$) :
	\begin{itemize}
		\item $s = \{(x_{i},y_{i})\ :\ \forall i \in [1,n]\ x_{i} \in X,\ y_{i} \in Y, \forall j \neq i\ x_{i} \neq x_{j}\}$
		\item $s(x_{i}) = y_{i}\ \forall i \in [1,n]$
		\item $s(x_{i}) = x_{i}\ \forall i \in [n+1,|X|]\ x_{i} \in X$
	\end{itemize}

\subsection{Unificateur logique}
	Un unificateur logique entre deux atomes $a_{1}$ et $a_{2}$ est une substitution $\mu$ telle que :
	\begin{itemize}
		\item $\mu : var(a_{1}) \cup var(a_{2}) \rightarrow dom(a_{1}) \cup dom(a_{2})$
		\item $\mu(a_{1}) = \mu(a_{2})$
	\end{itemize}
	Cette d\'efinition s'\'etend aux conjonctions d'atomes.

\subsection{Unificateur de conclusion atomique}
	Un unificateur de conclusion atomique est un unificateur logique $\mu = \{(x_{i},t_{i}) : \forall i \in [1,n]\}$
	entre l'hypoth\`ese d'une r\`egle $R_{1} = (H_{1},C_{1})$ et la conclusion atomique d'une r\`egle $R_{2} = (H_{2},C_{2})$ et est d\'efini de la mani\`ere suivante :
	\begin{itemize}
		\item $\mu : fr(R_{2}) \cup var(H_{1}) \rightarrow dom(C_{2}) \cup cst(H_{1})$
		\item $\forall (x_{i},t_{i}) \in \mu\ si\ x_{i} \in fr(R_{2})\ alors\ t_{i} \in cutp(R_{2}) \cup cst(H_{1})$
	\end{itemize}
    
\subsection{Bonne unification atomique}
	Un bon unificateur de conclusion atomique est un unificateur de conclusion atomique $\mu = \{x_{i},t_{i}) : \forall i \in [1,n]\}$
	entre un sous ensemble $Q$ de l'hypoth\`ese d'une r\`egle $R_{1} = (H_{1},C_{1})$ et la conclusion d'une r\`egle atomique
	$R_{2} = (H_{2},C_{2})$ tel que :\\
	$\forall (x_{i},t_{i}) \in \mu : si\ x_{i} \in H_{1} \setminus Q\ alors\ t_{i}\ n'est\ pas\ une\ \exists-var$

	Un tel ensemble $Q$ est appel\'e un {\em bon ensemble d'unification atomique} de l'hypoth\`ese de $R_{1}$ par la conclusion de $R_{2}$.
	On note que $Q$ est donc d\'efini comme suit :
	\begin{itemize}
		\item $Q \subseteq H_{1}$
		\item $\forall\ position\ i\ de\ \exists-var\ dans\ C_{2}, \forall\ atome\ a\ \in Q, si\ a_i \in var(H_{1})\ alors\ \forall\ atome\ b \in H_{1} : si\ \exists\ b_j \in b : a_i = b_j, alors\ b \in Q$
	\end{itemize}
.

	Un {\em bon ensemble d'unification atomique minimal} $Q$ de $H_{1}$ par $C_{2}$ enracin\'e en $a$ est d\'efini tel que :
	\begin{itemize}
		\item $Q$ est un bon ensemble d'unification atomique de $H_{1}$ par $C_{2}$
		\item $a \in Q$
		\item $|Q| = min(|Q_i| : Q_i\ est\ un\ bon\ ensemble\ d'unification\ atomique\ de\ H_1\ par\ C_2\ et\ a \in Q_i)$
	\end{itemize}


