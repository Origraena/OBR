\begin{center}
\begin{algorithm}[H]
\caption{Extension}\label{extension}
\algsetup{indent=2em,linenodelimiter= }
\begin{algorithmic}[1]
\REQUIRE	$H_{1}$ : conjonction d'atomes, 
			$a \in H_{1}$ : sommet atome racine, 
			$couleurLogique$ : tableau de taille \'egal au nombre d'atomes dans $H_{1}$ tel que $couleurLogique[a] = noir$ ssi $UnificationLocale(a,R) =$ succ\`es, 
			$E$ : ensemble des positions des variables existentielles
\ENSURE $Q$ : bon ensemble d'unification minimal des atomes de ${H_1}$ construit \`a partir de $a$ s'il existe, \'echec sinon.
\STATE $\triangleright$ Initialisation du parcours
%\FORALL {$sommet\ v \in H_{1}$}
%	\STATE $couleur[v] \leftarrow blanc$   $\triangleright$ coloration pour le parcours
%\ENDFOR
\FORALL {$sommet\ atome\ a \in H_{1}$}
	\IF {$couleurLogique[a] = noir$}
		\FORALL {$sommet\ terme\ t\ \in voisins(a)$}
			\STATE $couleur[t] = blanc$
		\ENDFOR
		\STATE $couleur[a] = blanc$
	\ENDIF
\ENDFOR
\STATE $couleur[a] \leftarrow noir$
\STATE $Q \leftarrow \{a\}$   $\triangleright$ conjonction d'atomes \`a traiter
\STATE $attente \leftarrow \{a\}$   $\triangleright$ file d'attente du parcours
\WHILE {$attente \neq \emptyset$}
	\STATE $u \leftarrow haut(attente)$
	\IF {$u\ est\ un\ atome$}
		\FORALL {$i \in E$}
			\STATE $v \leftarrow voisin(u,i)$
			\IF {$v\ est\ une\ constante$}
				\RETURN \'echec	
			\ELSIF {$couleur[v] = blanc$}
				\STATE $\triangleright$ v est une $\forall-var$ non marqu\'ee par le parcours
				\STATE $couleur[v] \leftarrow noir$
				\STATE $attente \leftarrow attente \cup \{v\}$
			\ENDIF
		\ENDFOR
	\ELSE
		\STATE $\triangleright$ u est un terme
		\FORALL {$v \in voisins(u)$}
			\IF {$couleurLogique[v] = blanc$}
				\RETURN \'echec
			\ELSE
				\IF {$couleur[v] = blanc$}
					\STATE $couleur[v] \leftarrow noir$
					\STATE $attente \leftarrow attente \cup \{v\}$
					\STATE $Q \leftarrow Q \cup \{v\}$
				\ENDIF
			\ENDIF
		\ENDFOR
	\ENDIF
\ENDWHILE $\ \ \ \triangleright$ Fin du parcours
\RETURN $Q$
\end{algorithmic}
\end{algorithm}
\end{center}

Remarque : \\
La phase d'initialisation du parcours pourrait simplement parcourir tous les sommets de $H_{1}$ et mettre leur couleur \`a blanc.
En pratique, cette solution serait sans doute plus efficace, mais d\'ependrait donc du nombre de sommets total dans $H_{1}$.
Ce qui en th\'eorie am\`enerait la complexit\'e de cette boucle en $\bigcirc(nombre\ d'atomes\ \times arite\ max\ de\ H_{1})$.
Or ici, la complexit\'e ne d\'epend pas de cette arit\'e max, mais uniquement de l'arit\'e du pr\'edicat de la conclusion $C_{2}$.

