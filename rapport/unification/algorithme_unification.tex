\begin{center}
\begin{algorithm}[H]
\caption{Unification}\label{unification}
\algsetup{indent=2em,linenodelimiter= }
\begin{algorithmic}[1]
\REQUIRE $H_{1}$ : conjonction d'atomes, $R = (H_{2},C_{2})$ : r\`egle \`a conclusion atomique 
\ENSURE succ\`es si $C_{2}$ peut s'unifier avec $H_{1}$, i.e. si $\exists H \subseteq H_{1}, \mu\ une\ substitution : \mu(H_{1}) = \mu(C_{2})$, \'echec sinon 
\STATE $\triangleright$ Pr\'ecoloration
\FORALL {$sommet\ atome\ a \in H_{1}$}
	\IF {$UnificationLocale(a,R) \neq$ \'echec}
		\STATE $couleurLogique[a] \leftarrow noir$
	\ELSE
		\STATE $couleurLogique[a] \leftarrow blanc$
	\ENDIF
\ENDFOR 

\STATE $\triangleright$ Initialisation du tableau contenant les positions des variables existentielles de $C_{2}$
\STATE $E \leftarrow \{i : c_{i}\ est\ une\ \exists-var\ de\ C_{2}\}$

\STATE $\triangleright$ Extension des ensembles
\FORALL {$sommet\ atome\ a \in H_{1} : couleurLogique[a] = noir$}
	\IF {$Q \leftarrow Extension(H_{1},a,couleurLogique,E) \neq $ \'echec} 
		\IF {$UnificationLocale(Q,R) \neq$ \'echec}
			\RETURN succ\`es
		\ENDIF
	\ENDIF
	\STATE $couleurLogique[a] \leftarrow blanc$
\ENDFOR
\RETURN \'echec
\end{algorithmic}
\end{algorithm}
\end{center}

