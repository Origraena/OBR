% dev.tex

\begin{frame}
	\frametitle{Structures de données}
	\begin{itemize}
		\item Différents types de graphes
		\item Structure générique et algorithmes
		\item Sommets et arcs de types paramétrables
	\end{itemize}
\end{frame}

\begin{frame}
	\frametitle{Structures de données}
	\begin{itemize}
		\item Règles : graphe biparti non orienté (prédicats et termes, position du terme)
		\item Dépendances des règles : graphe orienté (règles, aucun)
		\item Composantes fortement connexes : graphe orienté (ensembles de sommets,
		aucun) sans circuit
		\item Dépendances des positions : graphe orienté (positions des prédicats,
		spécial ou non)
	\end{itemize}
\end{frame}

\begin{frame}
	\frametitle{Analyseur}
	\begin{itemize}
		\item Divisé en deux parties
		\item Détermination des classes concrètes
		\item Combinaison des classes abstraites
		\item Règles à conclusion atomique (sans perte de généralité)
	\end{itemize}
\end{frame}

\begin{frame}
	\frametitle{Détermination des classes concrètes}
	\begin{itemize}
		\item Possibilité d'ajouter de nouveaux tests
		\item Calcul sur l'ensemble des règles, puis sur chaque composante connexe
		\item Renvoient des étiquettes
		\item Indiquent les classes abstraites satisfaites
	\end{itemize}
\end{frame}

\begin{frame}
	\frametitle{Combinaison des classes abstraites}
	\begin{itemize}
		\item Vérification de l'ensemble des règles
		\item Etiquettes des classes abstraites valuées
		\item Parcours du graphe des composantes à partir des sources
		\item Attribue à chaque composante l'étiquette la plus {\em petite} compatible
		\item Décidable si tous les sommets sont étiquetés
	%	\item Liste des types d'algorithmes à utiliser
	\end{itemize}
\end{frame}

\begin{frame}
	\frametitle{Fichiers}
	\begin{itemize}
		\item Lecture et écriture via un format interne
		\item Conversion des fichiers Datalog (.dtg)
		\item Sortie PostScript pour l'affichage des graphes
	\end{itemize}
\end{frame}

