% atome.tex

\begin{frame}
	\frametitle{Atome}
	\begin{center}
	\begin{itemize}
		\item Prédicat : symbole relationnel d'arité donnée
		\item Atome : prédicat et termes associés à ses positions
		\item Terme : variable ou constante (pas de fonction)
	\end{itemize}
	\vspace{10mm}
	\begin{exampleblock}{Exemple}
	"$x$ est le {\em père} de {\em Tom}."\\
	père($x$,Tom)
	\begin{itemize}
		\item prédicat : père
		\item variable $x$ en position 1
		\item constante $Tom$ en position 2
	\end{itemize}
	\end{exampleblock}
	\end{center}
\end{frame}

\begin{frame}
	\frametitle{Faits}
	\begin{center}
	\vspace{10mm}
	\begin{itemize}
		\item Conjonction de $k$ atomes
		\item Existentiellement fermée
		\item $A = \exists (atome_1 \wedge atome_2 \wedge ... \wedge atome_k)$
%		\item Quantification des variables : universellement ou existentiellement
%	\item Représentation par un graphe non orienté
	\end{itemize}
	\vfill
	\begin{exampleblock}{Exemple}
		"Il existe un homme qui est le père de Tom."\\
		$\exists x$ (homme($x$) $\wedge$ pere($x$,Tom))
	\end{exampleblock}
	\end{center}
	\vspace{10mm}
\end{frame}

