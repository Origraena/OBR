% regle.tex

\begin{frame}[t]
	\frametitle{Règle}
	\vspace{10mm}
	\begin{itemize}
		\item Deux conjonctions d'atomes : une hypothèse H et une conclusion C
		\item $H \rightarrow C$
		\item Variable $x$ soit universelle ($x \in H$) ou existentielle ($x \notin H$)
		\item Frontière : variable à la fois dans H et dans C ($x \in H \cap C$)\\
	\end{itemize}
\end{frame}

\begin{frame}[t]
	\frametitle{Règle}
	\vspace{10mm}
	\begin{itemize}
		\item Deux conjonctions d'atomes : une hypothèse H et une conclusion C
		\item $H \rightarrow C$
		\item Variable $x$ soit universelle ($x \in H$) ou existentielle ($x \notin H$)
		\item Frontière : variable à la fois dans H et dans C ($x \in H \cap C$)
	\end{itemize}
	\vfill
	\begin{exampleblock}{Exemple}
		"Tout homme a un père qui est un homme."\\
		$\forall x$ (homme($x$) $\rightarrow \exists z$ (homme($z$) $\wedge$ père($z$,$x$)))
		\begin{itemize}
			\item Hypothèse : $\forall x$ (homme($x$))
			\item Conclusion : $\forall x \exists z$ (homme($z$) $\wedge$ père($z$,$x$))
		\end{itemize}
	\end{exampleblock}
	\vspace{10mm}
\end{frame}

%%%%\begin{exampleblock}{Exemple}
%%%%	\begin{center}
%%%%	"Tout humain a un père qui est un homme."\\
%%%%	$\forall x$ (humain($x$) $\rightarrow \exists z$ (homme($z$) $\wedge$
%%%%	pere($z$,$x$)))
%%%%	\end{center}
%%%%\end{exampleblock}



