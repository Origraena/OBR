% logique.tex

\subsection{Atomes}
\begin{frame}
	\frametitle{Atome}
	\begin{center}
	\begin{itemize}
		\item Prédicat : symbole relationnel d'arité donnée
		\item Atome : prédicat et termes associés à ses positions
		\item Terme : variable ou constante (pas de fonction)
		\item Domaine d'un atome : ensemble de ses termes
	\end{itemize}
	\vspace{10mm}
	\begin{exampleblock}{Exemple}
	"Tom a un père."\\
	$\exists x$ (pere($x$,Tom))
	\end{exampleblock}
	\end{center}
\end{frame}

\begin{frame}
	\frametitle{Conjonction d'atomes}
	\begin{center}
	\begin{itemize}
		\item Composée de $k$ atomes
		\item $A = atome_1 \wedge atome_2 \wedge ... \wedge atome_k$
		\item Représentation par un graphe non orienté
	\end{itemize}
	\vspace{10mm}
	\begin{exampleblock}{Exemple}
		"Il existe un homme qui est le père de Tom."\\
		$\exists x$ (homme($x$) $\wedge$ pere($x$,Tom))
	\end{exampleblock}
	\end{center}
\end{frame}

\begin{frame}
	\frametitle{Représentation graphique d'une conjonction d'atomes}

	\begin{columns}
	
	\begin{column}{0.45\linewidth}
		On crée :
		\begin{itemize}
			\item un sommet par atome étiqueté par son prédicat 
			\item un sommet par terme étiqueté si constante 
			\item une arête pour chaque apparition de terme dans un atome 
			dont le poids est la position du terme 
		\end{itemize}
	\end{column}
	\vline
	\hfill
	\begin{column}{0.45\linewidth}
		$\exists x$ (homme($x$) $\wedge$ pere($x$,Tom))
	\end{column}
	\end{columns}
\end{frame}

\begin{frame}
	\frametitle{Représentation graphique d'une conjonction d'atomes}

	\begin{columns}
	
	\begin{column}{0.45\linewidth}
		On crée :
		\begin{itemize}
			\item un sommet par atome étiqueté par son prédicat \checkmark
			\item un sommet par terme étiqueté si constante 
			\item une arête pour chaque apparition de terme dans un atome 
			dont le poids est la position du terme 
		\end{itemize}
	\end{column}
	\vline
	\hfill
	\begin{column}{0.45\linewidth}
		$\exists x$ (homme($x$) $\wedge$ pere($x$,Tom))
		\begin{figure}
			\begin{tikzpicture}[
				auto,
				inner sep=1mm
			]
			\node [atom] (pere) at (3,0) {pere$\backslash$2};
			\node [atom] (homme) at (-0.2,0) {homme$\backslash$1};
			\end{tikzpicture}
		\end{figure}
	\end{column}
	\end{columns}
\end{frame}

\begin{frame}
	\frametitle{Représentation graphique d'une conjonction d'atomes}

	\begin{columns}
	
	\begin{column}{0.45\linewidth}
		On crée :
		\begin{itemize}
			\item un sommet par atome étiqueté par son prédicat \checkmark
			\item un sommet par terme étiqueté si constante \checkmark
			\item une arête pour chaque apparition de terme dans un atome 
			dont le poids est la position du terme 
		\end{itemize}
	\end{column}
	\vline
	\hfill
	\begin{column}{0.45\linewidth}
		$\exists x$ (homme($x$) $\wedge$ pere($x$,Tom))
		\begin{figure}
			\begin{tikzpicture}[
				auto,
				inner sep=1mm
			]
			\node [atom] (pere) at (3,0) {pere$\backslash$2};
			\node [atom] (homme) at (-0.2,0) {homme$\backslash$1};
			\node[term]	(x) at (1.5,0) {}
			;
			\node[term,inner sep=0mm] (Tom) at (3,-1.5) {Tom}
			;
			\end{tikzpicture}
		\end{figure}
	\end{column}
	\end{columns}
\end{frame}

\begin{frame}
	\frametitle{Représentation graphique d'une conjonction d'atomes}

	\begin{columns}
	
	\begin{column}{0.45\linewidth}
		On crée :
		\begin{itemize}
			\item un sommet par atome étiqueté par son prédicat \checkmark
			\item un sommet par terme étiqueté si constante \checkmark
			\item une arête pour chaque apparition de terme dans un atome 
			dont le poids est la position du terme \checkmark
		\end{itemize}
	\end{column}
	\vline
	\hfill
	\begin{column}{0.45\linewidth}
		$\exists x$ (homme($x$) $\wedge$ pere($x$,Tom))
		\begin{figure}
			\begin{tikzpicture}[
				auto,
				inner sep=1mm
			]
			\node [atom] (pere) at (3,0) {pere$\backslash$2};
			\node [atom] (homme) at (-0.2,0) {homme$\backslash$1};
			\node[term]	(x) at (1.5,0) {}
				edge[atomedge] node {1} (pere)
				edge[atomedge] node {1} (homme)
			;
			\node[term,inner sep=0mm]	(Tom) at (3,-1.5) {Tom}
				edge[atomedge] node {2} (pere)
			;
			\end{tikzpicture}
		\end{figure}
	\end{column}
	\end{columns}
\end{frame}

%%%%	\begin{frame}
%%%%		\frametitle{Représentation graphique d'une conjonction d'atomes}
%%%%		\begin{center}
%%%%		$\exists x$ (homme($x$) $\wedge$ pere($x$,Tom))
%%%%		\begin{figure}[h]
%%%%			\begin{tikzpicture}[
%%%%				auto,
%%%%				inner sep=1mm,
%%%%			]
%%%%			\node [atom] (pere) at (3,0) {pere$\backslash$2};
%%%%			\node [atom] (homme) at (-0.2,0) {homme$\backslash$1};
%%%%			\node[term]	(x) at (1.5,0) {}
%%%%				edge[atomedge] node {1} (pere)
%%%%				edge[atomedge] node {1} (homme)
%%%%			;
%%%%			\node[term,inner sep=0mm]	(Tom) at (3,-1.5) {Tom}
%%%%				edge[atomedge] node {2} (pere)
%%%%			;
%%%%			\end{tikzpicture}
%%%%		\end{figure}
%%%%		\end{center}
%%%%	\end{frame}



