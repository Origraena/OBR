\documentclass[9pt]{beamer}

% packages
\usepackage[french]{babel}       
\usepackage[utf8]{inputenc}
\usepackage{setspace}
\usepackage{listings}
\usepackage{lscape}
\usepackage{graphicx}
\usepackage{geometry}
\usepackage{hyperref}
\usepackage{url}
\usepackage{enumerate}
\usepackage{amsmath}
\usepackage{amssymb}
\usepackage{tikz}

% commands
\newcommand{\N}[0]{\mathbb{N}}
\newcommand{\code}[1]{{\em #1}}

% tikz_config.tex

\tikzstyle{atom}=[
	rectangle,
	draw
]

\tikzstyle{atomhead}=[
	rectangle,
	draw,
	fill=yellow!20
]

\tikzstyle{atombody}=[
	rectangle,
	draw,
	fill=blue!20
]

\tikzstyle{term}=[
	circle,
	draw,
	minimum size=5mm
]

\tikzstyle{termfr}=[
	circle,
	draw,
	fill=green!20,
	minimum size=5mm
]

\tikzstyle{termhead}=[
	circle,
	draw,
	fill=yellow!20,
	minimum size=5mm
]

\tikzstyle{termcst}=[
	circle,
	draw,
	fill=red!20,
	minimum size=5mm
]

\tikzstyle{atomedge}=[
	-
]

\tikzstyle{class}=[
	rectangle,
	draw,
	fill=black!10,
	minimum size=5mm
]

\tikzstyle{classedge}=[
	->
]

\newcommand{\atomlegend}[2]{
	\node [rectangle,draw,fill=blue!20,minimum width=0.8cm] (body) at(#1,#2) {};
	\node [] (bodyLabel)							at(#1+1.4,#2)			{corps};
	\node [rectangle,draw,fill=yellow!20,minimum width=0.8cm] (head) at(#1,#2-0.5) {};
	\node [] (headLabel)							at(#1+1.4,#2-0.5)				{tête};
	\node [rectangle,draw,fill=green!20,minimum width=0.8cm] (fr) at(#1,#2-1) {};
	\node [] (frLabel)							at(#1+1.4,#2-1)				{frontière};
	\node [rectangle,draw,fill=red!20,minimum width=0.8cm] (cst) at(#1,#2-1.5) {};
	\node [] (cstLabel)							at(#1+1.4,#2-1.5)				{constantes};
}




%\geometry{left=1.5cm,right=1cm,marginparwidth=0.5cm,marginparsep=2mm,top=1cm,bottom=1cm}
\hypersetup{urlcolor=blue,linkcolor=black,citecolor=black,colorlinks=true}
\definecolor{listinggray}{gray}{0.9}
\definecolor{lbcolor}{rgb}{0.95,0.95,0.95}
\definecolor{titlegrey}{gray}{0.9}
\colorlet{blockred}{red!30!black!15}
\colorlet{blocktitlered}{red!40!black!30}
\colorlet{titlered}{red!66!black}

\setbeamertemplate{navigation symbols}{}

\useinnertheme[shadow=true]{rounded} 
\useoutertheme{infolines}
\usecolortheme{beaver}

\setbeamerfont{block title}{size={}}
\setbeamercolor{titlelike}{parent=structure,bg=titlegrey}

\setbeamercolor{block body example}{bg=blockred}
\setbeamercolor{block title example}{fg=titlered,bg=blocktitlered}

\title[Outils d'analyse d'une base de règles]{Outils d'Analyse d'une Base de Règles}
\author[]{Swan Rocher}
\institute{Université Montpellier 2}
\date{\today}

\hypersetup{ 
pdfpagemode = FullScreen,
pdfauthor = {Swan Rocher},
pdftitle = {Outils d'analyse d'une base de règles - présentation},
}

\AtBeginSection[]{ 
	\begin{frame} 
		\frametitle{Table des matières}
		\tableofcontents[currentsection,hideallsubsections] 
	\end{frame} 
}

\AtBeginSubsection[]{ 
	\begin{frame} 
		\frametitle{Table des matières}
		\tableofcontents[currentsection,currentsubsection,hideothersubsections] 
	\end{frame} 
}

% exemples.tex


\newcommand{\ruleA}[0]{
$R_1$ : $\forall x$ (homme($x$) $\rightarrow$ humain($x$))
}

\newcommand{\ruleB}[0]{
$R_2$ : $\forall x$ (humain($x$) 
$\rightarrow$ $\exists z$ (homme($z$) $\wedge$ père($z$,$x$)))
}

\newcommand{\ruleC}[0]{
$R_3$ : $\forall x \forall y$ (parent($x$,$y$) $\wedge$ homme($x$) 
$\rightarrow$ père($x$,$y$))
}

\newcommand{\ruleD}[0]{
$R_4$ : $\forall x \forall y$ (père($x$,$y$) $\rightarrow$ parent($x$,$y$))
}

\newcommand{\ruleE}[0]{
$R_5$ : $\forall x \forall y \forall z$ (mêmeFamille($x$,$y$) $\wedge$ mêmeFamille($y$,$z$)
$\rightarrow$ mêmeFamille($x$,$z$))
}



\begin{document}
\Large
%\setlength{\parskip}{10mm plus5mm minus5mm}
%\renewenvironment{itemize}{\setlength{\itemsep}{10mm}}
\lstset{
	backgroundcolor=\color{lbcolor},
	tabsize=4,
	rulecolor=,
	language=Java,
		basicstyle=\scriptsize,
		numbers=left,
		numberstyle=\tiny,
        aboveskip={1.5\baselineskip},
        columns=fixed,
        showstringspaces=false,
        extendedchars=true,
        breaklines=true,
        prebreak = \raisebox{0ex}[0ex][0ex]{\ensuremath{\hookleftarrow}},
        frame=single,
        showtabs=false,
        showspaces=false,
        showstringspaces=false,
        identifierstyle=\ttfamily,
        keywordstyle=\color[rgb]{1,0,0},
        commentstyle=\color[rgb]{0,0,1},
        stringstyle=\color[rgb]{0.627,0.126,0.941},
}


% TOC
%\setcounter{tocdepth}{1}
%\tableofcontents

% title.tex

\begin{frame}
	\titlepage
\end{frame}

\begin{frame}
	\frametitle{Table des matières}
	\tableofcontents[hideothersubsections]
\end{frame}



\section{Contexte}
% contexte.tex

Base de données sans ontologie => ne permet pas de répondre à la requête-exemple.
On ajoute l'ontologie => on peut déduire de nouveaux faits, ...

Requête / base de connaissance (vision globale).

Universelles / Existentielles.

Non décidabilité de la réponse à une requête dans une base contenant des existentielles.

Exemples avec des MOTS.

% problematique.tex

\begin{frame}
	\frametitle{Problématique}
	\begin{itemize}
		\item Développement d'un outil analysant une base de règles
		\item Construction du graphe de dépendances associé
		\item Détermination des classes de règles 
		\item Décidabilité de la base
		\item Déduction des algorithmes à utiliser
		\item Lecture et écriture d'une base à partir et vers un fichier
		\item Langage Java
	\end{itemize}
\end{frame}




% problematique.tex

\begin{frame}
	\frametitle{Problématique}
	\begin{itemize}
		\item Développement d'un outil analysant une base de règles
		\item Construction du graphe de dépendances associé
		\item Détermination des classes de règles 
		\item Décidabilité de la base
		\item Déduction des algorithmes à utiliser
		\item Lecture et écriture d'une base à partir et vers un fichier
		\item Langage Java
	\end{itemize}
\end{frame}



\section{Notions de logique}
% logique.tex

\subsection{Atomes}
\begin{frame}
	\frametitle{Atome}
	\begin{center}
	\begin{itemize}
		\item Prédicat : symbole relationnel d'arité donnée
		\item Atome : prédicat et termes associés à ses positions
		\item Terme : variable ou constante (pas de fonction)
		\item Domaine d'un atome : ensemble de ses termes
	\end{itemize}
	\begin{exampleblock}{Exemple}
	"Tom a un père."\\
	$\exists x$ (pere($x$,Tom))
	\end{exampleblock}
	\end{center}
\end{frame}

\begin{frame}
	\frametitle{Conjonction d'atomes}
	\begin{center}
	\begin{itemize}
		\item Composée de $k$ atomes
		\item $A = atome_1 \wedge atome_2 \wedge ... \wedge atome_k$
		\item Représentation par un graphe non orienté
	\end{itemize}
	\begin{exampleblock}{Exemple}
		"Il existe un homme qui est le père de Tom."\\
		$\exists x$ (homme($x$) $\wedge$ pere($x$,Tom))
	\end{exampleblock}
	\end{center}
\end{frame}

\begin{frame}
	\frametitle{Représentation graphique d'une conjonction d'atomes}

	\begin{columns}
	
	\begin{column}{0.45\linewidth}
		On crée :
		\begin{itemize}
			\item un sommet par atome étiqueté par son prédicat 
			\item un sommet par terme étiqueté si constante 
			\item une arête pour chaque apparition de terme dans un atome 
			dont le poids est la position du terme 
		\end{itemize}
	\end{column}
	\vline
	\hfill
	\begin{column}{0.45\linewidth}
		$\exists x$ (homme($x$) $\wedge$ pere($x$,Tom))
	\end{column}
	\end{columns}
\end{frame}

\begin{frame}
	\frametitle{Représentation graphique d'une conjonction d'atomes}

	\begin{columns}
	
	\begin{column}{0.45\linewidth}
		On crée :
		\begin{itemize}
			\item un sommet par atome étiqueté par son prédicat \checkmark
			\item un sommet par terme étiqueté si constante 
			\item une arête pour chaque apparition de terme dans un atome 
			dont le poids est la position du terme 
		\end{itemize}
	\end{column}
	\vline
	\hfill
	\begin{column}{0.45\linewidth}
		$\exists x$ (homme($x$) $\wedge$ pere($x$,Tom))
		\begin{figure}
			\begin{tikzpicture}[
				auto,
				inner sep=1mm
			]
			\node [atom] (pere) at (3,0) {pere$\backslash$2};
			\node [atom] (homme) at (-0.2,0) {homme$\backslash$1};
			\end{tikzpicture}
		\end{figure}
	\end{column}
	\end{columns}
\end{frame}

\begin{frame}
	\frametitle{Représentation graphique d'une conjonction d'atomes}

	\begin{columns}
	
	\begin{column}{0.45\linewidth}
		On crée :
		\begin{itemize}
			\item un sommet par atome étiqueté par son prédicat \checkmark
			\item un sommet par terme étiqueté si constante \checkmark
			\item une arête pour chaque apparition de terme dans un atome 
			dont le poids est la position du terme 
		\end{itemize}
	\end{column}
	\vline
	\hfill
	\begin{column}{0.45\linewidth}
		$\exists x$ (homme($x$) $\wedge$ pere($x$,Tom))
		\begin{figure}
			\begin{tikzpicture}[
				auto,
				inner sep=1mm
			]
			\node [atom] (pere) at (3,0) {pere$\backslash$2};
			\node [atom] (homme) at (-0.2,0) {homme$\backslash$1};
			\node[term]	(x) at (1.5,0) {}
			;
			\node[term,inner sep=0mm] (Tom) at (3,-1.5) {Tom}
			;
			\end{tikzpicture}
		\end{figure}
	\end{column}
	\end{columns}
\end{frame}

\begin{frame}
	\frametitle{Représentation graphique d'une conjonction d'atomes}

	\begin{columns}
	
	\begin{column}{0.45\linewidth}
		On crée :
		\begin{itemize}
			\item un sommet par atome étiqueté par son prédicat \checkmark
			\item un sommet par terme étiqueté si constante \checkmark
			\item une arête pour chaque apparition de terme dans un atome 
			dont le poids est la position du terme \checkmark
		\end{itemize}
	\end{column}
	\vline
	\hfill
	\begin{column}{0.45\linewidth}
		$\exists x$ (homme($x$) $\wedge$ pere($x$,Tom))
		\begin{figure}
			\begin{tikzpicture}[
				auto,
				inner sep=1mm
			]
			\node [atom] (pere) at (3,0) {pere$\backslash$2};
			\node [atom] (homme) at (-0.2,0) {homme$\backslash$1};
			\node[term]	(x) at (1.5,0) {}
				edge[atomedge] node {1} (pere)
				edge[atomedge] node {1} (homme)
			;
			\node[term,inner sep=0mm]	(Tom) at (3,-1.5) {Tom}
				edge[atomedge] node {2} (pere)
			;
			\end{tikzpicture}
		\end{figure}
	\end{column}
	\end{columns}
\end{frame}

%%%%	\begin{frame}
%%%%		\frametitle{Représentation graphique d'une conjonction d'atomes}
%%%%		\begin{center}
%%%%		$\exists x$ (homme($x$) $\wedge$ pere($x$,Tom))
%%%%		\begin{figure}[h]
%%%%			\begin{tikzpicture}[
%%%%				auto,
%%%%				inner sep=1mm,
%%%%			]
%%%%			\node [atom] (pere) at (3,0) {pere$\backslash$2};
%%%%			\node [atom] (homme) at (-0.2,0) {homme$\backslash$1};
%%%%			\node[term]	(x) at (1.5,0) {}
%%%%				edge[atomedge] node {1} (pere)
%%%%				edge[atomedge] node {1} (homme)
%%%%			;
%%%%			\node[term,inner sep=0mm]	(Tom) at (3,-1.5) {Tom}
%%%%				edge[atomedge] node {2} (pere)
%%%%			;
%%%%			\end{tikzpicture}
%%%%		\end{figure}
%%%%		\end{center}
%%%%	\end{frame}




% regle.tex

\subsection{Règles}
\begin{frame}[t]
	\frametitle{Règle}
	\vspace{10mm}
	\begin{center}
	\begin{itemize}
		\item Deux conjonctions d'atomes : une hypothèse H et une conclusion C
		\item $H \rightarrow C$
		\item Variable soit universelle ($\in H$) ou existentielle ($\notin H$)
		\item Frontière : variable à la fois dans H et dans C ($\in H \cap C$)\\
	\end{itemize}
	\end{center}
\end{frame}

\begin{frame}[t]
	\frametitle{Règle}
	\vspace{10mm}
	\begin{center}
	\begin{itemize}
		\item Deux conjonctions d'atomes : une hypothèse H et une conclusion C
		\item $H \rightarrow C$
		\item Variable soit universelle ($\in H$) ou existentielle ($\notin H$)
		\item Frontière : variable à la fois dans H et dans C ($\in H \cap C$)\\
	\end{itemize}

	\begin{exampleblock}{Exemple règle universelle}
		"Tout homme est un humain."\\
		$\forall x$ (homme($x$) $\rightarrow$ humain($x$))
	\end{exampleblock}

	\end{center}
\end{frame}

\begin{frame}[t]
	\frametitle{Règle}
	\vspace{10mm}
	\begin{center}
	\begin{itemize}
		\item Deux conjonctions d'atomes : une hypothèse H et une conclusion C
		\item $H \rightarrow C$
		\item Variable soit universelle ($\in H$) ou existentielle ($\notin H$)
		\item Frontière : variable à la fois dans H et dans C ($\in H \cap C$)\\
	\end{itemize}

	\begin{exampleblock}{Exemple règle universelle}
		"Tout homme est un humain."\\
		$\forall x$ (homme($x$) $\rightarrow$ humain($x$))
	\end{exampleblock}

	\begin{exampleblock}{Exemple règle existentielle}
		"Tout humain a un père qui est un homme."\\
		$\forall x$ (humain($x$) $\rightarrow \exists z$ (homme($z$) $\wedge$ pere($z$,Tom)))
	\end{exampleblock}
	\end{center}
\end{frame}

\begin{frame}
	\frametitle{Représentation graphique d'une règle}
	\begin{center}
	\begin{itemize}
		\item Ensembles de sommets et d'arêtes identiques à une conjonction d'atomes
		\item 2-coloration des atomes pour différencier hypothèse et conclusion
	\end{itemize}
	\begin{exampleblock}{Exemple}
		\begin{center}
		"Tout humain a un père qui est un homme."\\
		$\forall x$ (humain($x$) $\rightarrow \exists z$ (homme($z$) $\wedge$
		pere($z$,$x$)))
		\end{center}
	\end{exampleblock}
	\end{center}
\end{frame}

\begin{frame}
	\frametitle{Représentation graphique d'une règle}

	\begin{columns}
	
	\begin{column}{0.45\linewidth}
		\begin{itemize}
			\item un sommet par atome étiqueté par son prédicat \checkmark
			\item un sommet par terme étiqueté si constante \checkmark
			\item une arête pour chaque apparition de terme dans un atome 
			dont le poids est la position du terme \checkmark
			\item coloration des sommets atomes en fonction de leur position
		\end{itemize}
	\end{column}
	\vline
	\hfill
	\begin{column}{0.45\linewidth}
		$\forall x$ (humain($x$) $\rightarrow \exists z$ (homme($z$) $\wedge$
		pere($z$,$x$)))
		\begin{figure}
			\begin{tikzpicture}[
				auto,
				inner sep=1mm
			]
			\node [atom] (humain) at (0,-1.5) {humain$\backslash$1};
			\node [atom] (pere) at (3,0) {pere$\backslash$2};
			\node [atom] (homme) at (-0.2,0) {homme$\backslash$1};
			\node[term]	(z) at (1.5,0) {}
				edge[atomedge] node {1} (pere)
				edge[atomedge] node {1} (homme)
			;
			\node[term] (x) at (3,-1.5) {}
				edge[atomedge] node {1} (humain)
				edge[atomedge] node {2} (pere)
			;
			\end{tikzpicture}
		\end{figure}
	\end{column}
	\end{columns}
\end{frame}

\begin{frame}
	\frametitle{Représentation graphique d'une règle}

	\begin{columns}
	
	\begin{column}{0.45\linewidth}
		\begin{enumerate}
			\item un sommet par atome étiqueté par son prédicat \checkmark
			\item un sommet par terme étiqueté si constante \checkmark
			\item une arête pour chaque apparition de terme dans un atome 
			dont le poids est la position du terme \checkmark
			\item coloration des sommets atomes en fonction de leur position \checkmark
		\end{enumerate}
	\end{column}
	\vline
	\hfill
	\begin{column}{0.45\linewidth}
		$\forall x$ (humain($x$) $\rightarrow \exists z$ (homme($z$) $\wedge$
		pere($z$,$x$)))
		\begin{figure}
			\begin{tikzpicture}[
				auto,
				inner sep=1mm
			]
			\node [atombody] (humain) at (0,-1.5) {humain$\backslash$1};
			\node [atomhead] (pere) at (3,0) {pere$\backslash$2};
			\node [atomhead] (homme) at (-0.2,0) {homme$\backslash$1};
			\node[term]	(z) at (1.5,0) {}
				edge[atomedge] node {1} (pere)
				edge[atomedge] node {1} (homme)
			;
			\node[term] (x) at (3,-1.5) {}
				edge[atomedge] node {1} (humain)
				edge[atomedge] node {2} (pere)
			;
			\lightatomlegend{0}{-2.5}
			\end{tikzpicture}
		\end{figure}
	\end{column}

	\end{columns}
\end{frame}

\begin{frame}
	\frametitle{Les différents éléments d'une règle}
	\begin{center}
	$\forall x$ (humain($x$) $\rightarrow \exists z$ (homme($z$) $\wedge$
	pere($z$,$x$)))
	\end{center}
	\begin{figure}
		\begin{tikzpicture}[
			auto,
			inner sep=1mm
		]
		\node [atombody] (humain) at (0,-1.5) {humain$\backslash$1};
		\node [atomhead] (pere) at (3,0) {pere$\backslash$2};
		\node [atomhead] (homme) at (-0.2,0) {homme$\backslash$1};
		\node[termhead]	(z) at (1.5,0) {}
			edge[atomedge] node {1} (pere)
			edge[atomedge] node {1} (homme)
		;
		\node[termfr] (x) at (3,-1.5) {}
			edge[atomedge] node {1} (humain)
			edge[atomedge] node {2} (pere)
		;
		\atomlegend{-4}{0}
		\end{tikzpicture}
	\end{figure}
\end{frame}


% algos.tex

\subsection{Requêtes}

\begin{frame}
	\frametitle{Base de connaissance}
	\begin{itemize}
		\item Notée $K = (F,R)$
		\item Ensemble de faits représenté par une conjonction d'atomes $F$
		\item Ensemble de règles $R$
		\item Requête $Q$ : conjonction d'atomes existentiellement fermée
	\end{itemize}
\end{frame}

\begin{frame}
	\frametitle{Chaînage avant}
	\begin{itemize}
		\item Génération de nouveaux faits à partir des précédents et de l'ontologie
		\item A chaque création de fait, vérification de la présence de la requête dans
		$F$
		\item Arrêt lorsque tous les faits sont générés ou si réponse positive
	\end{itemize}
	\begin{exampleblock}{Exemple}
	Base :
	\begin{itemize}
		\item "{\em Jean} est un des {\em parents} de {\em Tom}"
		\item "{\em Jean} est un {\em homme}"
		\item "{\bf Si} un {\em homme} est le {\em parent} de {\em quelqu'un}, {\bf
		alors} {\em il} est son {\em père}."\\
	\end{itemize}
	Requête : "Jean est-il le père de Tom ?"\\
	\end{exampleblock}
\end{frame}

\begin{frame}
	\frametitle{Chaînage avant}
	\begin{itemize}
		\item Génération de nouveaux faits à partir des précédents et de l'ontologie
		\item A chaque création de fait, vérification de la présence de la requête dans
		$F$
		\item Arrêt lorsque tous les faits sont générés ou si réponse positive
	\end{itemize}

	\begin{exampleblock}{Exemple}
	Base :
	\begin{itemize}
		\item $F = parent(Jean,Tom) \wedge homme(Jean)$
		\item $R = \forall x \forall y (homme(x) \wedge parent(x,y) \rightarrow
		$père$(x,y))$
	\end{itemize}
	Requête : père$(Jean,Tom)$
	\begin{itemize}
		\item Application de l'unique règle
	\end{itemize}
	\end{exampleblock}
\end{frame}


\begin{frame}
	\frametitle{Chaînage avant}
	\begin{itemize}
		\item Génération de nouveaux faits à partir des précédents et de l'ontologie
		\item A chaque création de fait, vérification de la présence de la requête dans
		$F$
		\item Arrêt lorsque tous les faits sont générés ou si réponse positive
	\end{itemize}

	\begin{exampleblock}{Exemple}
	Base :
	\begin{itemize}
		\item $F = parent(Jean,Tom) \wedge homme(Jean)$
		\item $R = \forall x \forall y (homme(x) \wedge parent(x,y) \rightarrow
		$père$(x,y))$
	\end{itemize}
	Requête : père$(Jean,Tom)$
	\begin{itemize}
		\item Application de l'unique règle
		\item $x \leftarrow Jean, y \leftarrow Tom$
	\end{itemize}
	\end{exampleblock}
\end{frame}


\begin{frame}
	\frametitle{Chaînage avant}
	\begin{itemize}
		\item Génération de nouveaux faits à partir des précédents et de l'ontologie
		\item A chaque création de fait, vérification de la présence de la requête dans
		$F$
		\item Arrêt lorsque tous les faits sont générés ou si réponse positive
	\end{itemize}

	\begin{exampleblock}{Exemple}
	Base :
	\begin{itemize}
		\item $F = parent(Jean,Tom) \wedge homme(Jean)$
		\item $R = \forall x \forall y (homme(x) \wedge parent(x,y) \rightarrow
		$père$(x,y))$
	\end{itemize}
	Requête : père$(Jean,Tom)$
	\begin{itemize}
		\item Application de l'unique règle 
		\item $x \leftarrow Jean, y \leftarrow Tom$
		\item $F = parent(Jean,Tom) \wedge homme(Jean) \wedge $père$(Jean,Tom)$
	\end{itemize}
	\end{exampleblock}
\end{frame}


\begin{frame}
	\frametitle{Chaînage avant}
	\begin{itemize}
		\item Génération de nouveaux faits à partir des précédents et de l'ontologie
		\item A chaque création de fait, vérification de la présence de la requête dans
		$F$
		\item Arrêt lorsque tous les faits sont générés ou si réponse positive
	\end{itemize}

	\begin{exampleblock}{Exemple}
	Base :
	\begin{itemize}
		\item $F = parent(Jean,Tom) \wedge homme(Jean)$
		\item $R = \forall x \forall y (homme(x) \wedge parent(x,y) \rightarrow
		$père$(x,y))$
	\end{itemize}
	Requête : père$(Jean,Tom)$
	\begin{itemize}
		\item Application de l'unique règle
		\item $x \leftarrow Jean, y \leftarrow Tom$
		\item $F = parent(Jean,Tom) \wedge homme(Jean) \wedge $père$(Jean,Tom)$
		\item $Q \in F \rightarrow$ réponse positive !
	\end{itemize}
	\end{exampleblock}
\end{frame}

\begin{frame}
	\frametitle{Chaînage arrière}
	\begin{itemize}
		\item Réécriture de la requête via l'ontologie
		\item A chaque réécriture, vérification de la présence d'une de celles-ci dans
		$F$
		\item Arrêt lorsque celle-ci ne peut plus être réécrire ou si réponse positive
	\end{itemize}
	\begin{exampleblock}{Exemple}
	Base :
	\begin{itemize}
		\item "{\em Jean} est un des {\em parents} de {\em Tom}"
		\item "{\em Jean} est un {\em homme}"
		\item "{\bf Si} un {\em homme} est le {\em parent} de {\em quelqu'un}, {\bf
		alors} {\em il} est son {\em père}."\\
	\end{itemize}
	Requête : "Jean est-il le père de Tom ?"\\
	\end{exampleblock}
\end{frame}

\begin{frame}
	\frametitle{Chaînage arrière}
	\begin{itemize}
		\item Réécriture de la requête via l'ontologie
		\item A chaque réécriture, vérification de la présence d'une de celles-ci dans
		$F$
		\item Arrêt lorsque celle-ci ne peut plus être réécrire ou si réponse positive
	\end{itemize}
	\begin{exampleblock}{Exemple}
	Base :
	\begin{itemize}
		\item $F = parent(Jean,Tom) \wedge homme(Jean)$
		\item $R = \forall x \forall y (homme(x) \wedge parent(x,y) \rightarrow
		$père$(x,y))$
	\end{itemize}
	Requête : $Q = $ père$(Jean,Tom)$\\
	\begin{itemize}
		\item Réécriture via l'unique règle
	\end{itemize}
	\end{exampleblock}
\end{frame}


\begin{frame}
	\frametitle{Chaînage arrière}
	\begin{itemize}
		\item Réécriture de la requête via l'ontologie
		\item A chaque réécriture, vérification de la présence d'une de celles-ci dans
		$F$
		\item Arrêt lorsque celle-ci ne peut plus être réécrire ou si réponse positive
	\end{itemize}
	\begin{exampleblock}{Exemple}
	Base :
	\begin{itemize}
		\item $F = parent(Jean,Tom) \wedge homme(Jean)$
		\item $R = \forall x \forall y (homme(x) \wedge parent(x,y) \rightarrow
		$père$(x,y))$
	\end{itemize}
	Requête : $Q = $ père$(Jean,Tom)$\\
	\begin{itemize}
		\item Réécriture via l'unique règle
		\item $Jean \rightarrow x, Tom \rightarrow y$
	\end{itemize}
	\end{exampleblock}
\end{frame}


\begin{frame}
	\frametitle{Chaînage arrière}
	\begin{itemize}
		\item Réécriture de la requête via l'ontologie
		\item A chaque réécriture, vérification de la présence d'une de celles-ci dans
		$F$
		\item Arrêt lorsque celle-ci ne peut plus être réécrire ou si réponse positive
	\end{itemize}
	\begin{exampleblock}{Exemple}
	Base :
	\begin{itemize}
		\item $F = parent(Jean,Tom) \wedge homme(Jean)$
		\item $R = \forall x \forall y (homme(x) \wedge parent(x,y) \rightarrow
		$père$(x,y))$
	\end{itemize}
	Requête : $Q = $ père$(Jean,Tom)$\\
	\begin{itemize}
		\item Réécriture via l'unique règle
		\item $Jean \rightarrow x, Tom \rightarrow y$
		\item $Q' = \{Q_0 = $père$(Jean,Tom)$, $Q_1 = parent(Jean,Tom) \wedge
		homme(Jean) \}$
	\end{itemize}
	\end{exampleblock}
\end{frame}


\begin{frame}
	\frametitle{Chaînage arrière}
	\begin{itemize}
		\item Réécriture de la requête via l'ontologie
		\item A chaque réécriture, vérification de la présence d'une de celles-ci dans
		$F$
		\item Arrêt lorsque celle-ci ne peut plus être réécrire ou si réponse positive
	\end{itemize}
	\begin{exampleblock}{Exemple}
	Base :
	\begin{itemize}
		\item $F = parent(Jean,Tom) \wedge homme(Jean)$
		\item $R = \forall x \forall y (homme(x) \wedge parent(x,y) \rightarrow
		$père$(x,y))$
	\end{itemize}
	Requête : $Q = $ père$(Jean,Tom)$\\
	\begin{itemize}
		\item Réécriture via l'unique règle
		\item $Jean \rightarrow x, Tom \rightarrow y$
		\item $Q' = \{Q_0 = $père$(Jean,Tom)$, $Q_1 = parent(Jean,Tom) \wedge
		homme(Jean) \}$
		\item $Q_1 \in F \rightarrow$ réponse positive !
	\end{itemize}
	\end{exampleblock}
\end{frame}




\section{Dépendance des règles}
% grd.tex

Le graphe de dépendances de règles est une représentation d'une base de règles très
intéressante. En effet il permet de vérifier rapidement quelles règles pourront
éventuellement être déclenchées après l'application d'une règle donnée.

De plus il permet de déterminer l'appartenance à certaines classes de règles, et le
calcul de ses composantes fortement connexes permet de "découper" la base de manière à 
effectuer les requêtes de manières différentes selon celles-ci. 


\section{Définition}\label{grd_def}
Le graphe de dépendances des règles associé à une base de règles $B_R$ est défini comme le 
graphe orienté $GRD = (V_{GRD},E_{GRD})$ avec :
\begin{itemize}
	\item $V_{GRD} = \{R_i \in B_R\}$,
	\item $E_{GRD} = \{(R_i,R_j) : \exists$ un bon unificateur $\mu : \mu(C_i) =
	\mu(H_j)\}$.\\ 
\end{itemize}
\par{}
Intuitivement, on crée un sommet par règle et on relie $R_i$ à $R_j$ si $R_i$ {\em "peut
amener à déclencher"} $R_j$ ($R_j$ dépend de $R_i$).
La notion d'unificateur est abordée dans la section suivante.

\section{Unification de règles}\label{grd_unif}

Afin de pouvoir construire ce graphe, il faut donc pouvoir déterminer si une règle peut
en déclencher une autre, c'est à dire s'il existe un unificateur entre la conclusion de
la première et l'hypothèse de la seconde.
Tout d'abord, l'unification est définie, s'en suit un algorithme permettant de vérifier
si un tel unificateur existe, puis la correction de celui-ci ainsi que ses complexités. 

\subsection{Définitions}\label{grd_unif_def}
% unification.tex

\subsubsection{Substitution}\label{def_unification}
Une substitution de taille $n$ d'un ensemble de symboles $X$ dans un ensemble de symboles $Y$ est une fonction de $X$ 
vers $Y$ repr\'esent\'ee par l'ensemble de couples suivants (avec $n \leq |X|$) :
\begin{itemize}
	\item $s = \{(x_{i},y_{i})\ :\ \forall i \in [1,n]\ x_{i} \in X,\ y_{i} \in Y, \forall j \neq i\ x_{i} \neq x_{j}\}$
	\item $s(x_{i}) = y_{i}\ \forall i \in [1,n]$
	\item $s(x_{i}) = x_{i}\ \forall i \in [n+1,|X|]\ x_{i} \in X$
\end{itemize}

\subsubsection{Unificateur logique}\label{def_unificateur}
Un unificateur logique entre deux atomes $a_{1}$ et $a_{2}$ est une substitution $\mu$ telle que :
\begin{itemize}
	\item $\mu : var(a_{1}) \cup var(a_{2}) \rightarrow dom(a_{1}) \cup dom(a_{2})$
	\item $\mu(a_{1}) = \mu(a_{2})$
\end{itemize}
Cette d\'efinition s'\'etend aux conjonctions d'atomes.

\subsubsection{Unificateur de conclusion atomique}\label{def_unificateuratomique}
Un unificateur de conclusion atomique est un unificateur logique $\mu = \{(x_{i},t_{i}) : \forall i \in [1,n]\}$
entre l'hypoth\`ese d'une r\`egle $R_{1} = (H_{1},C_{1})$ et la conclusion atomique d'une r\`egle $R_{2} = (H_{2},C_{2})$ et est d\'efini de la mani\`ere suivante :
\begin{itemize}
	\item $\mu : fr(R_{2}) \cup var(H_{1}) \rightarrow dom(C_{2}) \cup cst(H_{1})$
	\item $\forall (x_{i},t_{i}) \in \mu\ si\ x_{i} \in fr(R_{2})\ alors\ t_{i} \in cutp(R_{2}) \cup cst(H_{1})$
\end{itemize}
    
\subsubsection{Bonne unification atomique}\label{def_bonneunification}
Un bon unificateur de conclusion atomique est un unificateur de conclusion atomique $\mu = \{x_{i},t_{i}) : \forall i \in [1,n]\}$
entre un sous ensemble $Q$ de l'hypoth\`ese d'une r\`egle $R_{1} = (H_{1},C_{1})$ et la conclusion d'une r\`egle atomique
$R_{2} = (H_{2},C_{2})$ tel que :\\
$\forall (x_{i},t_{i}) \in \mu : si\ x_{i} \in H_{1} \setminus Q\ alors\ t_{i}\ n'est\ pas\ une\ \exists-var$

Un tel ensemble $Q$ est appel\'e un {\em bon ensemble d'unification atomique} de l'hypoth\`ese de $R_{1}$ par la conclusion de $R_{2}$.
On note que $Q$ est donc d\'efini comme suit :
\begin{itemize}
	\item $Q \subseteq H_{1}$
	\item $\forall\ position\ i\ de\ \exists-var\ dans\ C_{2}, \forall\ atome\ a\ \in Q, si\ a_i \in var(H_{1})\ alors\ \forall\ atome\ b \in H_{1} : si\ \exists\ b_j \in b : a_i = b_j, alors\ b \in Q$
\end{itemize}

\subsubsection{Bon ensemble d'unification atomique minimal}\label{def_bonensemble}
Un {\em bon ensemble d'unification atomique minimal} $Q$ de $H_{1}$ par $C_{2}$ enracin\'e en $a$ est d\'efini tel que :
\begin{itemize}
	\item $Q$ est un bon ensemble d'unification atomique de $H_{1}$ par $C_{2}$
	\item $a \in Q$
	\item $|Q| = min(|Q_i| : Q_i\ est\ un\ bon\ ensemble\ d'unification\ atomique\ de\ H_1\ par\ C_2\ et\ a \in Q_i)$
\end{itemize}




\subsection{Algorithmes}\label{grd_algo}
	Vérifier qu'une règle atomique $R_i$ peut déclencher $R_j$ consiste donc à trouver un bon
	unificateur atomique entre $R_i$ et $R_j$. Dans cette section, un algorithme
	permettant de répondre à ce problème est détaillé.

	Dans la suite, les règles sont supposées représentées par des graphes (tels que
	définis en \ref{def_representation_regle}) et à conclusion atomique.

	% TODO changer les couleurLogique en estLocalementUnifiable, et les couleur en
	% estTraite
	Le premier algorithme fait appel aux deux suivants de manière à déterminer si il
	existe au moins un unificateur entre les deux conjonctions d'atomes.
	En première phase, il vérifie l'exitence d'unificateurs avec chaque atome de manière
	indépendante. S'ensuit une extension à partir des atomes préselectionnés, et dès
	qu'un bon ensemble d'unification est entièrement unifié, l'algorithme s'arrête en
	répondant avec succès.
	\begin{center}
\begin{algorithm}[H]
\caption{Unification}\label{algo_unification}
\algsetup{indent=2em,linenodelimiter= }
\begin{algorithmic}[1]
\REQUIRE $H_{1}$ : conjonction d'atomes, $R = (H_{2},C_{2})$ : r\`egle \`a conclusion atomique 
\ENSURE succ\`es si $C_{2}$ peut s'unifier avec $H_{1}$, i.e. si $\exists H \subseteq H_{1}, \mu\ une\ substitution : \mu(H_{1}) = \mu(C_{2})$, \'echec sinon 
\STATE $\triangleright$ Pr\'ecoloration
\FORALL {$sommet\ atome\ a \in H_{1}$}
	\IF {$UnificationLocale(a,R) \neq$ \'echec}
		\STATE $localementUnifiable[a] \leftarrow vrai$
	\ELSE
		\STATE $localementUnifiable[a] \leftarrow faux$
	\ENDIF
\ENDFOR 

\STATE $\triangleright$ Initialisation du tableau contenant les positions des variables existentielles de $C_{2}$
\STATE $E \leftarrow \{i : c_{i}\ est\ une\ \exists-var\ de\ C_{2}\}$

\STATE $\triangleright$ Extension des ensembles
\FORALL {$sommet\ atome\ a \in H_{1} : localementUnifiable[a] = noir$}
	\IF {$Q \leftarrow Extension(H_{1},a,localementUnifiable,E) \neq $ \'echec} 
		\IF {$UnificationLocale(Q,R) \neq$ \'echec}
			\RETURN succ\`es
		\ENDIF
	\ENDIF
	\STATE $localementUnifiable[a] \leftarrow faux$
\ENDFOR
\RETURN \'echec
\end{algorithmic}
\end{algorithm}
\end{center}



	Le deuxième algorithme est utilisé pour le calcul des bons ensembles d'unification à
	partir d'un atome racine. Tant qu'aucune erreur n'est détectée il {\em avale} les atomes
	voisins aux termes en positions existentielles.
	Les positions existentielles sont les indices des variables existentielles dans
	l'atome de conclusion.
	\begin{center}
\begin{algorithm}[H]
\caption{Extension}\label{algo_extension}
\algsetup{indent=2em,linenodelimiter= }
\begin{algorithmic}[1]
\REQUIRE	$H_{1}$ : conjonction d'atomes, 
			$a \in H_{1}$ : sommet atome racine, 
			$localementUnifiable$ : tableau de taille \'egal au nombre d'atomes dans
			$H_{1}$ tel que $localementUnifiable[a] = vrai$ ssi $UnificationLocale(a,R) =$ succ\`es, 
			$E$ : ensemble des positions des variables existentielles
\ENSURE $Q$ : bon ensemble d'unification minimal des atomes de ${H_1}$ construit \`a partir de $a$ s'il existe, \'echec sinon.
\STATE $\triangleright$ Initialisation du parcours
%\FORALL {$sommet\ v \in H_{1}$}
%	\STATE $couleur[v] \leftarrow blanc$   $\triangleright$ coloration pour le parcours
%\ENDFOR
\FORALL {$sommet\ atome\ a \in H_{1}$}
	\IF {$localementUnifiable[a] = vrai$}
		\FORALL {$sommet\ terme\ t\ \in voisins(a)$}
			\STATE $couleur[t] = blanc$
		\ENDFOR
		\STATE $couleur[a] = blanc$
	\ENDIF
\ENDFOR
\STATE $couleur[a] \leftarrow noir$
\STATE $Q \leftarrow \{a\}$   $\triangleright$ conjonction d'atomes \`a traiter
\STATE $attente \leftarrow \{a\}$   $\triangleright$ file d'attente du parcours
\WHILE {$attente \neq \emptyset$}
	\STATE $u \leftarrow haut(attente)$
	\IF {$u\ est\ un\ atome$}
		\FORALL {$i \in E$}
			\STATE $v \leftarrow voisin(u,i)$
			\IF {$v\ est\ une\ constante$}
				\RETURN \'echec	
			\ELSIF {$couleur[v] = blanc$}
				\STATE $\triangleright$ v est une $\forall-var$ non marqu\'ee par le parcours
				\STATE $couleur[v] \leftarrow noir$
				\STATE $attente \leftarrow attente \cup \{v\}$
			\ENDIF
		\ENDFOR
	\ELSE
		\STATE $\triangleright$ u est un terme
		\FORALL {$v \in voisins(u)$}
			\IF {$localementUnifiable[v] = faux$}
				\RETURN \'echec
			\ELSE
				\IF {$couleur[v] = blanc$}
					\STATE $couleur[v] \leftarrow noir$
					\STATE $attente \leftarrow attente \cup \{v\}$
					\STATE $Q \leftarrow Q \cup \{v\}$
				\ENDIF
			\ENDIF
		\ENDFOR
	\ENDIF
\ENDWHILE $\ \ \ \triangleright$ Fin du parcours
\RETURN $Q$
\end{algorithmic}
\end{algorithm}
\end{center}



	Remarque : \\
	La phase d'initialisation du parcours pourrait simplement parcourir tous les sommets
	de $H_{1}$ et mettre leur couleur \`a blanc.
	En pratique, cette solution serait sans doute plus efficace, mais d\'ependrait donc
	du nombre de sommets total dans $H_{1}$.
	Ce qui en th\'eorie am\`enerait la complexit\'e de cette boucle en 
	$\bigcirc(nombre\ d'atomes\ \times arite\ max\ de\ H_{1})$.
	Or ici, la complexit\'e ne d\'epend pas de cette arit\'e max, mais uniquement de
	l'arit\'e du pr\'edicat de la conclusion $C_{2}$.

	Le dernier algorithme est celui qui teste réellement si il existe un unificateur
	entre une conclusion atomique, et un bon ensemble d'unification atomique minimal.
	Il est appelé une première fois pour tester les atomes de la conjonction séparement,
	et permettre une préselection des atomes (qui vont servir de racine).
	Durant la dernière phase (lorsqu'il est appelé sur les ensembles étendus), s'il trouve
	un unificateur, celui-ci assure que la règle peut amener à déclencher la conjonction.
	\begin{center}
\begin{algorithm}[H]
\caption{UnificationLocale}\label{unification_locale}
\algsetup{indent=2em,linenodelimiter= }
\begin{algorithmic}[1]
\REQUIRE $H_{1}$ : conjonction d'atomes, $R = (H_{2},C_{2})$ : r\`egle \`a conclusion atomique 
\ENSURE succ\`es si $C_{2}$ peut s'unifier avec $H_{1}$ 
\STATE $\triangleright$ V\'erification des pr\'edicats
\FORALL {$atome\ a \in H_{1}$}
	\IF {pr\'edicat($a$) $\neq$ pr\'edicat($C_{2}$)}
		\RETURN \'echec
	\ENDIF
\ENDFOR
\STATE $u \leftarrow \emptyset\ \ \ \triangleright$ substitution 
\FORALL {$terme\ t_{i} \in C_{2}$}
	\STATE $\triangleright$ def : $a_{i}$ = terme de a en position i
	\STATE $E \leftarrow \{a_{i} : \forall\ atome\ a \in H_{1}\}$
	\IF {$t_{i}\ est\ une\ constante$}
		\IF {$\exists\ v \in E : v\ est\ une\ constante\ et\ v \neq t_{i},\ ou\ v\ est\ une\ \exists-variable$}
			\RETURN \'echec
		\ELSE
			\STATE $u \leftarrow \{(v,t_{i}) : v \in E\ et\ v \neq t_{i}\}$
			%\STATE $u \leftarrow u \circ \{(v,t_{i}) : v \in E\ et\ v \neq t_{i}\}$
		\ENDIF
	\ELSIF {$t_{i}\ est\ une\ \exists-variable$}
		\IF {$\exists\ v \in E : v\ est\ une\ \exists-variable\ et\ v \neq t_{i},\ ou\ v\ est\ une\ constante$}
			\RETURN \'echec
		\ELSE
			\STATE $u \leftarrow \{(v,t_{i}) : v \in E\ et\ v \neq t_{i}\}$
			%\STATE $u \leftarrow u \circ \{(v,t_{i}) : v \in E\ et\ v \neq t_{i}\}$
		\ENDIF
	\ELSE
		\IF {$\exists\ v_{1},v_{2} \in E : v_{1} \neq v_{2}\ et\ v_{1},v_{2}\ ne\ sont\ pas\ des\ \forall-variables$}
			\RETURN \'echec
		\ELSIF {$\exists\ c \in E : c\ est\ une\ constante$}
			\STATE $u \leftarrow \{(v,c) : v \in E \cup \{t_{i}\}\ et\ v \neq c\}$
			%\STATE $u \leftarrow u \circ \{(v,c) : v \in E\ et\ v \neq c\} \cup \{(t_{i},c)\}$
		\ELSE
			\STATE $u \leftarrow \{(v,t_{i}) : v \in E\ et\ v \neq t_{i}\}$
			%\STATE $u \leftarrow u \circ \{(v,t_{i}) : v \in E\ et\ v \neq t_{i}\}$
		\ENDIF
	\ENDIF	
	\STATE $H_{1} \leftarrow u(H_{1})$
	\STATE $C_{2} \leftarrow u(C_{2})$
\ENDFOR
\RETURN succ\`es 

\end{algorithmic}
\end{algorithm}
\end{center}





\subsection{Correction}\label{grd_correction}
%\subsubsection{Propri\'et\'es}\label{unification_correction_proprietes}
%	{\em Propri\'et\'e 1} \\
%	Soit $R = (H,C)$ une r\`egle \`a conclusion atomique, \\
%	Soit $K = (k_1, k_2, ..., k_n)$ une conjonction d'atomes,\\
%	si $\exists a \in K$ : a ne peut pas s'unifier avec C alors $K$ ne peut pas s'unifier avec $C$

\subsection{Complexites}\label{grd_complexites}
	
Soit $R$ une règle, et $G_R$ son graphe associé.
On note :
\begin{itemize}
	\item $k$ : nombre d'atomes dans $R$
	\item $p$ : arit\'e maximum des prédicats de $R$
	\item $t \leq n \times p$ : nombre de termes de $R$
\end{itemize}
et :
\begin{itemize}
	\item $n = k + t$ : nombre de sommets dans $G_R$
	\item $m \leq n \times p$ : nombre d'arêtes dans $G_R$
\end{itemize}

Avec notre représentation nous avons donc les complexités suivantes :
\begin{itemize}
	\item Parcourir les sommets prédicats : $\bigcirc(k)$.
	\item Parcourir tous les sommets : $\bigcirc(n)$.
	\item Acc\'eder au $i^{eme}$ voisin d'un sommet : $\bigcirc(1)$.
\end{itemize}



\begin{itemize}
	\item $n$ = nombre d'atomes dans $H_{1}$
	\item $p$ = arit\'e de $C_{2}$
	\item $t$ = nombre de termes "colorables" dans $H_{1}$
	\item $m$ = nombre d'ar\^etes "suivables" dans $H_{1}$
\end{itemize}

On remarque que dans le pire des cas on a :
\begin{itemize}
	\item $t = n \times p$
	\item $m = n \times p$
\end{itemize}

\subsubsection{Temps}
	Ainsi, les complexités en temps d'exécution des algorithmes nécessaires à l'unification 
sont les suivantes :
\begin{itemize}
	\item UnificationLocale (pire des cas) $= \bigcirc(k \times p)$
	\item Extension (pire des cas) $= \bigcirc(k \times p)$
	\item Unification (pire des cas) $= \bigcirc(k^{2} \times p)$
\end{itemize}

%%%%	Extension\\
%%%%	On effectue simplement un parcours en largeur \`a partir d'un sommet donn\'e qui peut \'eventuellement s'arr\^eter plus
%%%%	t\^ot qu'un parcours classique. La complexit\'e en temps dans le pire des cas est donc au plus la m\^eme, c'est \`a dire lin\'eaire
%%%%	au nombre de sommets plus le nombre d'arcs (voir \cite{algorithmique} page 517-525).
%%%%	Le graphe repr\'esentant la conjonction d'atomes $H_{1}$ poss\`edent $n \times ariteMax(H_{1})$ ar\^etes
%%%%	et $n \times (ariteMax(H_{1}) + 1)$ sommets.
%%%%	On sait donc que $C_{Extension}^{temps} = \bigcirc(n \times ariteMax(H_1))$.


%%%%	Deux cas :\\
%%%%		i) D\'ecouverte d'un sommet atome : \\
%%%%			parcours classique => on r\'ecup\`ere tous les voisins blancs \\
%%%%			parcours extension => on r\'ecup\'ere tous les voisins blancs de couleur logique noire (en effet arr\^et imm\'ediat si un voisin de couleur logique noire est d\'ecouvert.\\
%%%%		ii) D\'ecouverte d'un sommet variable :\\
%%%%			pas de diff\'erence\\
%%%%	On a donc que l'algo ne suit que les sommets atomes pouvant \^etre unifies localement. C'est \`a dire (entre autres) que leur pr\'edicat est \'egal \`a pr\'edicat($C_{2}$).



%Le parcours suit au pire $2(n \times p)+1$ ar\^etes.
%Supposons que l'algorithme ait d\'ej\`a parcouru $2(n \times p)$ ar\^etes %et qu'il d\'ecouvre un sommet toujours blanc.
%...


%On a donc $C_{Extension}^{temps} = \bigcirc$




\subsubsection{Espace}
	% complexites_espace.tex

\begin{itemize}
	\item $|localementUnifiable|$ : $k$
	\item $|couleur|$ : $n$
	\item 
\end{itemize}





\section{Composantes fortement connexes}\label{grd_scc}

Les composantes fortement connexes du graphe de dépendances sont utilisées pour le 
{\em découper} de façon à attribuer des étiquettes différentes à celles-ci en fonction
des classes concrètes auxquelles son ensemble de règles appartient.

De plus une fois chaque sous ensemble étiqueté, il faut encore vérifier que celles-ci
sont compatibles entre elles.

Pour cela, nous définissons le graphe orienté des composantes fortement connexes associé
tel que son ensemble de sommets est l'ensemble des composantes du graphe, et
qu'il existe un arc entre deux composantes $C_i$ et $C_j$ si et seulement s'il existe un
arc d'un sommet de $C_i$ vers un sommet de $C_j$.
Par définition des composantes fortement connexes, ce graphe est évidemment sans
circuit, et les arcs de celui-ci influent directement sur la décidabilité de l'ensemble
de règles.

On dit que $C_i$ {\em précède} $C_j$ s'il n'existe aucun arc de $C_j$ vers $C_i$, et on
note cette relation $C_i \triangleright C_j$.

De plus,
on associe à chaque $C_i$ une étiquette qui déterminera la classe abstraite considérée pour
cette composante.
Une condition (suffisante)%TODO
pour que l'ensemble de règles soit décidable est la suivante :\\
$\{C_i : etiquette(C_i) = FES\} \triangleright \{C_i : etiquette(C_i) = GBTS\} \triangleright
\{C_i : etiquette(C_i) = FUS\}$
C'est à dire qu'aucune règle \fes ne doit dépendre d'une règle \fus ou \gbts, et
qu'aucune règle \gbts ne doit dépendre d'une règle \fus.

En effet les algorithmes de chaînage arrière par exemple réécrivent la requête jusqu'à ce
qu'elle corresponde à la base, tandis que ceux avant ajoutent des faits jusqu'à obtenir
la requête. Il est donc évident que si une composante n'accepte que le chaînage arrière,
il ne doit exister aucune règle de celle-ci de laquelle dépende une règle de la
composante acceptant uniquement le chaînage avant (si tel était le cas, cette règle ne
serait jamais déclenchée).





\section{Classes de règles}
% classes_regles.tex



\subsection{Classes abstraites}

\begin{frame}[t]
	\frametitle{Classes de règles abstraites}
	\vspace{10mm}
	\begin{itemize}
		\item Aucune propriété vérifiable
		\item Assurent la décidabilité du problème en suivant certains algorithmes
	\end{itemize}
\end{frame}

\begin{frame}[t]
	\frametitle{Classes de règles abstraites}
	\vspace{10mm}
	\begin{itemize}
		\item Aucune propriété vérifiable
		\item Assurent la décidabilité du problème en suivant certains algorithmes
		\item Finite Extension Set : algorithmes de chaînage avant
	\end{itemize}
\end{frame}

\begin{frame}[t]
	\frametitle{Classes de règles abstraites}
	\vspace{10mm}
	\begin{itemize}
		\item Aucune propriété vérifiable
		\item Assurent la décidabilité du problème en suivant certains algorithmes
		\item Finite Extension Set : algorithmes de chaînage avant
		\item (Greedy) Bounded Treewidth Set : algorithmes de chaînage avant avec condition
		d'arrêt particulière
	\end{itemize}
\end{frame}

\begin{frame}[t]
	\frametitle{Classes de règles abstraites}
	\vspace{10mm}
	\begin{itemize}
		\item Aucune propriété vérifiable
		\item Assurent la décidabilité du problème en suivant certains algorithmes
		\item Finite Extension Set
		\item (Greedy) Bounded Treewidth Set : algorithmes de chaînage avant avec condition
			d'arrêt particulière
		\item Finite Unification Set : algorithmes de chaînage arrière
	\end{itemize}
\end{frame}

\begin{frame}[t]
	\frametitle{Classes de règles abstraites}
	\vspace{10mm}
	\begin{itemize}
		\item Aucune propriété vérifiable
		\item Assurent la décidabilité du problème en suivant certains algorithmes
		\item Finite Extension Set
		\item (Greedy) Bounded Treewidth Set : algorithmes de chaînage avant avec condition
			d'arrêt particulière
		\item Finite Unification Set : algorithmes de chaînage arrière
		\item Classes incomparables
	\end{itemize}
\end{frame}


\subsection{Classes concrètes}

\begin{frame}
	\frametitle{Classes de règles concrètes}
	\vspace{10mm}
	\begin{itemize}
		\item Imposent des contraintes sur la forme des règles ou de la base
		\item Spécialisent les classes abstraites
		\item Classes pouvant être comparables
	\end{itemize}
\end{frame}

\begin{frame}
	\frametitle{Guardée}
	\begin{itemize}
		\item Un atome de l'hypothèse contient toutes les variables de celle-ci
		\item $\exists a \in H_i : variable(H_i) \subseteq variables(a)$
		\item Simple à vérifier
		\item $guarded(R) = \{\forall R_i \in R : R_i $ est gardée$\} \in GBTS$
	\end{itemize}
	\vspace{10mm}
	\begin{exampleblock}{Exemple}
		\ruleC \\
		Garde : parent($x$,$y$)
	\end{exampleblock}
\end{frame}

\begin{frame}
	\frametitle{Frontière gardée}
	\begin{itemize}
		\item Un atome de l'hypothèse contient toutes les variables de la frontière
		\item $\exists a \in H :$ frontière($R$) $\subseteq variables(a)$
		\item Seule la frontière influe sur l'application d'une règle
		\item Généralisation des règles gardées
		\item $fr-guarded(R) = \{\forall R_i \in R : R_i $ a une frontière gardée$\} \in GBTS$
	\end{itemize}
	\begin{exampleblock}{Exemple}
		$\forall x \forall y (p(x) \wedge q(y) \rightarrow \exists z (r(y,z)))$\\
		Garde-frontière : $q(y)$
	\end{exampleblock}
\end{frame}

\begin{frame}
	\frametitle{Frontière de taille 1}
	\begin{itemize}
		\item La frontière de la règle est de taille 1
		\item Spécialisation des règles à frontière gardée
		\item Utiles pour les notions d'héritage
		\item $fr-1(R) = \{\forall R_i \in R : |$frontière$(R_i)| = 1 \} \in GBTS$
	\end{itemize}
	\begin{exampleblock}{Exemple}
		\ruleB \\
		frontière$(R_2) = \{x\}$
	\end{exampleblock}
\end{frame}

\begin{frame}
	\frametitle{Hypothèse atomique}
	\begin{itemize}
		\item L'hypothèse de la règle ne contient qu'un seul atome
		\item Spécialisation des règles gardées
		\item $ah(R) = \{\forall R_i = (H_i,C_i) \in R : |H_i| = 1 \} \in GBTS \cap FUS$
	\end{itemize}
	\begin{exampleblock}{Exemple}
		\ruleA
	\end{exampleblock}
\end{frame}

\begin{frame}
	\frametitle{Domaine restreint}
	\begin{itemize}
		\item Les atomes de la conclusion contiennent soit toutes les variables de
		l'hypothèse, soit aucune
		\item $\forall a_j \in R_i (variables(H_i) \subseteq variables(a_j)) \vee 
		(variables(H_i) \cap variables(a_j) = \emptyset)$
		\item Incomparable avec les autres classes concrètes exhibées
		\item $dr(R) = \{\forall R_i \in R : R_i$ a un domaine restreint $\} \in FUS$
	\end{itemize}
	\begin{exampleblock}{Exemple}
		\ruleB \\
		$variables(homme(z)) \cap variables(H_2) = \emptyset$\\
		$variables(H_2) \subseteq variables(pere(z,x))$
	\end{exampleblock}
\end{frame}

\begin{frame}
	\frametitle{Déconnectée}
	\begin{itemize}
		\item Frontière vide
		\item Spécialisation des règles de domaine restreint
		\item Une seule application nécessaire
		\item $disc(R) = \{\forall R_i \in R : $frontière$(R_i) = \emptyset \} \in FUS$
	\end{itemize}
	\begin{exampleblock}{Exemple}
		$\forall x (p(x) \wedge q(a) \rightarrow \exists z (r(a,z)))$
	\end{exampleblock}
\end{frame}

\begin{frame}
	\frametitle{Universelle}
	\begin{itemize}
		\item Aucune variable existentielle
		\item Couramment utilisées
		\item $rr(R) = \{\forall R_i \in R : variables(R_i) \subseteq variables(H_i) \}
		\in FES \cap GBTS$
	\end{itemize}
	\begin{exampleblock}{Exemple}
		\ruleE
	\end{exampleblock}
\end{frame}

\begin{frame}
	\frametitle{Faiblement acyclique}
	\begin{itemize}
		\item Contrainte sur l'ensemble des règles
		\item Nécessite l'usage d'une nouvelle structure : le graphe de dépendances des
		positions
		\item $wa(R) \in FES$
	\end{itemize}
	\begin{exampleblock}{Exemple de non faible acyclicité}
		\ruleA \\
		\ruleB \\
	\end{exampleblock}
	\begin{exampleblock}{Exemple de faible acyclicité}
		\ruleC \\
		\ruleD \\
	\end{exampleblock}
\end{frame}

\begin{frame}
	\frametitle{Sticky}
	\begin{itemize}
		\item Contrainte sur l'ensemble des règles
		\item Marquage des variables
		\item Une variable marquée ne doit pas apparaître plusieurs fois dans l'hypothèse
		d'une règle
		\item $sticky(R) \in FES$
	\end{itemize}
	\begin{exampleblock}{Exemple}
		\ruleC \\
		\ruleD \\
	\end{exampleblock}
\end{frame}

\begin{frame}
	\frametitle{Weakly sticky}
	\begin{itemize}
		\item Généralisation de faiblement acyclique et de sticky
		\item Les variables marquées ne doivent pas apparaître plusieurs fois dans
		l'hypothèse ou ne pas être dans une position de rang infini
		\item $ws(R) \notin FES \cup GBTS \cup FUS$
	\end{itemize}
\end{frame}

\begin{frame}
	\frametitle{Graphe de dépendances des règles acyclique}
	\begin{itemize}
		\item L'ensemble des règles forment un graphe de dépendances sans circuit
		\item $aGRD(R) \in FES \cup FUS$
	\end{itemize}
	\begin{exampleblock}{Exemple de faible acyclicité}
		\ruleB \\
		\ruleD
	\end{exampleblock}
\end{frame}


\begin{frame}
	\frametitle{Schéma récapitulatif}
\begin{figure}[H]
\begin{center}
\begin{tikzpicture}[
	auto,
	node distance=1.5cm,
	inner sep=1mm,
	scale=0.95
]
	\draw [blue] (-3.8,-5.5) -- (3.2,-5.5) -- (3.2,-2.5) -- (-3.8,-2.5) -- (-3.8,-5.5);
	\draw [red] (0.3,1) -- (3.7,1) -- (3.7,-5.7) -- (0.3,-5.7) -- (0.3,1);
	\draw [green] (4,1) -- (9,1) -- (9,-4.5) -- (-2.2,-4.5) -- (-2.2,-3.5) --
	(0.5,-3.5) -- (0.5,-2) -- (4,-2) -- (4,1);

	\node [blue] (fes) at (-3,-5) {FES};
	\node [red] (fus) at (1,0.5) {FUS};
	\node [green] (gbts) at (8,0.5) {GBTS};

	% terms
	\node [class]	(ws)		at (-2,0)	{weakly sticky};
	\node [class]	(dr)		at (2,-1)	{domaine restreint};
	\node [class]	(agrd)		at (2,-5)	{grd acyclique};
	\node [class]	(fg)		at (6,0)	{frontière guardée};
	\node [class]	(sticky)	at (2,0)	{sticky}
		edge[classedge]	(ws);
	\node [class]	(wa)		at (-2,-3)	{faiblement acyclique}
		edge[classedge]	(ws);
	\node [class]	(disc)		at (2,-3)	{déconnectée}
		edge[classedge] (wa)
		edge[classedge] (dr)
		edge[classedge] (fg);
	\node [class]	(guarded)	at (6,-2)	{guardée}
		edge[classedge]	(fg);
	\node [class]	(fr1)		at (8,-2)	{frontière 1}
		edge[classedge] (fg);
	\node [class]	(ah)		at (6,-4)	{hypothèse atomique}
		edge[classedge] (guarded);
	\node [class]	(rr)		at (-1,-4)	{universelles}
		edge[classedge] (wa);
\end{tikzpicture}
\end{center}
\end{figure}
\end{frame}



\section{Développement}
% dev.tex

\begin{frame}
	\frametitle{Structures de données}
	\begin{itemize}
		\item Différents types de graphes
		\item Structure générique et algorithmes
		\item Sommets et arcs de types paramétrables
	\end{itemize}
\end{frame}

\begin{frame}
	\frametitle{Structures de données}
	\begin{itemize}
		\item Règles : graphe biparti non orienté (prédicats et termes, position du terme)
		\item Dépendances des règles : graphe orienté (règles, aucun)
		\item Composantes fortement connexes : graphe orienté (ensembles de sommets,
		aucun) sans circuit
		\item Dépendances des positions : graphe orienté (positions des prédicats,
		spécial ou non)
	\end{itemize}
\end{frame}

\begin{frame}
	\frametitle{Analyseur}
	\begin{itemize}
		\item Divisé en deux parties
		\item Détermination des classes concrètes
		\item Combinaison des classes abstraites
		\item Règles à conclusion atomique (sans perte de généralité)
	\end{itemize}
\end{frame}

\begin{frame}
	\frametitle{Détermination des classes concrètes}
	\begin{itemize}
		\item Possibilité d'ajouter de nouveaux tests
		\item Calcul sur l'ensemble des règles, puis sur chaque composante connexe
		\item Renvoient des étiquettes
		\item Indiquent les classes abstraites satisfaites
	\end{itemize}
\end{frame}

\begin{frame}
	\frametitle{Combinaison des classes abstraites}
	\begin{itemize}
		\item Vérification de l'ensemble des règles
		\item Etiquettes des classes abstraites valuées
		\item Parcours du graphe des composantes à partir des sources
		\item Attribue à chaque composante l'étiquette la plus {\em petite} compatible
		\item Décidable si tous les sommets sont étiquetés
	%	\item Liste des types d'algorithmes à utiliser
	\end{itemize}
\end{frame}

\begin{frame}
	\frametitle{Fichiers}
	\begin{itemize}
		\item Lecture et écriture via un format interne
		\item Conversion des fichiers Datalog (.dtg)
		\item Sortie PostScript pour l'affichage des graphes
	\end{itemize}
\end{frame}



\section{Conclusion}
% conclusion.tex

\begin{frame}
	\frametitle{Gestion du projet}
	\begin{itemize}
		\item Gestionnaire de versions : git
		\item Réunions fréquentes avec les encadrants
		\item Problèmes :
		\begin{itemize}
			\item Domaine nouveau
			\item Disparition d'un membre du groupe
		\end{itemize}
	\end{itemize}
\end{frame}

\begin{frame}
	\frametitle{Contributions}
	\begin{itemize}
		\item Lecture et écriture d'une base à partir et vers un fichier
		\item Mise en place d'un algorithme d'unification
		\item Construction du graphe de dépendances des règles
		\item Calcul des classes concrètes
		\item Combinaison des classes abstraites
		\item Etude de la décidabilité
	\end{itemize}
\end{frame}

\begin{frame}
	\frametitle{Perspectives}
	\begin{itemize}
		\item Combinaison des classes de règles en fonction des complexités
		\item Interface graphique
	\end{itemize}
\end{frame}



\bibliographystyle{plain}
\bibliography{src/bib}

\end{document}

