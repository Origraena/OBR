% combine.tex

\begin{frame}
	\frametitle{Réponse à une requête en temps fini}
	\begin{itemize}
		\item Graphe de dépendances de règles sans circuit
		\item L'ensemble des règles étiqueté par une même classe 
		\item Sinon : 
		calcul du graphe orienté des composantes fortement connexes
		\item Détermination des classes de chaque composante
		\item Vérification de la propriété de {\em précédence}
	\end{itemize}
\end{frame}

\begin{frame}
	\frametitle{Précédence}
	\begin{itemize}
		\item Graphe des composantes fortement connexes : un sommet par composante, un
		arc entre $C_i$ et $C_j$ si une règle de $C_i$ peut déclencher une règle de $C_j$
		\item Composante $C_i$ {\em précède} $C_j$ si aucun arc de $C_j$ vers $C_i$
		\item Notée $C_i \triangleright C_j$
		\item Décidable si $FES \triangleright GBTS \triangleright FUS$
	\end{itemize}
\end{frame}

\begin{frame}
	\frametitle{Exemple}
	\begin{exampleblock}{Exemple}
	\begin{itemize}
		\item $R_1$ : $\forall x$ (homme($x$) $\rightarrow$ humain($x$))
		\item $R_2$ : $\forall x$ (humain($x$) $\rightarrow$ $\exists z$ (homme($z$)
		$\wedge$ pere($z$,$x$)))
		\item $R_3$ : $\forall x \forall y$ (parent($x$,$y$) $\wedge$ homme($x$)
		$\rightarrow$ pere($x$,$y$))
		\item $R_4$ : $\forall x \forall y$ (pere($x$,$y$) $\rightarrow$
		parent($x$,$y$))
	\end{itemize}
	\end{exampleblock}
	\begin{figure}
		\begin{tikzpicture}[
			auto,
			inner sep=1mm
		]
		\node[rule] (R1) at (0,0) {$R_1$};
		\node[rule] (R2) at (3,0) {$R_2$} 
			edge[ruleedge] (R1)
			edge[ruleedger] (R1)
		;
		\node[rule] (R3) at (0,3) {$R_3$};
		\node[rule] (R4) at (3,3) {$R_4$}
			edge[ruleedger,bend left=0] (R2)
			edge[ruleedge] (R3)
			edge[ruleedger] (R3)
		;
		\end{tikzpicture}
	\end{figure}
\end{frame}

\begin{frame}
	\frametitle{Exemple}
	\begin{columns}
	\begin{column}{0.45\linewidth}
		\begin{figure}
			\begin{tikzpicture}[
				auto,
				inner sep=1mm
			]
			\node[rule] (R1) at (0,0) {$R_1$};
			\node[rule] (R2) at (3,0) {$R_2$} 
				edge[ruleedge] (R1)
				edge[ruleedger] (R1)
			;
			\node[rule] (R3) at (0,3) {$R_3$};
			\node[rule] (R4) at (3,3) {$R_4$}
				edge[ruleedger,bend left=0] (R2)
				edge[ruleedge] (R3)
				edge[ruleedger] (R3)
			;
			\end{tikzpicture}
			\caption{GRD}
		\end{figure}
	\end{column}
	\vline
	\hfill
	\begin{column}{0.45\linewidth}
		\begin{figure}
			\begin{tikzpicture}[
				auto,
				inner sep=1mm
			]
			\node[comp] (C1) at (0,-1.5) {$C_1:R_1,R_2$};
			\node[comp] (C2) at (0,1.5) {$C_2:R_3,R_4$} 
				edge[ruleedger,bend left=0] (C1)
			;
			\end{tikzpicture}
		\caption{Composantes fortement connexes}
		\end{figure}
	\end{column}
	\end{columns}
\end{frame}

\begin{frame}
	\frametitle{Exemple}
	\begin{columns}
	\begin{column}{0.45\linewidth}
		\begin{itemize}
			\item $C_1 \triangleright C_2$
			\item Si $C_1$ est $FUS$ uniquement, $C_2$ doit l'être également
			\item Si $C_1$ est $GBTS$ uniquement, $C_2$ ne peut pas être $FES$
		\end{itemize}
	\end{column}
	\vline
	\hfill
	\begin{column}{0.45\linewidth}
		\begin{figure}
			\begin{tikzpicture}[
				auto,
				inner sep=1mm
			]
			\node[comp] (C1) at (0,-1.5) {$C_1:R_1,R_2$};
			\node[comp] (C2) at (0,1.5) {$C_2:R_3,R_4$} 
				edge[ruleedger,bend left=0] (C1)
			;
			\end{tikzpicture}
		\caption{Composantes fortement connexes}
		\end{figure}
	\end{column}
	\end{columns}
\end{frame}

\begin{frame}
	\frametitle{Exemple}
	\begin{columns}
	\begin{column}{0.45\linewidth}
		$C_1$ :
		\begin{itemize}
			\item $R_1$ : $\forall x$ (homme($x$) $\rightarrow$ humain($x$))
			\item $R_2$ : $\forall x$ (humain($x$) $\rightarrow$ $\exists z$ (homme($z$)
			$\wedge$ pere($z$,$x$)))
		\end{itemize}
		\begin{itemize}
			\item Hypothèses atomiques donc $FUS$
			\item Gardées donc $GBTS$
		\end{itemize}
	\end{column}
	\vline
	\hfill
	\begin{column}{0.45\linewidth}
		\begin{figure}
			\begin{tikzpicture}[
				auto,
				inner sep=1mm
			]
			\node[comp] (C1) at (0,-1.5) {$FUS,GBTS$};
			\node[comp] (C2) at (0,1.5) {$C_2:R_3,R_4$} 
				edge[ruleedger,bend left=0] (C1)
			;
			\end{tikzpicture}
		\caption{Composantes fortement connexes}
		\end{figure}
	\end{column}
	\end{columns}
\end{frame}

\begin{frame}
	\frametitle{Exemple}
	\begin{columns}
	\begin{column}{0.45\linewidth}
		$C_2$ :
		\begin{itemize}
			\item $R_3$ : $\forall x \forall y$ (parent($x$,$y$) $\wedge$ homme($x$)
			$\rightarrow$ pere($x$,$y$))
			\item $R_4$ : $\forall x \forall y$ (pere($x$,$y$) $\rightarrow$
			parent($x$,$y$))
		\end{itemize}
		Classes :
		\begin{itemize}
			\item Portée réduite donc $FES$
			\item Domaine restreint donc $FUS$
		\end{itemize}
	\end{column}
	\vline
	\hfill
	\begin{column}{0.45\linewidth}
		\begin{figure}
			\begin{tikzpicture}[
				auto,
				inner sep=1mm
			]
			\node[comp] (C1) at (0,-1.5) {$FUS,GBTS$};
			\node[comp] (C2) at (0,1.5) {$FES,FUS$} 
				edge[ruleedger,bend left=0] (C1)
			;
			\end{tikzpicture}
		\caption{Composantes fortement connexes}
		\end{figure}
	\end{column}
	\end{columns}
\end{frame}

\begin{frame}
	\frametitle{Exemple}
	\begin{columns}
	\begin{column}{0.45\linewidth}
		Algorithmes à utiliser :
		\begin{itemize}
			\item $C_1$ : chaînage avant ou arrière
			\item $C_2$ : chaînage arrière
		\end{itemize}
	\end{column}
	\vline
	\hfill
	\begin{column}{0.45\linewidth}
		\begin{figure}
			\begin{tikzpicture}[
				auto,
				inner sep=1mm
			]
			\node[comp] (C1) at (0,-1.5) {$GBTS,FUS$};
			\node[comp] (C2) at (0,1.5) {$FUS$} 
				edge[ruleedger,bend left=0] (C1)
			;
			\end{tikzpicture}
		\caption{Composantes fortement connexes}
		\end{figure}
	\end{column}
	\end{columns}
\end{frame}


