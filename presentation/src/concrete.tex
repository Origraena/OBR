% concrete.tex

\begin{frame}
	\frametitle{Classes de règles concrètes}
	\vspace{10mm}
	\begin{itemize}
		\item Imposent des contraintes sur la forme des règles ou de la base
		\item Spécialisent les classes abstraites
		\item Classes pouvant être comparables
	\end{itemize}
\end{frame}

\begin{frame}
	\frametitle{Guardée}
	\begin{itemize}
		\item Un atome de l'hypothèse contient toutes les variables de celle-ci
		\item $\exists a \in H_i : variable(H_i) \subseteq variables(a)$
		\item Simple à vérifier
		\item $guarded(R) = \{\forall R_i \in R : R_i $ est gardée$\} \in GBTS$
	\end{itemize}
	\vspace{10mm}
\end{frame}

\begin{frame}
	\frametitle{Guardée}
	\begin{itemize}
		\item Un atome de l'hypothèse contient toutes les variables de celle-ci
		\item $\exists a \in H_i : variable(H_i) \subseteq variables(a)$
		\item Simple à vérifier
		\item $guarded(R) = \{\forall R_i \in R : R_i $ est gardée$\} \in GBTS$
	\end{itemize}
	\vspace{10mm}
	\begin{exampleblock}{Exemple}
		\ruleC \\
		Garde : parent($x$,$y$)
	\end{exampleblock}
\end{frame}

\begin{frame}
	\frametitle{Frontière gardée}
	\begin{itemize}
		\item Un atome de l'hypothèse contient toutes les variables de la frontière
		\item $\exists a \in H :$ frontière($R$) $\subseteq variables(a)$
		\item Seule la frontière influe sur l'application d'une règle
		\item Généralisation des règles gardées
		\item $fr-guarded(R) = \{\forall R_i \in R : R_i $ a une frontière gardée$\} \in GBTS$
	\end{itemize}
	\vspace{10mm}
\end{frame}

\begin{frame}
	\frametitle{Frontière gardée}
	\begin{itemize}
		\item Un atome de l'hypothèse contient toutes les variables de la frontière
		\item $\exists a \in H :$ frontière($R$) $\subseteq variables(a)$
		\item Seule la frontière influe sur l'application d'une règle
		\item Généralisation des règles gardées
		\item $fr-guarded(R) = \{\forall R_i \in R : R_i $ a une frontière gardée$\} \in GBTS$
	\end{itemize}
	\vspace{10mm}
	\begin{exampleblock}{Exemple}
		$\forall x \forall y (p(x) \wedge q(y) \rightarrow \exists z (r(y,z)))$\\
		Garde-frontière : $q(y)$
	\end{exampleblock}
\end{frame}

\begin{frame}
	\frametitle{Frontière de taille 1}
	\begin{itemize}
		\item La frontière de la règle est de taille 1
		\item Spécialisation des règles à frontière gardée
		\item Utiles pour les notions d'héritage
		\item $fr-1(R) = \{\forall R_i \in R : |$frontière$(R_i)| = 1 \} \in GBTS$
	\end{itemize}
	\vspace{10mm}
\end{frame}

\begin{frame}
	\frametitle{Frontière de taille 1}
	\begin{itemize}
		\item La frontière de la règle est de taille 1
		\item Spécialisation des règles à frontière gardée
		\item Utiles pour les notions d'héritage
		\item $fr-1(R) = \{\forall R_i \in R : |$frontière$(R_i)| = 1 \} \in GBTS$
	\end{itemize}
	\vspace{10mm}
	\begin{exampleblock}{Exemple}
		\ruleB \\
		frontière$(R_2) = \{x\}$
	\end{exampleblock}
\end{frame}

\begin{frame}
	\frametitle{Hypothèse atomique}
	\begin{itemize}
		\item L'hypothèse de la règle ne contient qu'un seul atome
		\item Spécialisation des règles gardées
		\item $ah(R) = \{\forall R_i = (H_i,C_i) \in R : |H_i| = 1 \} \in GBTS \cap FUS$
	\end{itemize}
	\vspace{10mm}
\end{frame}

\begin{frame}
	\frametitle{Hypothèse atomique}
	\begin{itemize}
		\item L'hypothèse de la règle ne contient qu'un seul atome
		\item Spécialisation des règles gardées
		\item $ah(R) = \{\forall R_i = (H_i,C_i) \in R : |H_i| = 1 \} \in GBTS \cap FUS$
	\end{itemize}
	\vspace{10mm}
	\begin{exampleblock}{Exemple}
		\ruleA
	\end{exampleblock}
\end{frame}

\begin{frame}
	\frametitle{Domaine restreint}
	\begin{itemize}
		\item Les atomes de la conclusion contiennent soit toutes les variables de
		l'hypothèse, soit aucune
		\item $\forall a_j \in R_i (variables(H_i) \subseteq variables(a_j)) \vee 
		(variables(H_i) \cap variables(a_j) = \emptyset)$
		\item $dr(R) = \{\forall R_i \in R : R_i$ a un domaine restreint $\} \in FUS$
	\end{itemize}
	\vspace{10mm}
\end{frame}

\begin{frame}
	\frametitle{Domaine restreint}
	\begin{itemize}
		\item Les atomes de la conclusion contiennent soit toutes les variables de
		l'hypothèse, soit aucune
		\item $\forall a_j \in R_i (variables(H_i) \subseteq variables(a_j)) \vee 
		(variables(H_i) \cap variables(a_j) = \emptyset)$
		\item $dr(R) = \{\forall R_i \in R : R_i$ a un domaine restreint $\} \in FUS$
	\end{itemize}
	\vspace{10mm}
	\begin{exampleblock}{Exemple}
		\ruleB 
		\begin{itemize}
			\item $variables(homme(z)) \cap variables(H_2) = \emptyset$
			\item $variables(H_2) \subseteq variables(pere(z,x))$
		\end{itemize}
	\end{exampleblock}
\end{frame}

\begin{frame}
	\frametitle{Déconnectée}
	\begin{itemize}
		\item Frontière vide
		\item Spécialisation des règles de domaine restreint
		\item Une seule application nécessaire
		\item Partage possible de constantes entre l'hypothèse et la conclusion
		\item $disc(R) = \{\forall R_i \in R : $frontière$(R_i) = \emptyset \} \in FUS$
	\end{itemize}
	\vspace{10mm}
\end{frame}

\begin{frame}
	\frametitle{Déconnectée}
	\begin{itemize}
		\item Frontière vide
		\item Spécialisation des règles de domaine restreint
		\item Une seule application nécessaire
		\item Partage possible de constantes entre l'hypothèse et la conclusion
		\item $disc(R) = \{\forall R_i \in R : $frontière$(R_i) = \emptyset \} \in FUS$
	\end{itemize}
	\vspace{10mm}
	\begin{exampleblock}{Exemple}
		$\forall x (p(x) \wedge q(a) \rightarrow \exists z (r(a,z)))$
	\end{exampleblock}
\end{frame}

\begin{frame}
	\frametitle{Universelle}
	\begin{itemize}
		\item Aucune variable existentielle
		\item Couramment utilisées
		\item $rr(R) = \{\forall R_i \in R : variables(R_i) \subseteq variables(H_i) \}
		\in FES \cap GBTS$
	\end{itemize}
	\vspace{10mm}
\end{frame}

\begin{frame}
	\frametitle{Universelle}
	\begin{itemize}
		\item Aucune variable existentielle
		\item Couramment utilisées
		\item $rr(R) = \{\forall R_i \in R : variables(R_i) \subseteq variables(H_i) \}
		\in FES \cap GBTS$
	\end{itemize}
	\vspace{10mm}
	\begin{exampleblock}{Exemple}
		\ruleE
	\end{exampleblock}
\end{frame}

\begin{frame}
	\frametitle{Faiblement acyclique}
	\begin{itemize}
		\item Contrainte sur l'ensemble des règles
		\item Nécessite l'usage d'une nouvelle structure : le graphe de dépendances des
		positions
		\item $wa(R) \in FES$
	\end{itemize}
\end{frame}

\begin{frame}
	\frametitle{Faiblement acyclique}
	\begin{itemize}
		\item Contrainte sur l'ensemble des règles
		\item Nécessite l'usage d'une nouvelle structure : le graphe de dépendances des
		positions
		\item $wa(R) \in FES$
	\end{itemize}
	\begin{exampleblock}{Exemple de non faible acyclicité}
		\ruleA \\
		\ruleB \\
	\end{exampleblock}
\end{frame}

\begin{frame}
	\frametitle{Faiblement acyclique}
	\begin{itemize}
		\item Contrainte sur l'ensemble des règles
		\item Nécessite l'usage d'une nouvelle structure : le graphe de dépendances des
		positions
		\item $wa(R) \in FES$
	\end{itemize}
	\begin{exampleblock}{Exemple de non faible acyclicité}
		\ruleA \\
		\ruleB \\
	\end{exampleblock}
	\begin{exampleblock}{Exemple de faible acyclicité}
		\ruleC \\
		\ruleD \\
	\end{exampleblock}
\end{frame}

\begin{frame}
	\frametitle{Sticky}
	\begin{itemize}
		\item Contrainte sur l'ensemble des règles
		\item Marquage des variables
		\item Une variable marquée ne doit pas apparaître plusieurs fois dans l'hypothèse
		d'une règle
		\item $sticky(R) \in FES$
	\end{itemize}
	\vspace{10mm}
\end{frame}

\begin{frame}
	\frametitle{Sticky}
	\begin{itemize}
		\item Contrainte sur l'ensemble des règles
		\item Marquage des variables
		\item Une variable marquée ne doit pas apparaître plusieurs fois dans l'hypothèse
		d'une règle
		\item $sticky(R) \in FES$
	\end{itemize}
	\vspace{10mm}
	\begin{exampleblock}{Exemple}
		\ruleC \\
		\ruleD \\
	\end{exampleblock}
\end{frame}

\begin{frame}
	\frametitle{Weakly sticky}
	\begin{itemize}
		\item Généralisation de faiblement acyclique et de sticky
		\item Les variables marquées ne doivent pas apparaître plusieurs fois dans
		l'hypothèse ou ne pas être dans une position de rang infini
		\item $ws(R) \notin FES \cup GBTS \cup FUS$
	\end{itemize}
\end{frame}

\begin{frame}
	\frametitle{Graphe de dépendances des règles acyclique}
	\begin{itemize}
		\item L'ensemble des règles forment un graphe de dépendances sans circuit
		\item $aGRD(R) \in FES \cup FUS$
	\end{itemize}
	\vspace{10mm}
\end{frame}

\begin{frame}
	\frametitle{Graphe de dépendances des règles acyclique}
	\begin{itemize}
		\item L'ensemble des règles forment un graphe de dépendances sans circuit
		\item $aGRD(R) \in FES \cup FUS$
	\end{itemize}
	\vspace{10mm}
	\begin{exampleblock}{Exemple}
		\ruleB \\
		\ruleD
	\end{exampleblock}
\end{frame}

\begin{frame}
	\frametitle{Schéma récapitulatif}
\begin{figure}[H]
\begin{center}
\begin{tikzpicture}[
	auto,
	node distance=1.5cm,
	inner sep=1mm,
	scale=0.95
]
	\draw [blue] (-3.8,-5.5) -- (3.2,-5.5) -- (3.2,-2.5) -- (-3.8,-2.5) -- (-3.8,-5.5);
	\draw [red] (0.3,1) -- (3.7,1) -- (3.7,-5.7) -- (0.3,-5.7) -- (0.3,1);
	\draw [green] (4,1) -- (9,1) -- (9,-4.5) -- (-2.2,-4.5) -- (-2.2,-3.5) --
	(0.5,-3.5) -- (0.5,-2) -- (4,-2) -- (4,1);

	\node [blue] (fes) at (-3,-5) {FES};
	\node [red] (fus) at (1,0.5) {FUS};
	\node [green] (gbts) at (8,0.5) {GBTS};

	% terms
	\node [class]	(ws)		at (-2,0)	{weakly sticky};
	\node [class]	(dr)		at (2,-1)	{domaine restreint};
	\node [class]	(agrd)		at (2,-5)	{grd acyclique};
	\node [class]	(fg)		at (6,0)	{frontière guardée};
	\node [class]	(sticky)	at (2,0)	{sticky}
		edge[classedge]	(ws);
	\node [class]	(wa)		at (-2,-3)	{faiblement acyclique}
		edge[classedge]	(ws);
	\node [class]	(disc)		at (2,-3)	{déconnectée}
		edge[classedge] (wa)
		edge[classedge] (dr)
		edge[classedge] (fg);
	\node [class]	(guarded)	at (6,-2)	{guardée}
		edge[classedge]	(fg);
	\node [class]	(fr1)		at (8,-2)	{frontière 1}
		edge[classedge] (fg);
	\node [class]	(ah)		at (6,-4)	{hypothèse atomique}
		edge[classedge] (guarded);
	\node [class]	(rr)		at (-1,-4)	{universelles}
		edge[classedge] (wa);
\end{tikzpicture}
\end{center}
\end{figure}
\end{frame}

% FIGURE => Tableau des complexités

