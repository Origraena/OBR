% algos.tex

\subsection{Requêtes}

\begin{frame}
	\frametitle{Base de connaissance}
	\begin{itemize}
		\item Notée $K = (F,R)$
		\item Ensemble de faits représenté par une conjonction d'atomes $F$
		\item Ensemble de règles $R$
		\item Requête $Q$ : conjonction d'atomes existentiellement fermée
	\end{itemize}
\end{frame}

\begin{frame}
	\frametitle{Chaînage avant}
	\begin{itemize}
		\item Génération de nouveaux faits à partir des précédents et de l'ontologie
		\item A chaque création de fait, vérification de la présence de la requête dans
		$F$
		\item Arrêt lorsque tous les faits sont générés ou si réponse positive
	\end{itemize}
	\begin{exampleblock}{Exemple}
	Base :
	\begin{itemize}
		\item "{\em Jean} est un des {\em parents} de {\em Tom}"
		\item "{\em Jean} est un {\em homme}"
		\item "{\bf Si} un {\em homme} est le {\em parent} de {\em quelqu'un}, {\bf
		alors} {\em il} est son {\em père}."\\
	\end{itemize}
	Requête : "Jean est-il le père de Tom ?"\\
	\end{exampleblock}
\end{frame}

\begin{frame}
	\frametitle{Chaînage avant}
	\begin{itemize}
		\item Génération de nouveaux faits à partir des précédents et de l'ontologie
		\item A chaque création de fait, vérification de la présence de la requête dans
		$F$
		\item Arrêt lorsque tous les faits sont générés ou si réponse positive
	\end{itemize}

	\begin{exampleblock}{Exemple}
	Base :
	\begin{itemize}
		\item $F = parent(Jean,Tom) \wedge homme(Jean)$
		\item $R = \forall x \forall y (homme(x) \wedge parent(x,y) \rightarrow
		$père$(x,y))$
	\end{itemize}
	Requête : père$(Jean,Tom)$
	\begin{itemize}
		\item Application de l'unique règle
	\end{itemize}
	\end{exampleblock}
\end{frame}


\begin{frame}
	\frametitle{Chaînage avant}
	\begin{itemize}
		\item Génération de nouveaux faits à partir des précédents et de l'ontologie
		\item A chaque création de fait, vérification de la présence de la requête dans
		$F$
		\item Arrêt lorsque tous les faits sont générés ou si réponse positive
	\end{itemize}

	\begin{exampleblock}{Exemple}
	Base :
	\begin{itemize}
		\item $F = parent(Jean,Tom) \wedge homme(Jean)$
		\item $R = \forall x \forall y (homme(x) \wedge parent(x,y) \rightarrow
		$père$(x,y))$
	\end{itemize}
	Requête : père$(Jean,Tom)$
	\begin{itemize}
		\item Application de l'unique règle
		\item $x \leftarrow Jean, y \leftarrow Tom$
	\end{itemize}
	\end{exampleblock}
\end{frame}


\begin{frame}
	\frametitle{Chaînage avant}
	\begin{itemize}
		\item Génération de nouveaux faits à partir des précédents et de l'ontologie
		\item A chaque création de fait, vérification de la présence de la requête dans
		$F$
		\item Arrêt lorsque tous les faits sont générés ou si réponse positive
	\end{itemize}

	\begin{exampleblock}{Exemple}
	Base :
	\begin{itemize}
		\item $F = parent(Jean,Tom) \wedge homme(Jean)$
		\item $R = \forall x \forall y (homme(x) \wedge parent(x,y) \rightarrow
		$père$(x,y))$
	\end{itemize}
	Requête : père$(Jean,Tom)$
	\begin{itemize}
		\item Application de l'unique règle 
		\item $x \leftarrow Jean, y \leftarrow Tom$
		\item $F = parent(Jean,Tom) \wedge homme(Jean) \wedge $père$(Jean,Tom)$
	\end{itemize}
	\end{exampleblock}
\end{frame}


\begin{frame}
	\frametitle{Chaînage avant}
	\begin{itemize}
		\item Génération de nouveaux faits à partir des précédents et de l'ontologie
		\item A chaque création de fait, vérification de la présence de la requête dans
		$F$
		\item Arrêt lorsque tous les faits sont générés ou si réponse positive
	\end{itemize}

	\begin{exampleblock}{Exemple}
	Base :
	\begin{itemize}
		\item $F = parent(Jean,Tom) \wedge homme(Jean)$
		\item $R = \forall x \forall y (homme(x) \wedge parent(x,y) \rightarrow
		$père$(x,y))$
	\end{itemize}
	Requête : père$(Jean,Tom)$
	\begin{itemize}
		\item Application de l'unique règle
		\item $x \leftarrow Jean, y \leftarrow Tom$
		\item $F = parent(Jean,Tom) \wedge homme(Jean) \wedge $père$(Jean,Tom)$
		\item $Q \in F \rightarrow$ réponse positive !
	\end{itemize}
	\end{exampleblock}
\end{frame}

\begin{frame}
	\frametitle{Chaînage arrière}
	\begin{itemize}
		\item Réécriture de la requête via l'ontologie
		\item A chaque réécriture, vérification de la présence d'une de celles-ci dans
		$F$
		\item Arrêt lorsque celle-ci ne peut plus être réécrire ou si réponse positive
	\end{itemize}
	\begin{exampleblock}{Exemple}
	Base :
	\begin{itemize}
		\item "{\em Jean} est un des {\em parents} de {\em Tom}"
		\item "{\em Jean} est un {\em homme}"
		\item "{\bf Si} un {\em homme} est le {\em parent} de {\em quelqu'un}, {\bf
		alors} {\em il} est son {\em père}."\\
	\end{itemize}
	Requête : "Jean est-il le père de Tom ?"\\
	\end{exampleblock}
\end{frame}

\begin{frame}
	\frametitle{Chaînage arrière}
	\begin{itemize}
		\item Réécriture de la requête via l'ontologie
		\item A chaque réécriture, vérification de la présence d'une de celles-ci dans
		$F$
		\item Arrêt lorsque celle-ci ne peut plus être réécrire ou si réponse positive
	\end{itemize}
	\begin{exampleblock}{Exemple}
	Base :
	\begin{itemize}
		\item $F = parent(Jean,Tom) \wedge homme(Jean)$
		\item $R = \forall x \forall y (homme(x) \wedge parent(x,y) \rightarrow
		$père$(x,y))$
	\end{itemize}
	Requête : $Q = $ père$(Jean,Tom)$\\
	\begin{itemize}
		\item Réécriture via l'unique règle
	\end{itemize}
	\end{exampleblock}
\end{frame}


\begin{frame}
	\frametitle{Chaînage arrière}
	\begin{itemize}
		\item Réécriture de la requête via l'ontologie
		\item A chaque réécriture, vérification de la présence d'une de celles-ci dans
		$F$
		\item Arrêt lorsque celle-ci ne peut plus être réécrire ou si réponse positive
	\end{itemize}
	\begin{exampleblock}{Exemple}
	Base :
	\begin{itemize}
		\item $F = parent(Jean,Tom) \wedge homme(Jean)$
		\item $R = \forall x \forall y (homme(x) \wedge parent(x,y) \rightarrow
		$père$(x,y))$
	\end{itemize}
	Requête : $Q = $ père$(Jean,Tom)$\\
	\begin{itemize}
		\item Réécriture via l'unique règle
		\item $Jean \rightarrow x, Tom \rightarrow y$
	\end{itemize}
	\end{exampleblock}
\end{frame}


\begin{frame}
	\frametitle{Chaînage arrière}
	\begin{itemize}
		\item Réécriture de la requête via l'ontologie
		\item A chaque réécriture, vérification de la présence d'une de celles-ci dans
		$F$
		\item Arrêt lorsque celle-ci ne peut plus être réécrire ou si réponse positive
	\end{itemize}
	\begin{exampleblock}{Exemple}
	Base :
	\begin{itemize}
		\item $F = parent(Jean,Tom) \wedge homme(Jean)$
		\item $R = \forall x \forall y (homme(x) \wedge parent(x,y) \rightarrow
		$père$(x,y))$
	\end{itemize}
	Requête : $Q = $ père$(Jean,Tom)$\\
	\begin{itemize}
		\item Réécriture via l'unique règle
		\item $Jean \rightarrow x, Tom \rightarrow y$
		\item $Q' = \{Q_0 = $père$(Jean,Tom)$, $Q_1 = parent(Jean,Tom) \wedge
		homme(Jean) \}$
	\end{itemize}
	\end{exampleblock}
\end{frame}


\begin{frame}
	\frametitle{Chaînage arrière}
	\begin{itemize}
		\item Réécriture de la requête via l'ontologie
		\item A chaque réécriture, vérification de la présence d'une de celles-ci dans
		$F$
		\item Arrêt lorsque celle-ci ne peut plus être réécrire ou si réponse positive
	\end{itemize}
	\begin{exampleblock}{Exemple}
	Base :
	\begin{itemize}
		\item $F = parent(Jean,Tom) \wedge homme(Jean)$
		\item $R = \forall x \forall y (homme(x) \wedge parent(x,y) \rightarrow
		$père$(x,y))$
	\end{itemize}
	Requête : $Q = $ père$(Jean,Tom)$\\
	\begin{itemize}
		\item Réécriture via l'unique règle
		\item $Jean \rightarrow x, Tom \rightarrow y$
		\item $Q' = \{Q_0 = $père$(Jean,Tom)$, $Q_1 = parent(Jean,Tom) \wedge
		homme(Jean) \}$
		\item $Q_1 \in F \rightarrow$ réponse positive !
	\end{itemize}
	\end{exampleblock}
\end{frame}


