% contexte.tex

\begin{frame}
	\frametitle{Contexte}
	\begin{itemize}
		\item Base de données très utilisées dans le monde de l'informatique
		\item Effectuer des requêtes sur ces bases
		\item Généralement, seuls des faits sont pris en compte
	\end{itemize}
	\begin{exampleblock}{Exemple}
		Base :
		\begin{itemize}
			\item "{\em Jean} est un des {\em parents} de {\em Tom}"
			\item "{\em Jean} est un {\em homme}"
		\end{itemize}
		Requête : "Jean est-il le père de Tom ?"\\
		Réponse : NON.
	\end{exampleblock}
\end{frame}

\begin{frame}
	\frametitle{Contexte}
	\begin{itemize}
		\item Base de données très utilisées dans le monde de l'informatique
		\item Effectuer des requêtes sur ces bases
		\item Généralement, seuls des faits sont pris en compte
		\item Ajout d'un ensemble de {\em règles}
	\end{itemize}
	\begin{exampleblock}{Exemple}
		Base :
		\begin{itemize}
			\item "{\em Jean} est un des {\em parents} de {\em Tom}"
			\item "{\em Jean} est un {\em homme}"
			\item "{\bf Si} un {\em homme} est le {\em parent} de {\em quelqu'un}, {\bf
			alors} {\em il} est son {\em père}." 
		\end{itemize}
		Requête : "Jean est-il le père de Tom ?"\\
		Réponse : OUI.
	\end{exampleblock}
\end{frame}

\begin{frame}
	\frametitle{Contexte}
	\begin{itemize}
		\item Indécidable de manière générale
		\item Nécessaire d'ajouter des contraintes (classes de règles) 
		\item Les contraintes influent sur les méthodes de réponse
	\end{itemize}
\end{frame}


