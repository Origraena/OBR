\begin{frame}
	\frametitle{Construction du graphe}
	\begin{exampleblock}{Exemple}
		"Tout homme est un humain.
		Tout humain a un père qui est un homme.
		Si un homme est le parent d'un autre, alors il est son père.
		Tout père d'un individu est un de ses parents."

		\begin{itemize}
			\item $R_1$ : $\forall x$ (homme($x$) $\rightarrow$ humain($x$))
			\item $R_2$ : $\forall x$ (humain($x$) $\rightarrow$ $\exists z$ (homme($z$)
			$\wedge$ pere($z$,$x$)))
			\item $R_3$ : $\forall x \forall y$ (parent($x$,$y$) $\wedge$ homme($x$)
			$\rightarrow$ pere($x$,$y$))
			\item $R_4$ : $\forall x \forall y$ (pere($x$,$y$) $\rightarrow$
			parent($x$,$y$))
		\end{itemize}
	\end{exampleblock}
\end{frame}

\begin{frame}
	\frametitle{Construction du graphe}
	\begin{columns}
	\begin{column}{0.55\linewidth}
		Base de règles :
		\begin{itemize}
			\item $R_1$ : $\forall x$ (homme($x$) $\rightarrow$ humain($x$))
			\item $R_2$ : $\forall x$ (humain($x$) $\rightarrow$ $\exists z$ (homme($z$)
			$\wedge$ pere($z$,$x$)))
			\item $R_3$ : $\forall x \forall y$ (parent($x$,$y$) $\wedge$ homme($x$)
			$\rightarrow$ pere($x$,$y$))
			\item $R_4$ : $\forall x \forall y$ (pere($x$,$y$) $\rightarrow$
			parent($x$,$y$))
		\end{itemize}
		\vfill
	\end{column}
	\vline
	\hfill
	\begin{column}{0.35\linewidth}
		\begin{figure}
			\begin{tikzpicture}[
				auto,
				inner sep=1mm
			]
			\node[rule] (R1) at (0,0) {$R_1$};
			\node[rule] (R2) at (3,0) {$R_2$};
			\node[rule] (R3) at (0,3) {$R_3$};
			\node[rule] (R4) at (3,3) {$R_4$};
			\end{tikzpicture}
		\end{figure}
	\end{column}
	\end{columns}
\end{frame}

\begin{frame}
	\frametitle{Construction du graphe}
	\begin{columns}
	\begin{column}{0.55\linewidth}
		Base de règles :
		\begin{itemize}
			\item $R_1$ : $\forall x$ (homme($x$) $\rightarrow$ humain($x$))
			\item $R_2$ : $\forall x$ (humain($x$) $\rightarrow$ $\exists z$ (homme($z$)
			$\wedge$ pere($z$,$x$)))
			\item $R_3$ : $\forall x \forall y$ (parent($x$,$y$) $\wedge$ homme($x$)
			$\rightarrow$ pere($x$,$y$))
			\item $R_4$ : $\forall x \forall y$ (pere($x$,$y$) $\rightarrow$
			parent($x$,$y$))
		\end{itemize}
		\vfill
		$R_1$ peut elle se redéclencher ?\\
		$C_1$ a un prédicat différent de $H_1$
	\end{column}
	\vline
	\hfill
	\begin{column}{0.35\linewidth}
		\begin{figure}
			\begin{tikzpicture}[
				auto,
				inner sep=1mm
			]
			\node[rule] (R1) at (0,0) {$R_1$};
			\node[rule] (R2) at (3,0) {$R_2$} 
			;
			\node[rule] (R3) at (0,3) {$R_3$};
			\node[rule] (R4) at (3,3) {$R_4$}
			;
			\end{tikzpicture}
		\end{figure}
	\end{column}
	\end{columns}
\end{frame}


\begin{frame}
	\frametitle{Construction du graphe}
	\begin{columns}
	\begin{column}{0.55\linewidth}
		Base de règles :
		\begin{itemize}
			\item $R_1$ : $\forall x$ (homme($x$) $\rightarrow$ humain($x$))
			\item $R_2$ : $\forall x$ (humain($x$) $\rightarrow$ $\exists z$ (homme($z$)
			$\wedge$ pere($z$,$x$)))
			\item $R_3$ : $\forall x \forall y$ (parent($x$,$y$) $\wedge$ homme($x$)
			$\rightarrow$ pere($x$,$y$))
			\item $R_4$ : $\forall x \forall y$ (pere($x$,$y$) $\rightarrow$
			parent($x$,$y$))\\
		\end{itemize}
		\vfill
		$R_1$ peut amener à déclencher $R_2$ ?\\
		Unification de $C_1$ avec $R_1$
	\end{column}
	\vline
	\hfill
	\begin{column}{0.35\linewidth}
		\begin{figure}
			\begin{tikzpicture}[
				auto,
				inner sep=1mm
			]
			\node[rule] (R1) at (0,0) {$R_1$};
			\node[rule] (R2) at (3,0) {$R_2$} 
				edge[ruleedger] (R1)
			;
			\node[rule] (R3) at (0,3) {$R_3$};
			\node[rule] (R4) at (3,3) {$R_4$}
			;
			\end{tikzpicture}
		\end{figure}
	\end{column}
	\end{columns}
\end{frame}

\begin{frame}
	\frametitle{Construction du graphe}
	\begin{columns}
	\begin{column}{0.55\linewidth}
		Base de règles :
		\begin{itemize}
			\item $R_1$ : $\forall x$ (homme($x$) $\rightarrow$ humain($x$))
			\item $R_2$ : $\forall x$ (humain($x$) $\rightarrow$ $\exists z$ (homme($z$)
			$\wedge$ pere($z$,$x$)))
			\item $R_3$ : $\forall x \forall y$ (parent($x$,$y$) $\wedge$ homme($x$)
			$\rightarrow$ pere($x$,$y$))
			\item $R_4$ : $\forall x \forall y$ (pere($x$,$y$) $\rightarrow$
			parent($x$,$y$))
		\end{itemize}
		\vfill
		$R_2$ peut amener à déclencher $R_1$ ?\\
		$R_2$ amène l'existence d'un nouvel individu et	
		l'hypothèse de $R_1$ est vérifiée pour celui ci
	\end{column}
	\vline
	\hfill
	\begin{column}{0.35\linewidth}
		\begin{figure}
			\begin{tikzpicture}[
				auto,
				inner sep=1mm
			]
			\node[rule] (R1) at (0,0) {$R_1$};
			\node[rule] (R2) at (3,0) {$R_2$} 
				edge[ruleedger] (R1)
				edge[ruleedge] (R1)
			;
			\node[rule] (R3) at (0,3) {$R_3$};
			\node[rule] (R4) at (3,3) {$R_4$}
			;
			\end{tikzpicture}
		\end{figure}
	\end{column}
	\end{columns}
\end{frame}



\begin{frame}
	\frametitle{Construction du graphe}
	\begin{columns}
	\begin{column}{0.55\linewidth}
		Base de règles :
		\begin{itemize}
			\item $R_1$ : $\forall x$ (homme($x$) $\rightarrow$ humain($x$))
			\item $R_2$ : $\forall x$ (humain($x$) $\rightarrow$ $\exists z$ (homme($z$)
			$\wedge$ pere($z$,$x$)))
			\item $R_3$ : $\forall x \forall y$ (parent($x$,$y$) $\wedge$ homme($x$)
			$\rightarrow$ pere($x$,$y$))
			\item $R_4$ : $\forall x \forall y$ (pere($x$,$y$) $\rightarrow$
			parent($x$,$y$))
		\end{itemize}
	\end{column}
	\vline
	\hfill
	\begin{column}{0.35\linewidth}
		\begin{figure}
			\begin{tikzpicture}[
				auto,
				inner sep=1mm
			]
			\node[rule] (R1) at (0,0) {$R_1$};
			\node[rule] (R2) at (3,0) {$R_2$} 
				edge[ruleedge] (R1)
				edge[ruleedger] (R1)
			;
			\node[rule] (R3) at (0,3) {$R_3$};
			\node[rule] (R4) at (3,3) {$R_4$}
				edge[ruleedger,bend left=0] (R2)
				edge[ruleedge] (R3)
				edge[ruleedger] (R3)
			;
			\end{tikzpicture}
		\end{figure}
	\end{column}

	\end{columns}
\end{frame}

