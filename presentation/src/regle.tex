% regle.tex

\subsection{Règles}
\begin{frame}[t]
	\frametitle{Règle}
	\begin{center}
	\begin{itemize}
		\item Deux conjonctions d'atomes : une hypothèse H et une conclusion C
		\item $H \rightarrow C$
		\item Variable $x$ soit universelle ($x \in H$) ou existentielle ($x \notin H$)
		\item Frontière : variable à la fois dans H et dans C ($x \in H \cap C$)\\
	\end{itemize}
	\end{center}
\end{frame}

\begin{frame}[t]
	\frametitle{Règle}
	\begin{center}
	\begin{itemize}
		\item Deux conjonctions d'atomes : une hypothèse H et une conclusion C
		\item $H \rightarrow C$
		\item Variable $x$ soit universelle ($x \in H$) ou existentielle ($x \notin H$)
		\item Frontière : variable à la fois dans H et dans C ($x \in H \cap C$)\\
	\end{itemize}

	\begin{exampleblock}{Exemple règle universelle}
		"Tout homme est un humain."\\
		$\forall x$ (homme($x$) $\rightarrow$ humain($x$))
	\end{exampleblock}

	\end{center}
\end{frame}

\begin{frame}[t]
	\frametitle{Règle}
	\begin{center}
	\begin{itemize}
		\item Deux conjonctions d'atomes : une hypothèse H et une conclusion C
		\item $H \rightarrow C$
		\item Variable $x$ soit universelle ($x \in H$) ou existentielle ($x \notin H$)
		\item Frontière : variable à la fois dans H et dans C ($x \in H \cap C$)\\
	\end{itemize}

	\begin{exampleblock}{Exemple règle universelle}
		"Tout homme est un humain."\\
		$\forall x$ (homme($x$) $\rightarrow$ humain($x$))
	\end{exampleblock}

	\begin{exampleblock}{Exemple règle existentielle}
		"Tout humain a un père qui est un homme."\\
		$\forall x$ (humain($x$) $\rightarrow \exists z$ (homme($z$) $\wedge$ pere($z$,$x$)))
	\end{exampleblock}
	\end{center}
\end{frame}

\begin{frame}
	\frametitle{Représentation graphique d'une règle}
	\begin{center}
	\begin{itemize}
		\item Ensembles de sommets et d'arêtes identiques à une conjonction d'atomes
		\item 2-coloration des atomes pour différencier hypothèse et conclusion
	\end{itemize}
	\vspace{10mm}
	\begin{exampleblock}{Exemple}
		\begin{center}
		"Tout humain a un père qui est un homme."\\
		$\forall x$ (humain($x$) $\rightarrow \exists z$ (homme($z$) $\wedge$
		pere($z$,$x$)))
		\end{center}
	\end{exampleblock}
	\end{center}
\end{frame}

\begin{frame}
	\frametitle{Représentation graphique d'une règle}

	\begin{columns}
	
	\begin{column}{0.45\linewidth}
		\begin{itemize}
			\item un sommet par atome étiqueté par son prédicat \checkmark
			\item un sommet par terme étiqueté si constante \checkmark
			\item une arête pour chaque apparition de terme dans un atome 
			dont le poids est la position du terme \checkmark
			\item coloration des sommets atomes en fonction de leur position
		\end{itemize}
	\end{column}
	\vline
	\hfill
	\begin{column}{0.45\linewidth}
		$\forall x$ (humain($x$) $\rightarrow \exists z$ (homme($z$) $\wedge$
		pere($z$,$x$)))
		\begin{figure}
			\begin{tikzpicture}[
				auto,
				inner sep=1mm
			]
			\node [atom] (humain) at (0,-1.5) {humain$\backslash$1};
			\node [atom] (pere) at (3,0) {pere$\backslash$2};
			\node [atom] (homme) at (-0.2,0) {homme$\backslash$1};
			\node[term]	(z) at (1.5,0) {}
				edge[atomedge] node {1} (pere)
				edge[atomedge] node {1} (homme)
			;
			\node[term] (x) at (3,-1.5) {}
				edge[atomedge] node {1} (humain)
				edge[atomedge] node {2} (pere)
			;
			\end{tikzpicture}
		\end{figure}
	\end{column}
	\end{columns}
\end{frame}

\begin{frame}
	\frametitle{Représentation graphique d'une règle}

	\begin{columns}
	
	\begin{column}{0.45\linewidth}
		\begin{itemize}
			\item un sommet par atome étiqueté par son prédicat \checkmark
			\item un sommet par terme étiqueté si constante \checkmark
			\item une arête pour chaque apparition de terme dans un atome 
			dont le poids est la position du terme \checkmark
			\item coloration des sommets atomes en fonction de leur position \checkmark
		\end{itemize}
	\end{column}
	\vline
	\hfill
	\begin{column}{0.45\linewidth}
		$\forall x$ (humain($x$) $\rightarrow \exists z$ (homme($z$) $\wedge$
		pere($z$,$x$)))
		\begin{figure}
			\begin{tikzpicture}[
				auto,
				inner sep=1mm
			]
			\node [atombody] (humain) at (0,-1.5) {humain$\backslash$1};
			\node [atomhead] (pere) at (3,0) {pere$\backslash$2};
			\node [atomhead] (homme) at (-0.2,0) {homme$\backslash$1};
			\node[term]	(z) at (1.5,0) {}
				edge[atomedge] node {1} (pere)
				edge[atomedge] node {1} (homme)
			;
			\node[term] (x) at (3,-1.5) {}
				edge[atomedge] node {1} (humain)
				edge[atomedge] node {2} (pere)
			;
			\lightatomlegend{0}{-2.5}
			\end{tikzpicture}
		\end{figure}
	\end{column}

	\end{columns}
\end{frame}

\begin{frame}
	\frametitle{Les différents éléments d'une règle}
	\begin{center}
	$\forall x$ (humain($x$) $\rightarrow \exists z$ (homme($z$) $\wedge$
	pere($z$,$x$)))
	\end{center}
	\begin{figure}
		\begin{tikzpicture}[
			auto,
			inner sep=1mm
		]
		\node [atombody] (humain) at (0,-1.5) {humain$\backslash$1};
		\node [atomhead] (pere) at (3,0) {pere$\backslash$2};
		\node [atomhead] (homme) at (-0.2,0) {homme$\backslash$1};
		\node[termhead]	(z) at (1.5,0) {}
			edge[atomedge] node {1} (pere)
			edge[atomedge] node {1} (homme)
		;
		\node[termfr] (x) at (3,-1.5) {}
			edge[atomedge] node {1} (humain)
			edge[atomedge] node {2} (pere)
		;
		\atomlegend{-4}{0}
		\end{tikzpicture}
	\end{figure}
\end{frame}

