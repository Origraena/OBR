\documentclass[a4paper]{article}

% packages
\usepackage[french]{babel}       
\usepackage[utf8]{inputenc}
\usepackage{setspace}
\usepackage{listings}
\usepackage{lscape}
\usepackage{graphicx}
\usepackage{geometry}
\usepackage{hyperref}
\usepackage{url}
\usepackage{enumerate}
\usepackage{amsmath}
\usepackage{amssymb}
\usepackage{tikz}


\pagestyle{headings}
\thispagestyle{empty}
\geometry{a4paper,twoside,left=2.5cm,right=2.5cm,marginparwidth=1.2cm,marginparsep=3mm,top=2.5cm,bottom=2.5cm}
\title{Outils d'Analyse d'une Base de Règles\\Bilan pré-soutenance}

\begin{document}
\setlength{\parskip}{5mm plus2mm minus2mm}
\begin{center}
	\Huge 
	Outils d'Analyse d'une Base de Règles\\
	\huge
	Bilan pré-soutenance\\
	\Large
	27 Avril 2012
\end{center}

\large

{\bfseries Nom du groupe :} OBR

%{\bfseries Encadrante :} Marie-Laure Mugnier

{\bfseries Membres du groupe :} Swan Rocher (swan.rocher@gmail.com)

\section{Articles}\label{articles}
\begin{itemize}
	\item 
	Query Answering under Non-guarded Rules in Datalog+/-, Andrea Calì, Georg Gottlob,
	Andreas Pierris.
	\item 

\end{itemize}

\section{Tâches effectuées}\label{tasks}
\begin{itemize}
	\item implémentation des différents graphes utilisés comme structures de données 
	\item construction du graphe de dépendances des règles
	\item mise en place d'un algorithme d'unification de règles
	\item lecture d'une base à partir d'un fichier
	\item conversion des fichiers Datalog (.dtg)
	\item écriture de la base de règles dans un fichier
	\item construction du graphe des composantes fortement connexes du graphe de
	dépendances des règles
	\item visualisation des graphes via une sortie PostScript 
	\item reconnaissance des classes de règles concrètes
	\item déterminer si une base de règle est décidable, et quels algorithmes utiliser
	sur celle-ci
\end{itemize}

\section{Problèmes rencontrés}\label{problems}
L'un des deux membres du groupe ayant abandonné le projet, des problèmes de finition
ont été rencontrés,
et donc aucune interface graphique n'a été mise en place afin de
pouvoir compléter les autres taches plus prioritaires.


\vfill
{\raggedleft R\'ealis\'e avec \LaTeX{} \par}

\end{document}

